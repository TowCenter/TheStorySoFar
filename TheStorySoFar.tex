\chapter{Introduction}
Few news organizations can match the setting of the Miami Herald. The paper's
headquarters is perched on the edge of Biscayne Bay, offering sweeping views of
the islands that buffer the city of Miami from the Atlantic Ocean. Pelicans and
gulls float near the building; colorful cruise ships ply the waters a few miles away.
And Miami Herald executives long held some of the best views in the city,
from the fifth floor of the company's headquarters.
Not any longer.
The Herald, like most U.S. daily newspapers, has faced severe financial troubles
in recent years, suffering deep cuts in the newsroom and other departments. So,
in one of many efforts to raise revenue, executives attached a billboard to the east
side of the Herald building, completely obscuring the bay views of many newspaper
employees, including the publisher.
The benefits of the billboard are obvious: the low six figures in annual revenue,
according to a Herald executive, or enough to pay the salaries of a few
junior reporters.
The irony is obvious as well, for the advertiser buying the space is Apple—the
company that now controls a commerce and publishing system crucial to the
future of the news business. And the product being advertised on the Herald's
wall is the iPad, a device that is both disruptive and helpful to media economics.
Indeed, the two companies provide a way to see the destruction and creation
in the media business over the past decade. At the end of March 2001, the stock
market valued the Herald's parent company, Knight-Ridder, at almost precisely
the same amount as Apple: \$3.8 billion.
Ten years later, Apple's valuation is more than \$300 billion. And Knight-Ridder
no longer exists as an independent company.^{\href{#endnotes-introduction}{1}}%The Story So Far: What We Know About the Business of Digital Journalism

* * *
The difficult financial state of the U.S. news industry is no longer new. Most
big newsrooms have faced severe cutbacks, and even though online-only outlets
have sprung up in communities throughout the country, they haven't fully taken
the place of what has been lost.
These issues were explored in a precursor report^{\href{#endnotes-introduction}{2}}to this one, sponsored by
Columbia University's Graduate School of Journalism and written by Leonard
Downie Jr., former executive editor of the Washington Post, and Michael Schudson,
a professor at Columbia. At the end of that report, which was published in
late 2009, the authors provided a number of recommendations to stanch the
losses in independent reporting.
Most of the recommendations were based in policy, including changes in the
tax code to provide news organizations easier access to nonprofit status and
encouraging philanthropists to support news gathering. Most controversially,
Miami Herald building, with Herald logo on right and Apple iPad billboard on left, April 2011 (Jeff Binion photo)
%Columbia Journalism School | Tow Center for Digital Journalism
%Introduction
Downie and Schudson recommended the creation of a national ``Fund for Local
News'' supported with fees the Federal Communications Commission would
collect from telecom users, broadcast licensees or Internet service providers.
That suggestion drew praise from some, as well as criticism from those who
saw it as an intrusion by the government in a free and robust press. In the words
of Seth Lipsky,^{\href{#endnotes-introduction}{3}}editor of the New York Sun, ``The best strategy to strengthen the
press would be to maximize protection of the right to private property—and the
right to competition. Subsidies are the enemy of competition.''
This report stands on the shoulders of the first one, but takes a different approach.
Without addressing the merits of philanthropists or governments supporting
news gathering,^{\href{#endnotes-introduction}{4}}we wanted to address another question: What kinds of
digitally based journalism in the U.S. is the commercial market likely to support,
and how?
While this report will examine some traditional, or ``legacy,'' business models
for media, our focus is on the economic issues that news organizations—large
and small, old and new—face with their digital ventures.
This report focuses on news organizations that do original journalism, defined
for our purposes as independent fact-finding undertaken for the benefit of communities
of citizens. Those communities can be defined in the traditional way,
by geography, but can also be brought together by topics or commonalities of
interest. We also look into media companies that aggregate content and generate
traffic in the process.
We confine our report mostly to for-profit news enterprises. We recognize
the outstanding work done by such national organizations as ProPublica and the
Center for Investigative Reporting, as well as local sites like Voice of San Diego
and MinnPost. But for the purposes of this study, we felt it was more valuable
to spend our time examining organizations that rely as much as possible on the
commercial market.
We do have a bias: We think the world needs journalism and journalists. We
welcome the tremendous access people now have to data and information, but
much of what Americans need to know will go unreported and unexposed
%The Story So Far: What We Know About the Business of Digital Journalism

without skilled, independent journalists doing their work. That work can include
reporting and editing in the traditional way, as well as aggregating information
from other sources, or sorting and presenting data to make it accessible
and understandable.
We decided to restrict our studies mostly to the U.S. market. We found the
domestic news scene to be a rich and textured one, with plenty of complexity
of its own, though we appreciate that a great deal of innovation is taking place
beyond U.S. borders.
We define digital journalism broadly. While many publishers still see it as an
online phenomenon—that is, displaying content on a PC screen via the Internet—
we have included other platforms, including mobile phones and tablets.
We found several challenges in preparing this study. First, while a great deal of
data about digital ventures is available, much of it is unverifiable. Small startups
and other private companies have no legal reporting requirements, so some of
the figures we cite here are taken with appreciation and on good faith. Further,
digital revenue is still such a small sliver of the total for publicly traded companies
that, when it is broken out at all, it is rarely displayed in such a way that reveals
how much comes from a particular station or publication. And, it often isn't clear
how much of a company's stated digital revenue represents genuinely new income
as opposed to legacy dollars reapportioned to online businesses.
We sought to make this report accessible to newcomers and useful to those
who have spent years in this field. We have tried to explain such terms as ``CPM''
(cost per thousand of views) and ``impressions'' (advertising spaces that appear
on a digital page) in the text. And we have tried to be as rigorous as possible in
examining numbers that media companies provide when describing their digital
results. We also consulted a number of secondary sources to provide background
and data unavailable elsewhere. These included important texts from the dawn
of the digital age, such as Stewart Brand's ``The Media Lab,''^{\href{#endnotes-introduction}{5}}and more recent
books, such as James Hamilton's ``All The News that's Fit to Sell''^{\href{#endnotes-introduction}{6}}and ``Information
Rules''^{\href{#endnotes-introduction}{7}}by Carl Shapiro and Hal R. Varian. Of course, we also relied on
more current sources, particularly such sites as paidcontent.org, niemanlab.org
and cjr.org.
%Columbia Journalism School | Tow Center for Digital Journalism
%Introduction 
But the bulk of the research in this report is based on a series of interviews we
did in late 2010 and early 2011. We visited mainstream print and broadcast organizations
with rich histories and Pulitzers and duPonts lining the walls; we also
interviewed the founders and editors of innovative new journalistic enterprises.
In most cases, publishers and editors were open, candid and willing to be quoted
on the record. In a few instances, we decided to trade confidentiality for access to
internal numbers or insights that would not otherwise be available.
We recognize, finally, that digital journalism is such a dynamic field that some
of the findings and conclusions we reach in May 2011 will be outdated within
months. That is what makes this subject so fertile for researchers and so humbling
for seers. And we conclude our study not with predictions but with recommendations
for how news businesses large and small, new and old, can more effectively
meet the challenges brought on by the digital transformation.

%Columbia Journalism School | Tow Center for Digital Journalism
%News from Everywhere: The Economics of Digital Journalism 7
\chapter{News from Everywhere: The Economics of Digital Journalism}
In early 2005, a researcher at the Poynter Institute published a column that was
instantaneously read and—by many—misunderstood.^{\href{#endnotes-chapter-1}{1}}Rick Edmonds, who studies the financial side of the news business for Poynter's
website, speculated about how long it would take for online newspaper revenue
to match the dollars brought in by the print side. He estimated that digital
ads accounted for around 3 percent of the total revenue for an average U.S. paper.
Edmonds assumed an optimistic online growth rate, around 33 percent a year,
and what seemed then to be a reasonably sober estimate of print growth, around
4 percent.

%The Story So Far: What We Know About the Business of Digital Journalism

Given how low online sales were at the time, Edmonds noted it would take 14
years for digital revenue to catch up to that of print. As he wrote, these calculations
provided ``little cause for cheer.'' He also noted ``there isn't any reason to
believe any of these numbers will remain steady state over time.''
His disclaimers were lost on many readers. At several conferences later that
year, participants pointed to the study and cheered one of the presumptions in
the column—that digital revenue would grow by a third every year, as far as the
eye could see.
For a few years, it seemed as if this scenario might be realistic. Newspapers'
online revenue grew by more than 30 percent in both 2005 and 2006.^{\href{#endnotes-chapter-1}{2}}But
growth slowed the next year, came to a halt during the recession and still hasn't
fully returned to what it was in 2007. Meanwhile, print revenue hasn't grown at
4 percent a year since 2005; indeed, newspapers' print revenue in 2010 was less
than half what it was in 2005.
Fifteen years after most news organizations went online, it is clear that old media
business models have been irrevocably disrupted and that the new models are
fundamentally different from what they once were. What made traditional media
so vulnerable to the Web? Or perhaps the better question is this: Why has digital
technology, which has been such a powerful force for transmitting news, not yet
provided the same energy for companies to maintain and increase profits?
Mainstream news organizations had already started losing audience before
the Internet became popular. Broadcast network news programs have been
sliding steadily since 1980 and now reach slightly over 20 million viewers a
night, down more than half in three decades. Newspapers began to experience
significant circulation declines decades ago. Total daily newspaper circulation
has fallen by 30 percent in 20 years, from 62.3 million in 1990 to 43.4 million
in 2010, as people found other sources, particularly local television news, to be
an adequate substitute.^{\href{#endnotes-chapter-1}{3}}Revenue, however, held steady or increased for mainstream news outlets, even
as audiences shrank. This was true in the early days of the Web, too, thanks in part
to an advertising bubble spawned by the Internet boom.
%Columbia Journalism School | Tow Center for Digital Journalism
%News from Everywhere: The Economics of Digital Journalism 9
To begin to understand the disruptions of the digital transformation, it is important
to appreciate the circumstances that made the news business—whether
in broadcast, cable, magazines or newspapers—so profitable for so long. The
commercial heyday that buoyed the fortunes of American newsrooms in the last
half-century had its roots in changes that began much earlier.
Through the 19th Century, newspapers benefited from economic and demographic
shifts that accompanied industrialization—in particular, rapid urbanization
and the attendant rise of the big-city retail economy. The growing advertising
market encouraged urban publishers, who had begun to loosen their ties to
political parties and to think of themselves as independent businesspeople. In the
process, they realized they could make most of their money from local retailers,
rather than from people in the street paying a few pennies to buy their papers.
Historians of journalism argue that these economic and political shifts underpinned
an increasingly professionalized and objective journalism that became the
norm in the 1920s and 1930s. The move toward general-interest, advertisingsupported
newspapers aimed at broad audiences also drove a cycle of concentration
and consolidation that would continue for decades.
With audiences and ad revenue growing even as competitors disappeared,
newsrooms and newspapers swelled in size. An analysis of major metropolitan
dailies by the American Journalism Review found that between 1965 and 1999,
eight of the 10 newspapers studied saw at least one competitor disappear.^{\href{#endnotes-chapter-1}{4}}During
the same period, on average, each of the surviving newspapers doubled the
amount of news it produced. Even as new or expanded sections—sports, business,
lifestyle—claimed a larger share of each edition, the total coverage of local,
national and international news continued to increase.
The trend of increasing consolidation in a growing advertising market helped
to compensate for declining readership. By the early 1980s, most U.S. cities had
just one daily newspaper. Or, in markets with two papers, one was clearly dominant
and the other was kept afloat by favorable terms negotiated in joint operating
agreements that Congress had created to preserve local journalistic competition.
Radio and television newsrooms enjoyed similar access to a lucrative
%The Story So Far: What We Know About the Business of Digital Journalism
%10
market. The advertising business in broadcast was so strong that even television
and radio stations with small market shares were profitable; those with a strong
command of the audience were cash machines.
The monopoly or oligopoly that most metropolitan news organizations
enjoyed by the last quarter of the 20th Century meant they could charge
high rates to advertisers, even if their audiences had shrunk. If a local business
needed to reach a community to promote a sale or announce a new store, the
newspaper and TV station were usually the best way to do it. Even if the station
or newspaper could deliver only 30 percent of the local market, down from
50 percent a decade earlier, that was still a greater share than any other single
medium could provide.
That changed after 2001. The recession that followed the September 11 attacks
forced many companies to cut spending, reducing media companies' advertising
stream. More importantly, the digital transformation accelerated, and more users
began to get their news, for free, on personal computers. The link between
a consumer's getting the news and a provider's expensive investment in publishing,
broadcast and delivery was broken; this brought a flood of new competitors.
Craigslist helped devastate classified ads, newspapers' most lucrative source of revenue,
and in 2008, the deep recession fueled by the financial crisis undermined
real estate and employment advertising.
As we get further into the digital age, we can more plainly see how the transformation
has affected news organizations and the citizens who depend on them.
Consumers certainly have benefited—they have more choices, speedier delivery
of news and more platforms. But as legacy companies shrink, these advantages
have often been accompanied by a loss of original news coverage. New entrants
have achieved impressive editorial results, but not many of them have achieved
financial stability without some philanthropic or other non-market support.
The move to digital delivery has transformed not just the business of news, but
also the way news is reported, aggregated, distributed and shared. Each of those
changes has an underlying economic rationale, and the media industry has sometimes
been slow to recognize the changes or has been paralyzed by their impact.
%Columbia Journalism School | Tow Center for Digital Journalism
%News from Everywhere: The Economics of Digital Journalism 11
Below, we list some of the most consequential changes brought on by the digital
era and offer thoughts on how they will affect the way journalism is supported
in the years to come.

\section{I. A Different Business}
\begin{itemize}
\item Digital requires a new way of thinking about your audience, one that
now feasts on an abundance of information. In the words of Syracuse
University Professor Vin Crosbie, ``Within the span of a single human generation,
people's access to information has shifted from relative scarcity to
surplus.''^{\href{#endnotes-chapter-1}{5}}As Crosbie notes, it isn't enough simply to transfer content from
a legacy platform to a new one. Digital journalism requires an entirely different
mind-set, one that recognizes the plethora of new options available
to consumers. Tom Woerner, a senior vice president at freelance-generated
site Examiner.com, notes that ``the old distribution model allowed for only
so much content. There are only so many pages you can print, only so many
minutes you can sell in a broadcast. … Now the limits are gone, for both
good and bad.''
Impact: Readers have access to far more information than they used to,
almost always for free. But for publishers, the competition is nearly infinite,
meaning much of the news has become a commodity, with pricing to match.
\item Digital is where the users are heading. In the most recent study by the
Pew Research Center for the People & the Press, 65 percent of people ages
18 to 29 get their news from the Internet—outpacing television for the first
time and far exceeding the 21 percent in that age group who rely primarily
on newspapers.^{\href{#endnotes-chapter-1}{6}}Among people ages 50 to 64, the Internet (34 percent)
and newspapers (38 percent) are almost tied. The Web's growing popularity
means the ``network effect'' can kick in. That is, as more people use news
sites, those sites become more valuable to their users, especially as readers and
viewers comment on—and contribute to—stories. Meanwhile, more usage
is gravitating from computer screens to smartphones, tablets and other mobile
devices. According to a January 2011 Pew study, 47 percent of American
adults say they get at least some local news and information on their cellphone
or tablet computer.^{\href{#endnotes-chapter-1}{7}}%The Story So Far: What We Know About the Business of Digital Journalism
%12
Impact: Digital platforms provide ways for audiences to build quickly with
lower marketing costs than in traditional media. And the shift to mobile provides
news organizations with more opportunities for targeted content and
advertising. But increased audiences don't always lead to proportional gains;
in other words, more people may be viewing a site, but that doesn't mean
revenue increases to the same or greater degree. Witness a recent report by
McClatchy Co., the third-largest newspaper firm in the U.S. The company
said the number of local daily unique visitors to its websites grew by 17.3
percent in 2010, yet digital revenue rose only 2.4 percent for the year.^{\href{#endnotes-chapter-1}{8}}And mobile ad sales have so far been less lucrative than those on Internet platforms.
Chris Hendricks, vice president of interactive media for McClatchy,
says that ``seven percent of our traffic comes from mobile. The traffic is significant,
the revenue is not.''
\item Digital provides a means to innovate rapidly, determine audience size
quickly and wind down unsuccessful businesses with minimal expense.
The substantial capital expenditures that used to be involved in
starting a new media company are largely gone. A video service need not
build tall antennae atop the highest hills in town, and print publishers can
avoid capital-intensive investments in printing presses. The large staffs associated
with getting information to readers—whether they're camera crews
or printing staffs—aren't as necessary. It took Sports Illustrated at least 10
years to get its formula right and become profitable9; it took Huffington
Post less than six years to go from an idea to a valuation of \$315 million in
its 2011 sale to AOL.
Impact: The development time from idea to market is shortened, greatly
increasing efficient use of a firm's resources. But because competitors can
imitate or adapt more quickly, it is difficult to cash in on innovations. The
shorter cycles can lessen the length of time that innovations remain unique,
relevant and valuable.
\item Digital platforms extend the lifespan of journalism. In the analog era, news
stories were as ephemeral as fruit flies. An article was prominent for a day, then
available only on a library's microfiche; a video would be broadcast to millions
%Columbia Journalism School | Tow Center for Digital Journalism
%News from Everywhere: The Economics of Digital Journalism 13
on the nightly news, then it would be sent to a network's vault. Journalism
now can be freely accessible for as long as a publisher wills it to be. In the words
of one programmer, ``There is no such thing as 'yesterday's news.' ''
Impact: News organizations can make money from their archives as part
of a subscription or pay-per-view service, or as part of a scheme to provide
more content and build traffic and ad revenue. But as increasing amounts of
content stream into archives, consumers may have greater difficulty finding
what they want.
\end{itemize}
\section{II. Content and Distribution: A Fundamental Change}
\begin{itemize}
\item Digital disrupts the aggregation model that was so profitable for so
long. Almost no one used to read the entire newspaper every morning,
and audiences frequently tuned in and out of the network news at night.
Yet, news organizations sold their advertising as if every page was turned
and every moment was viewed. Indeed, print publications applied a multiplier—
often up to 2.5 readers—to account for the audience for each
edition they sold. But in the online world, content has become atomized,
with each article existing independently of the next. It is as seamless for a
reader to go from a tallahasseedemocrat.com story to a video on msnbc.
com as it is to read back-to-back stories in Esquire magazine. The economic
consequences of this fickle information-gathering are devastating
for legacy news organizations, especially because they have ceded many
of the benefits of aggregation to sources like Drudge Report, Huffington
Post and Google News. Says Michael Golden, vice chairman and president
of the New York Times Co.: ``We've lost the power of the package.''^{\href{#endnotes-chapter-1}{10}}Impact: News relevant to a particular audience can be assembled cheaply
and easily, with significant benefit for readers seeking divergent and even
competing points of view. But low-cost aggregators compete with content
creators for page views, and often win. In the words of Aaron Kushner, an
investor trying to buy the Boston Globe, ``The definition of a competitor
now is someone who gives away your story for free.''
\item Journalists today can find readers wherever there is access to the Internet.
This is an enormous transformation after a century in which the
reach of print journalism was limited by a company's printing plants and
%The Story So Far: What We Know About the Business of Digital Journalism
%14
trucks, and most broadcast news was tied to narrow geographic areas. Even
when local newspapers expanded their circulation far beyond their metropolitan
areas, the results were usually disappointing—the more geographically
distant the reader, the less loyalty and interest in the content. (Three
national newspapers—USA Today, the New York Times and the Wall Street
Journal—avoided most of those constraints by delivering national rather
than local news in authoritative, attractive packages.) By contrast, publishing
online means that any article or video will become immediately available
around the world, at no added cost. Meanwhile, broadcast outlets' reach,
once defined largely by geographic and bandwidth constraints and enforced
by regulatory agencies, is expanding. Their content is no longer limited to
local markets and thus is less restricted by federal regulations.
Impact: Journalists and media companies can go where the audience is, expanding
markets at low costs. But the advantages that went along with distribution
limits—such as protection against new competitors—are disappearing.
\item Digital platforms enable publishers to deploy their readers and viewers
in publicizing and distributing their content. Print publishers used
to tout the ``pass-along audience''—people who didn't buy a magazine or
newspaper but picked it up in, say, a dentist's office, and could therefore be
counted as readers. Advertisers were often skeptical of the numbers, which
depended on surveys of readers trying to remember if they read a publication
they didn't pay for. But digital news organizations can track precisely
how people share content—a few years ago mainly by email, and now also
by social media like Facebook and Twitter. For journalists, such distribution
helps validate and publicize their work.
Impact: Publishers get free distribution with excellent, real-time information.
At the same time, they are losing control of the distribution platform that
generated such healthy profits. And they have less say over how their content
is portrayed; sometimes users post links and add a dollop of nasty criticism.
\end{itemize}
\section{III. What's Happening To Consumers?}
\begin{itemize}
\item News organizations can more easily build new audiences centered on
specialized topics or interests. Because everything online is instantaneously
and ubiquitously available, it's far easier to create offerings of more focused
content and find users no matter where they live. Fans of a city's football
team may be spread around the world, but a news organization can build a
site that will draw a substantial audience.
Impact: Highly focused audiences can provide more value to advertisers. But
separating audiences into too many niches can bring on a new set of problems.
Consumers may find that dealing with multiple content providers—
with few guideposts to judge the quality or authority of the source—isn't
worth the bother.
\item Publishers have more information about their readers, in real time.
Whether a citizen is using free Google Analytics on a blog, or a mainstream
organization is deploying more sophisticated usage-tracking services like
Omniture or Chartbeat, journalists know much more about who's viewing
their content, where the audience is coming from and how it is engaged.
Unfortunately, many of these numbers are unreliable, misconstrued
and prone to exaggeration. Usage estimates often vary by 200 percent or
more. This issue was explored in detail in a report last year by Columbia
University's Tow Center for Digital Journalism.^{\href{#endnotes-chapter-1}{11}}Metrics have always been
challenging for advertisers, especially in the broadcast world. But as the Tow
report notes, digital media have failed to come up with common standards;
they have not yet settled on metrics, whatever their flaws, as broadcast media
did generations ago. ``It is a long-appreciated irony of media measurement
that accuracy matters less than consensus,'' the report said. ``Doubts don't
matter much as long as no competitor is seen to benefit.''
Impact: Media companies can measure the popularity of articles, videos or
sections and adjust their strategy to maximize revenue and audience. But uncertainty
around metrics inhibits advertisers from investing fully in the digital
marketplace and depresses advertising rates for those who do take part.
%The Story So Far: What We Know About the Business of Digital Journalism

\item Digital platforms fundamentally change the customer experience, in
ways that are both advantageous and harmful for news organizations'
economics. Publishers can now capture highly valuable bits of user information,
ranging from areas of interest to credit-card numbers. But new media
rarely provide the immersive experience found in traditional platforms.
Many users keep numerous sites open on tabs in their Internet browsers and
don't focus on any one for very long; they often come to a news site through
a search and quickly leave for another. Links to other sites provide value to
readers but also send them elsewhere, sometimes never to return.
Impact: By tailoring content and advertising, publishers can charge higher
rates to advertisers and win greater loyalty from users. But privacy concerns
may lead to regulations that will limit the information publishers can glean
about their users. And most readers spend far less time on digital sites than
they did on legacy platforms, so news organizations have less opportunity to
attract advertising dollars. In the words of Steve Harbula, an editor at Examiner.
com: ``Readers have a large appetite but a short attention span.''
\end{itemize}
\section{IV. Cutting Costs And Seeking Revenue}
\begin{itemize}
\item Digital upsets media's typical pattern of high fixed costs and low variable
costs. It costs a lot—and often requires companies to take on a great
deal of debt—to produce the first copy of a newspaper or magazine. But the
second copy, and the thousands or millions that follow, are relatively cheap.
In the digital realm, many of those initial costs are eliminated, and in some
instances—such as starting a blog—they decline to zero.
Impact: This is a particular challenge for companies that have sunk mounds
of cash or taken on debt to make acquisitions that have high fixed costs;
those publishers now find such investments to be drags on profitability. Their
digital competitors aren't saddled with the same disadvantages.
\item Digital enables news organizations to trim the cost of doing journalism,
particularly if they can get citizens to provide content by bringing them into
news production or encouraging them to participate on comment boards.
%Columbia Journalism School | Tow Center for Digital Journalism
%News from Everywhere: The Economics of Digital Journalism 17
Impact: Since news and content supplied by paid professionals begets free
content by readers/users, the average cost to produce a page view is driven
lower. But the quality, accuracy and authority of this content are highly variable
and susceptible to manipulation.
\item On digital platforms, it is often hard to make sure that advertising supply
matches demand. Online editors frequently have a difficult time generating
enough page views when advertisers demand them—or filling up that
advertising space when reader traffic soars and ad demand is light. So news
sites often need to run cheap ads, called ``remnants,'' that may get a tenth of
the revenue their usual ads draw. Michael Barrett, the CEO of Admeld, a
company that tries to increase advertising rates on sites with traffic prone to
peaks and valleys, says that some of his clients view the situation ``like seats
on an airplane. They don't want to fly the plane with any empty seats.''
Impact: Because the cost of creating each additional page is close to zero,
media companies can have a wide range of prices, charging the highest rates
for the most desirable times, placement and audience. But all those unpredictable
page views exert constant downward pressure on ad prices.
\item Advertising is transformed in a digital format, and not always for the
better. Some journalists may not realize this, but many of their readers and
viewers see advertising as useful and entertaining. Indeed, access to advertising
is another incentive for people to buy magazines and newspapers or
listen to and watch broadcasts. But the appeal of online advertising is often
diminished by its format. A small, rectangular banner ad conveys little useful
information—certainly less than an insert in a newspaper or a glossy ad
in a fashion magazine. To get useful information from an online ad, a reader
often must click and head to a new site, something people rarely do. And the
more intrusive forms of online advertising—such as ``roadblock'' messages
that take over the entire screen for a few seconds—upset the user experience.
Some digital companies are bringing content value to ads, but they tend
not to be news media. Google became a powerhouse by tying advertising
directly to users' search queries. And Groupon, which attracts readers who
%The Story So Far: What We Know About the Business of Digital Journalism

are looking for online discount coupons, has become successful with witty
come-ons and obvious value. Groupon has expanded rapidly into hundreds
of markets and has turned down a \$6 billion offer from Google.^{\href{#endnotes-chapter-1}{12
}}Impact: Digital provides the ability to target advertisers' messages and better
metrics to determine impact. But users find that many digital ads on news
sites convey little information and value.
\item Digital platforms provide another way for advertising departments to
attract new clients and retain old ones. For salespeople who don't feel
they have enough arrows in their quiver, online and mobile can be a way to
get a reluctant advertiser into the fold.
Impact: Media companies can bolster more profitable legacy sales in traditional
media by adding digital, and in the process, can move their clients to
newer platforms. But deals that combine legacy and digital ad sales make it
difficult to determine how much revenue is truly attributable to new media.
At some companies, half of digital sales have been ``bundled'' with print or
broadcast, and the way those dollars are apportioned can be largely at the
whim of the accountants, rather than being an accurate reflection of the
value of the ads.
\item Many efforts to get readers to pay for content have been fitful, poorly
executed and motivated more by ideology than economics. Only a few
publications have had a successful, long-term plan to get readers to pay, and
even fewer have done it in a way that genuinely increases online revenue
rather than simply protects their traditional businesses. Was free content journalism's
``original sin''?^{\href{#endnotes-chapter-1}{13}}Perhaps, for news organizations must now ask readers
to start paying for material that has been free for 15 years. Meanwhile,
pay-per-article schemes, such as the one proposed in a 2009 Time cover
story by Walter Isaacson, haven't caught on for journalism.^{\href{#endnotes-chapter-1}{14}}Unlike Beatles
songs, news stories have little lasting value beyond a single use.
%Columbia Journalism School | Tow Center for Digital Journalism
%News from Everywhere: The Economics of Digital Journalism 19
Impact: Users have unlimited access to most content, and publishers have
unlimited access to most users. But circulation revenue, one of the mainstays
of the traditional media business, has withered. And one of the methods that
advertisers have used to judge audience quality—willingness to pay—has
evaporated as well.
\end{itemize}
* * *
As one looks at this list, it becomes clear that most of the economic disadvantages
have been fully realized at news organizations, while many of the benefits—
such as a surge in mobile-phone advertising—are more potential than real. At the
same time, some new models are emerging that can replace some, if not all, of the
revenue news organizations have relied upon. Journalists and publishers, new and
old, are responding to this new environment in a variety of ways. We'll examine
how they have coped, transformed and endeavored to meet the challenges of the
digital era.

\chapter{The Trouble with Traffic: Why Big Audiences Aren't Always Profitable}
At first glance, the numbers don't seem to add up: The New York Times
has more than 30 million online readers and weekday circulation of less than
900,000 newspapers. Yet, the print edition still accounts for more than 80 percent
of the Times' revenue.^{\href{#endnotes-chapter-2}{1}}A broader recent study revealed the same phenomenon: It
showed that the Internet occupied 28 percent of Americans' time spent in media
in 2009 but generated only 13 percent of total advertising spending.^{\href{#endnotes-chapter-2}{2}}To understand why, it's important to realize that the prices advertisers pay
digital news organizations depend on many factors. Some are tied to the overall
market, especially the vast and growing amount of ad space (or inventory) that's
available online. Other factors have to do with a site's own dynamics, including
the size of the audience it reaches and that nature of that audience—its demographics,
how much time its users spend with the site and so on.
So, the Web offers a lot of advantages to publishers and advertisers. But its audiences
are more wide than deep.
Journalists constantly feel the push and pull of these numbers. ``What am I
today?'' asks Jeff Cohen, editor of the Houston Chronicle. ``I'm an aggregator
of eyeballs. … We're doing around 79 million page views a month—almost a
billion in a year.'' Yet, for all that Web traffic, his newsroom's 206 employees are
about half the number employed in 2006.
Digital numbers are confusing when compared with traditional media metrics
and are often inflated for all sorts of reasons. Users can be counted several times
if they deploy multiple devices, such as a PC, laptop and mobile phone, to access
a site. Also, many people delete their computers' ``cookies,'' small text files that
allow them to be identified and tracked; because of that, they appear to be new
visitors to sites rather than returning ones.^{\href{#endnotes-chapter-2}{3}}%The Story So Far: What We Know About the Business of Digital Journalism

For all that, digital audiences usually far outnumber those from a traditional
outlet. As a result, the industry faces a perplexing set of questions: Why do so
many digital users generate so little advertising revenue? Is it simply that digital
systems are more efficient than the previous oligopolies of the print and broadcast
world? Or is there something more fundamentally askew about the way
media companies make money off digital customers? And, more importantly,
what should publishers be doing to make the most of the readers and viewers
they have?
* * *
At its most basic level, advertising is a numbers game. A news organization
needs a certain number of readers or viewers, and the more it gets, the more ads
it can sell and the more it can charge those advertisers. Users also spend varying
amounts of time with the magazine, newspaper or broadcast, and the more time
they spend, the more an advertiser values the audience.
Digital platforms, thanks to their ubiquity and ease of use, are terrific at the
first part of the numbers game. News sites have demonstrated the ability to attract
huge numbers of users. And in the first decade of the digital era, particularly as
search engines became more powerful, publishers and broadcasters focused on
building a mass audience. They poured resources into search engine optimization—
the term used to describe a way to improve the odds that headlines will be
picked up by Google or other search sites and that topics will be timely enough
to appear prominently on a results page. News sites initially welcomed aggregators,
such as Drudge Report and Huffington Post, that linked to their material
and increased traffic.
As a consequence, audience sizes swelled, and publishers have proclaimed that
to be a success. So when the Los Angeles Times in March 2011 logged a record
195 million page views, clicked on by 33 million users, the site's managing editor
took a moment to proudly announce those statistics on the site.^{\href{#endnotes-chapter-2}{4}}%Columbia Journalism School | Tow Center for Digital Journalism
%The Trouble with Traffic: Why Big Audiences Aren't Always Profitable 23
But within such numbers is another, less happy, story. The arithmetic shows
that each latimes.com user clicked on an average of six pages in March—or just
one page every fifth day of the month. That statistic demonstrates how the other
essential part of the advertising business—the amount of time and attention that
users pay to a site—has been undermined by some of the tactics that publishers
have used to attract large audiences.
The phenomenon is widespread. A 2010 Pew analysis of Nielsen media statistics
depicts depressingly low levels of usage, even at outstanding national sites.
``The average visitor spends only 3 minutes, 4 seconds per session on the typical
news sites,'' the study says.^{\href{#endnotes-chapter-2}{5}}``No one keeps visitors very long.'' And at toptrafficked
news sites, ranging from Yahoo News to the Washington Post to Fox
News, most people visit just a few times per month. Compare that to the media
of past decades. A 2005 study showed that about half of U.S. newspaper readers
spent more than 30 minutes reading their daily paper. And most of the lessdevoted
readers spent at least 15 minutes with the paper.^{\href{#endnotes-chapter-2}{6}}There are many reasons for the problems news sites have in getting reader attention,
also known as engagement. Consumers today have many digital options
available. The experience of getting news on a computer or mobile device, thus
far, is fundamentally different from the experience in TV or print; most users
tend to flit from site to site, rarely alighting for more than a brief spell.
But there's another way to look at these numbers, one that is more complex—
and, in some ways, more encouraging—for the journalism business.
One person studying this issue closely is Matt Shanahan, an analyst who is
relatively new to the media world. Shanahan is, by training, an electrical engineer,
and he spent much of his career consulting for big businesses in the fields of
aerospace and finance. In 2008, he joined a Seattle-area firm called Scout Analytics;
the company was looking to serve industries that were failing to realize their
potential in the digital world. ``The e-commerce market was noisy and crowded,''
he says. ``Media and information, however, were mature industries facing severe
revenue issues.'' Scout figured it could find new clients among media companies
that were trying to learn more about the real nature of their digital audiences.
The company signed up about 70 news and information sites—some geared to
consumers, but more of them in the business-to-business category—and started
%The Story So Far: What We Know About the Business of Digital Journalism

looking at the activity of users. They were able to track consumers' paths through
sites and to figure out how often they visited and what they did on each visit.
Shanahan was soon struck by an anomaly of the online news industry: the yawning
maw that separates the size of audiences from the level of engagement those
readers and viewers demonstrate.
Shanahan points to a website for a 90,000-circulation newspaper that serves a
medium-sized city on the East Coast. (The name of the company is confidential
because it's a client.) This site gets around 450,000 unique visitors a month.^{\href{#endnotes-chapter-2}{7}}But those visitors differ widely, and Shanahan separates them into four types:
The most loyal are the ``fans,'' who visit at least twice a week. Then there are the
``regulars,'' good for one or two visits a week. Sliding down the loyalty scale are
``occasionals,'' who stop by two or three times a month; and finally, the ``fly-bys,''
who come about once a month.
The most loyal visitors are a very small part of the overall audience: Fans make
up about 4 percent of the total number of visitors, and regulars 3 percent. Occasionals
account for 17 percent and fly-bys for more than 75 percent of the total.
In other words, more than three-fourths of the people who visit this news site
do so about once a month.
Then Shanahan went deeper, to see how the different kinds of users behaved
on the site. He knew the most loyal fans would generate more page views than
the fly-bys, since fans visit the site more often. But the disparities in usage were
far greater than one might expect.

Fans, despite their small numbers, were responsible for more than 55 percent of
the site's traffic. Fly-bys—those people most likely to come from a search engine
or a blog—clicked on barely three pages a month. Overall, each fan generated
about 50 times more traffic per person than a fly-by.
``When people talk about the size of an audience, that's a sham,'' Shanahan says.
In his view, stated numbers don't reflect how differently the varieties of users act
in the way they navigate a site. Publishers mistakenly focus on ``page views rather
than length of time,'' he writes on his blog, Digital Equilibrium.^{\href{#endnotes-chapter-2}{8}}Referring to ad
``impressions,'' which are appearances (not clicks) of ads, Shanahan adds, ``Using
today's standard, there is no difference between impressions that last 1 second, 10
seconds, or 2 minutes.''
``The digital world has changed the revenue dynamics for publishers,'' he adds
in another post. ``In the print world, a publisher's shipment of physical media was
the basis for generating revenue. In the digital world, consumption of media is
the basis for revenue. ... In other words, engagement is the unit of monetization.''^{\href{#endnotes-chapter-2}{9}}Shanahan says the benefit of more engagement isn't just in higher ad rates,
but in relationships that publishers need to build with their most loyal readers—
something that has been lost in the drive to attract mass audiences. By chasing
after large audiences rather than deeply engaged ones, he says, news organizations
are sacrificing advertising revenue. Publishers who have a ``direct relationship
with fans can push better contextual advertising''—that is, ads that relate directly
to a user's habits and interests. ``A publisher can know which fans saw which
advertisements in which context on their site.'' Sites can use that information to
provide readers with targeted ads or offers.
And, news organizations that hope to charge for online access need to locate—
and cater to—fans, since engaged users are more likely to subscribe. That
way of thinking is at the heart of the New York Times' decision to charge for
access to its digital editions. The Times built its pay scheme so light users of the
site won't have to pay for access, while heavier users—defined as those clicking
on at least 20 stories a month—will be charged. But the Times also sees a
connection between engagement and advertising. Michael Golden, vice chairman
and president of the New York Times Co., discussed the different kinds
of online audiences during a speech at a March 2011 advertising conference,
%The Story So Far: What We Know About the Business of Digital Journalism

shortly before the pay strategy went into effect. Speaking of less committed
users, Golden said, ``Their engagement is limited, but their numbers are impressive.
… We have to balance engagement and reach. The higher the engagement,
the higher the CPM.''^{\href{#endnotes-chapter-2}{10}}* * *
Two media outlets with very different editorial missions—Gawker Media and
PBS—have tackled this issue of trying to differentiate more and less engaged segments
of their audience when pricing advertising.
Gawker's network of sites, started by British journalist Nick Denton in 2003,
includes Gizmodo (for gadget lovers), Deadspin (for sports fans) and Jalopnik
(for car buffs). In March 2010, Gawker began touting a metric it calls ``branded
traffic.'' This was defined as people who have bookmarked the company's sites
or arrived at them by searching specifically for the site by name—by, say, typing
``Deadspin'' into a search engine.^{\href{#endnotes-chapter-2}{11}}Gawker found that roughly 40 percent of visits come via branded routes, as
contrasted to links from search engines. And such visitors are more devoted
and engaged, spending 91 seconds more per visit than others. That is a meaningful
difference with financial impact, says Erin Pettigrew, Gawker Media's
marketing director.
First, the more engaged users are more likely to see highly profitable ``roadblock''
advertisements—ads that take over the home page of the site for several
seconds and then fade away to reveal the editorial content of the page. That revenue
helps compensate when advertising rates fall at Gawker and other sites. As
Reuters' Felix Salmon has noted, the number of online ads has been growing so
fast that advertisers can demand lower rates. As a result, wrote Salmon, ``Denton
says that since 2008 he has been getting only half the revenue per page that he
used to get in 2004.''^{\href{#endnotes-chapter-2}{12}}One way to counteract that is to try to sell more ``branding'' advertising on the
Web—that is, ads designed to draw positive attention to a company's name and
image, rather than to trigger a direct response to an offer for a product or service.
%Columbia Journalism School | Tow Center for Digital Journalism
%The Trouble with Traffic: Why Big Audiences Aren't Always Profitable 27
Brand advertising has long been profitable for traditional media, especially television
and magazines. But because publishers have struggled to convince advertisers
that the Web is a good platform for branded ads, they've often missed out on
this lucrative revenue stream. ``We had been told ad frequency of more than two
or three exposures is a bad thing,'' said Pettigrew. ``But for brand advertising, it's
different. Six to 12 exposures [of an ad] increased the persuasiveness. And this is
the most useful segment to our advertisers.''
Because Gawker Media has an array of sites on different topics, it can sell engagement
across its network. Thus, the same readers who click on gadget stories
at Gizmodo can be served with branded ads at Lifehacker or Gawker. And if
Gawker Media can demonstrate to advertisers that its readers are loyal, it can
charge higher ad rates, Pettigrew said.
At the other end of cultural spectrum, the Public Broadcasting System's Web
strategists are also using engagement metrics to increase revenue.
Amy Sample, director of web analytics for PBS Interactive, says she and others
at the site modified a formula created by Web analysts Eric Peterson and Joseph
Carrabis to get a better sense of which readers were most devoted.^{\href{#endnotes-chapter-2}{13}}They came
up with their own criteria to determine PBS.org's most loyal audience, based on
the number of pages a reader views, the amount of time a reader spends on the
site, and how often and how recently readers have come.
As it turned out, less than 5 percent of the visits on the site came from users
who met all of PBS's engagement standards. But those people are a critical
group, says Sample. She found that they stay on PBS.org for 13.5 minutes per
visit (compared with a three-minute average for everyone else) and click on nine
pages per visit (versus three for other users). PBS saw economic benefits from
this audience. Such users were 38 percent more likely to donate money to PBS
than less engaged users; they were also more prone to encourage others to use
the site. And when PBS saw the usage patterns, executives decided that video,
a favorite platform for frequent users, should be promoted more prominently.
That translated into revenue, because the site's video ads get healthy \$30 CPMs,
Sample said, or about three times as high as other ads on the site.
%The Story So Far: What We Know About the Business of Digital Journalism

* * *
Engagement correlates with editorial content. To see how the relationship plays
out at a large site, we can examine some numbers for dallasnews.com, the main
site for the Dallas Morning News. (These metrics are from the full year of 2010,
a few months before the publisher began charging for access to much of the site.)
For the year, the site averaged around 40 million page views a month, driven
by 5 million visitors who visit, on average, about twice a month and click on
about four pages per visit. Those numbers are fairly typical for a site the size of
Dallas' and provided the publisher, James Moroney, some of the figures he used
to calculate the rationale for instituting the pay-for-access plan (see Chapter 5).
But the broad numbers tell only part of the story. In fact, dallasnews.com, like
many big online organizations, is many sites rolled into one. To analyze its data,
the company sorts its traffic statistics into various categories, including by content
areas: news, entertainment, sports, weather and blogs.
News gets the most traffic, in terms of total visitors and visits. News visitors
average around two visits a month and click on an average of about 1.5 pages
per visit. Their habits are typical of those found at many other news sites—not
particularly engaged.
Sports does better in engagement. Users average about 2.3 visits a month, and
about 3.4 pages per visit over the course of the year. During the fall of 2010,
when the Texas Rangers were in the Major League Baseball playoffs and the
Dallas Cowboys were on the football field, users clicked on four or more pages
per visit.
And then there's a feature on the site called High School GameTime, which
includes rosters, schedules and results from the state where ``Friday Night Lights''
is based. Users clicked on nearly nine pages per visit in November 2010, during
the height of the football season, and generated almost as many page views as the
entire news section. Over the year, high school sports fans were about five times
as engaged as the people coming to read news.
%Columbia Journalism School | Tow Center for Digital Journalism
%The Trouble with Traffic: Why Big Audiences Aren't Always Profitable 29
It's easy to see why. The site offers a dizzying array of statistics, rosters and
standings for more than 200 high schools in the Dallas-Fort Worth area. Mark
Francescutti, senior managing online editor for sports, says the site's engagement
demonstrates the power of ``great local content … that is exclusive and is
important to people.'' And loyalty, not search engine optimization, is the key to
maintaining the audience. ``We might get lucky and get linked off Google, but
we want people who will come back every single day,'' he says.
The site has a small but intense crew. The News' four full-time high school
sports reporters file frequently, and editors also rely on clerks who take scores
and statistics over the phone from stringers around Dallas. On Friday nights,
scores are updated during games, not just reported when the games are over.
There's also a live chat where reporters
update games—``controlled chaos,'' in
the words of Kyle Whitfield, the site's
editor. High School GameTime aggregates
heavily from other sources. ``Our
writers are not robots,'' says Francescutti.
``We don't have that old journalistic
ego that says, 'If we didn't write it,
it's not important.' ''
The News used High School Game-
Time as part of a package deal with
Time Warner Cable that also included
print ads, a radio show, a player-of-theweek
contest and a banquet at the end
of the football season. High School GameTime has brought in up to \$700,000,
says Richard Alfano, a general manager. For the next season, he says, the News
will sell a \$1.99 mobile app for High School GameTime that will include playby-
play from at least 100 games a week.

The key, says Whitfield, is focusing on something that readers care about
deeply and that no other news provider does as well. ``It's more difficult to sell
Cowboys coverage, because Cowboys fans are everywhere around the country,''
Whitfield said. ``We were able to organize our resources and monetize it, which
is oh-so-rare online.''
* * *
Rapid audience growth is often accompanied by thin engagement. Such has
been the case at Examiner.com, a freelance-driven site that has built an audience
of more than 22 million unique users in three years.
Examiner is owned by Clarity Digital Group, which is controlled by Philip
Anschutz, a Denver entrepreneur who has made billions in energy, railroads,
entertainment and sports franchises. The site, which was started in April 2008,
has brought aboard more than 72,000 freelancers who have written on topics
ranging from roses in Rhode Island to parenting in Portland.^{\href{#endnotes-chapter-2}{14}}There's not a
great deal of supervision: Writers must pass a criminal background check, and
they get some quick training. Their first story goes through an editor, but after
that, the writers usually post directly to the site.
Page views are a key factor in determining writers' pay, which amounts to
between \$1 and \$7.50 per thousand views, according to AdAge, or a few dollars
per article.^{\href{#endnotes-chapter-2}{15}}According to Mike Noe, senior director of recruiting, fewer than a
third of the writers are currently ``active,'' which, in Examiner parlance, means
they've posted something to the site within the last 90 days.
The content that Examiner.com produces mimics much of what has traditionally
appeared in the back of newspapers or at the end of broadcasts—subjects
like sports, weather, hobbies or opinion. Writers are hired in large part
based on their zeal for a topic. ``In a traditional newspaper, the reporter might
not be passionate about the [Denver] Broncos,'' says Jen Nestel, Examiner's director
of community. ``We do the reverse. We take someone who is already
passionate and we teach them how to write.'' The site doesn't claim to replace
the newsgathering functions of traditional media: ``Finding out how the school
%Columbia Journalism School | Tow Center for Digital Journalism
%The Trouble with Traffic: Why Big Audiences Aren't Always Profitable 31
board works is hard,'' says Rick Blair, chief executive officer of Examiner.com.
``It takes a special kind of digger. I could see other folks using platforms like ours
to do that. But we don't have the tools or the accredited manpower.''
For all its success in building an audience, Examiner has quite low engagement:
Its readers see about 65 million page views a month, or only about three pages
per visitor. That is likely tied to the site's dependence on search engine optimization,
or SEO.
``The problem with SEO is, the visitors are snackers,'' says Blair. ``If people
come in through the front door [the home page], they read seven to eight pages.
If they come in the side door [such as a search engine], they read maybe two,'' he
says. Suzie Austin, senior vice president for content and marketing, adds, ``From
the very beginning, we did search engine optimization right. The benefit is obvious—
you get a lot of eyeballs. The downside is, there's not a lot of engagement.
Page views per user is growing, but at a low rate.'' And as sites use SEO to boost
traffic, advertisers take advantage of the flood of page views around the Web to
``name their price,'' says Tom Woerner, Examiner's senior vice president for national
sales. There are two ways for publishers to deal with that, he says: ``Play the
price game, or add value to what you give the advertiser.''
So Examiner is shifting from simply selling display ads to selling the value of its
ability to project stories beyond the confines of its own site. Examiner coaches its
writers on deploying social media to broaden the influence of their stories. In the
marketing business, using social networks is now considered a form of ``earned
media''—that is, it's more akin to publicity, like an appearance in a news article,
than to an advertisement or paid product placement. ``Thirty years ago, if you got
a story into Sports Illustrated about your product, that was 'earned' media because
you didn't pay for it,'' says Woerner. Today, earned media includes messages
that go out via Facebook, Twitter and blogs. ``Marketers have to be willing to
give up a little control.'' Woerner also said, ``The key for traditional media is how
they're engaging with their audience. They got used to the role of the gatekeeper.
They need to invite the audience in.''
To attract ads from Iams, Procter & Gamble's pet-food company, for example,
Examiner invited (but didn't require) its writers who focus on animals to write
about pet adoption and shelters; just as importantly, editors encouraged writers
%The Story So Far: What We Know About the Business of Digital Journalism

to distribute their stories via social networks. Iams didn't control the content, but
given how innocuous the Examiner's coverage of animals tends to be, the company
was unlikely to be troubled by photo galleries of adorable homeless puppies
and feature articles about courageous German shepherds.^{\href{#endnotes-chapter-2}{16}}About 840 writers
responded with more than 5,200 articles and additional posts on social-media
sites linking back to the stories. Those extra links from Facebook and Twitter to
Examiner stories helped drive up advertising rates. Site executives say that ads
sold in this effort get CPMs of more than \$11, as contrasted to their usual display
ads that get CPMs of \$3 to \$5.
Most other online news organizations are also establishing fan pages on Facebook,
setting up Twitter feeds and encouraging readers to share links. They are
doing this not just because the networks are where the audiences are, but because
they think social media will bring readers who are more engaged than those
who come through search engines. At Gawker, Google-driven traffic ``is waning,''
says Pettigrew, the marketing director. Facebook is now the top referrer, and
Twitter is gaining. But it wasn't easy for Gawker management to come to terms
with social media. ``We didn't want to join in the 'fan-page game,' '' she said, lest
readers become more accustomed to accessing its stories from Facebook than
from Gawker's home pages. ``You want to own the distribution.'' But eventually,
Facebook's power as a traffic-driver won out. ``You can't ignore the way people
want to access content.''
Vadim Lavrusik, former community manager at social media site Mashable,
says that ``readers who come through social are far different in their behaviors.
They tend to view more articles on average and stick around the site longer.''
Facebook and Twitter visitors spent 29 percent more time on Mashable.com, he
said, and viewed 20 percent more pages than visitors arriving via search engines.
Similarly, at The Atlantic's website, ``The percentage of referrals from social
nets is coming in at about 15 percent. And it's growing,'' says Scott Havens, vice
president of digital strategy and operations. There's a wide array of social sharing
tools on TheAtlantic.com, including Facebook, Twitter, Digg and Reddit. The
Atlantic has also started using Tumblr, a microblogging platform that allows anyone—
from individuals to media companies—to post text, photos and videos. It
has a distinctive visual format and is another way to drive engaged traffic. NewsColumbia
Journalism School | Tow Center for Digital Journalism
The Trouble with Traffic: Why Big Audiences Aren't Always Profitable 33
week.com also uses Tumblr, including links to a wide variety of sources. By doing
that, the magazine can ``introduce people to Newsweek who would never read
it'' on its site or in print, says Mark Coatney, who worked at Newsweek before
joining Tumblr in 2010. And he says that while Newsweek's Tumblr audience
is smaller than the audience it gets through Twitter or Facebook, its readers are
more engaged.
* * *
The argument about whether it's more important to build large audiences
or engaged audiences has not been settled. Two news organizations that haven't
jumped on the engagement bandwagon are New York Magazine and Newser.
``The notion of engagement has been touted for a number of years, “ says Michael
Silberman, general manager of nymag.com. “This is not important in driving
our business. We want to grow uniques''—that is, the number of users—``so
we're really thinking about the scale. Secondarily, we want to drive page views.''
He might change his mind if nymag.com decided to start charging for online
access, but that isn't on the table for now. ``Engagement only makes sense in a
subscription model,'' he says.
At Newser, an aggregator with about 2.5 million unique visitors a month, the
audience breaks down in ways similar to mainstream news organizations. Executive
Chairman Patrick Spain says about 12,000 users are ``addicted'' and come
to the site many times a day; 225,000 are ``avid'' users who visit Newser many
times a week; and more than 2 million people pass by, with just a click or two.
But Spain argues that the passers-by are useful, because they are more likely than
addicted users to click on ads, though whether clicks on ads are a good indicator
of value is an open question.^{\href{#endnotes-chapter-2}{17}}* * *
For decades, many news organizations enjoyed higher profits dependent
largely on bigger audiences. Magazines and newspapers priced their wares artificially
low to boost circulation, even though that brought a group of lightly
%The Story So Far: What We Know About the Business of Digital Journalism

affiliated readers who had to be lured again and again with cheap come-ons and
giveaways. When this worked economically, it was because advertisers could be
persuaded to buy access to a big audience they didn't know much about. Today,
advertisers have far more choices and far more information. Moreover, many of
the firms competing for ad dollars never would have been defined as ``media
companies'' years ago. Facebook now delivers almost a quarter of all digital advertising
views in the U.S.^{\href{#endnotes-chapter-2}{18}}Search advertising, dominated by Google, soaks up
almost half the dollars spent on online ads.^{\href{#endnotes-chapter-2}{19}}So it is much harder for media companies, new or old, to compete purely on
audience size. They will never grow fast enough to counter the massive numbers
accumulated by giants like Google and Facebook. News organizations have to
offer something more.
When the New York Times announced details of its digital subscription plan
in March 2011, Andrew Swinand, president of global operations for the Starcom
MediaVest Group, a media-buying agency, said it wouldn't hurt—and might
help—the site's advertising revenue.^{\href{#endnotes-chapter-2}{20}}``I'm paying for an engaged audience, and
if that audience is willing to pay, that demonstrates just how engaged they are,''
he said. In a later interview, he added that editors need to start thinking about
engagement in broader terms, not just the amount of time people spend on a
site or the number of pages they click. ``I want to be able to look and say, 'Who
are these people [using a site], and what are they spending their time doing on
it?' '' By doing that, Swinand says, news organizations can help companies feel
more confident that their ad dollars are being spent wisely. The Times' Golden
adds that ``if we all go the way of outdoor advertising [e.g., billboards] where it
depends on who passes by, it'll be hard to build value. Engagement is the proxy
by which people value content.''
Audience size is still vitally important. A site with 10 million unique users will
get more attention from advertisers and agencies than one with a fifth that many.
Large companies want to make mass purchases of ads; they won't deal individually
with a host of small sites. But the chase for traffic has put news organizations
on a sugar high of fat audiences and thin revenue. It has also devalued their
journalism, as they have resorted to such tactics as celebrity photo slideshows to
boost search-driven traffic. In diminishing their brands and commoditizing their
%Columbia Journalism School | Tow Center for Digital Journalism
%The Trouble with Traffic: Why Big Audiences Aren't Always Profitable 35
content, they have fallen short in the crucial goal of attracting engaged, loyal
users. This needs to change. By producing relevant journalism, deploying data
intelligently, and relying on social media—not just search engines—to drive traffic,
they can gather a more devoted and involved readership, one that advertisers
will also prefer.

%Columbia Journalism School | Tow Center for Digital Journalism

\chapter{Local and Niche Sites: The Advantages of Being Small}
TBD.com ran into trouble right from the start. In February 2011, just six
months after going live, the Washington, D.C., area's high-profile experiment in
local online journalism announced that it would lay off half of its editorial staff,
detach its site from its TV-station partner and reinvent itself as a culture-andlifestyle
site. Many of those who did stay on looked for the exit as soon as they
could line up another job.
The reshuffling—which followed the departure of Jim Brady, the former
Washington Post online executive brought in to launch the site—marked the
meaningful end of one of the best-funded and best-pedigreed efforts to make
professional journalism work online. The site, whose name stands for To Be Determined,
had drawn a great deal of attention for the quality of its editorial staff
and for its use of social media.^{\href{#endnotes-chapter-3}{1}}Clearly there was an important lesson here for
other news sites, especially those plying the local or ``hyperlocal'' trade.
Just what that lesson was, though, is in dispute. Does TBD's failure prove that
``hyperlocal journalism is more hype than hope,'' as media analyst Alan Mutter
put it?^{\href{#endnotes-chapter-3}{2}}Or did it mainly signal a failure of nerve on the part of corporate parent
Allbritton Communications, whose CEO, Robert Allbritton, had pledged to
provide a three- to five-year runway to profitability?
It is clear that the site was, as expected, losing money, despite impressive traffic
growth. According to coverage in the Washington Post, unique visitors to TBD.
com had risen from 715,000 in November 2010 to 838,000 in December and
1.5 million in January 2011. But, as the incoming head of the site told the paper,
``It was still not generating enough [income] to offset the hefty costs.''^{\href{#endnotes-chapter-3}{3}}Two insiders
interviewed for this report said the January traffic spike was not as striking
as it looks, because much of it consisted of people looking for information on the
region's heavy snowstorms that month.
%The Story So Far: What We Know About the Business of Digital Journalism

Nobody involved has revealed the precise size of TBD's losses. Saul Carlin, an
Allbritton executive involved with TBD from the start, would say only that ``traffic
and revenue were being closely monitored. As a result, a change in management
was necessary.''
Brady says the picture was not that grim. The site had been budgeted to earn
revenue ``in the low millions'' in year one; he argues it was on track to reach perhaps
75 percent of that goal. ``The situation wasn't great when I left, but it wasn't
catastrophic either,'' he says.
Mutter, a former journalist and media executive, took the shortfall as a demonstration
of a fundamental misalignment between the expense of producing
local reporting and the potential online revenue from it, because of the built-in
constraints of small audiences and puny ad rates. Various hyperlocal flameouts
support this thesis: among others, the Washington Post's Loudoun Extra and the
shuttered New York Times site for suburban New Jersey, whose audience was
handed off to local start-up Baristanet.
Still, mistakes were made. Brady has said repeatedly that resistance from the
site's broadcast partner WJLA, the Allbritton-owned ABC affiliate in Washington,
proved a major hurdle. Both Brady and Steve Buttry, TBD's director of community
engagement, said the TV staff was unhappy to see its website rolled into
TBD.com—which linked out heavily to other media outlets—and that the station
failed to promote TBD wholeheartedly on the air.
``The first time we linked out to another TV site, that was a major collision,''
says Buttry. He adds that TBD linked out more moderately after that and gave the
TV news staff a ``heads-up'' when it planned to point to a story that WJLA had
missed. ``The lesson from our experience is that the legacy culture is powerful
and ingrained,'' he says. ``Whenever its revenue stream might be endangered or
disrupted, it's going to have a big influence on decisions.''
The most important example of that influence was the decision to fold TBD's
ad sales staff into WJLA's, which led to the departure of the small digital sales
team that Brady had assembled. ``Selling digital is hard. I was adamant that the
only way to be successful was to keep the sales force separate,'' he says. The tension
was visible from the start, Brady adds; an example is an ambitious launch
%Columbia Journalism School | Tow Center for Digital Journalism
%Local and Niche Sites: The Advantages of Being Small 39
sponsorship of more than \$75,000 that his sales team planned to pitch to a local
car dealership. WJLA intervened, arguing that it would damage a valuable relationship
between the dealer and the TV station, Brady says.
``When I was at the Post, we were routinely doing six-figure online deals,''
Brady says. ``If the sales force itself doesn't believe digital is worth it, how are they
going to sell it? To just assume nobody would ever spend that much online is
insecurity with your own inventory.''
Just as revealing as what went wrong is the list of things TBD seemed to have
going for it, which help to clarify the intense interest local online news ventures
have drawn for the last five years. Most important was its association with sister
site Politico.com, another well-funded, generously staffed Allbritton venture led
by high-profile news veterans. Like Politico, TBD promised to be not just on the
Web but ``of the Web,'' in Brady's phrase, meaning that it would link abundantly,
deploy social media aggressively and engage closely with users. And like Politico,
TBD promised to deliver to its advertisers a well-defined audience—not just
generic news consumers, but people intensely interested in the particular news it
had to offer. (Still, it is worth keeping in mind that at least through 2009, Politico
earned more than half its revenue from its free, ad-supported print edition,^{\href{#endnotes-chapter-3}{4}}with
a circulation of about 32,000 ^{\href{#endnotes-chapter-3}{5}})
TBD also sought to strike a balance between focus and scale: Visitors would
be drawn in by news about their immediate environs or interests, but behind the
scenes the operation could reap the ``efficiencies'' of serving a large metropolitan
area. (Brady insists ``hyperlocal'' is the wrong word for what was really a regional
site with a neighborhood interface.) Likewise, the site would include a mix of
aggregation and firsthand reporting. To describe the site, Robert Allbritton has
used the analogy of a supermarket bringing together items that previously could
not be found in one place. Before TBD, he declared when the site launched, finding
local news online was ``like trying to buy groceries in the old country. First
you went to the fishmonger, then to the baker, then to the grocer, and so on.''^{\href{#endnotes-chapter-3}{6}}Other hyperlocal ventures have tried to apply versions of that idea on a national
scale. A good recent example is Main Street Connect, a ``national community
news company'' that went live in 2009 and consists, so far, of 10 sites serving
towns in Fairfield County, Conn. The separate sites share editorial resources
%The Story So Far: What We Know About the Business of Digital Journalism

(neighboring communities see many of the same articles), technology infrastructure
and an ad sales team. ``We'll soon be bringing our vision to other groups of
towns,'' Main Street's web site promises. ``Watch for us.''^{\href{#endnotes-chapter-3}{7}}By far the grandest of the new hyperlocal journalism ventures is the nationwide
Patch network, which was bought by a struggling AOL in 2009. With sites
in 700 communities and counting as of March 2011, each led by a local editor
making \$40,000 to \$50,000 a year, Patch has turned AOL into one of the biggest
sources of new journalism jobs in the country. Visitors to a local Patch site
see news and information about a specific community, written and curated by
people in that community, whether it's Dublin, Calif., or Dunedin, Fla. (In March
2011, AOL also bought Outside.in, a hyperlocal network that automatically aggregates
news, blog posts, police reports and other public data, and says it is in
57,830 neighborhoods. Reports suggest AOL was interested in the underlying
technology more than the business itself.^{\href{#endnotes-chapter-3}{8}})
To run all of those Patch sites, AOL can count on centralized resources like
a massive in-house ad network, a sales force with ties to national brands, and
sophisticated search engine optimization technology for maximizing the meaningful
lifespan—and thus the economic return—of every piece of content it
produces. Like McDonald's, AOL uses sophisticated market research to assess the
commercial potential of the communities where it is considering planting the
Patch flag. (According to a New Yorker profile of AOL chief executive officer
Tim Armstrong, the Patch formula considers 59 factors, from average incomes
to voter turnout.^{\href{#endnotes-chapter-3}{9}})
The same infrastructure supports AOL's growing stable of niche or ``vertical''
content sites, anchored by acquisitions such as the martial-arts blog MMAFighting.
com, the tech-industry site TechCrunch and the sprawling Huffington Post,
which AOL bought for \$315 million in 2011. It's no exaggeration to say that
AOL, whose original business model (providing dial-up access to the Internet) is
badly out of date, has staked its declining fortunes on local and niche journalism.
Why? The underlying logic is the same at AOL as it was at TBD: In a world
where much of the daily news has become commodified, only news that people
can't find elsewhere will command a loyal audience. This is hardly a novel insight
%Columbia Journalism School | Tow Center for Digital Journalism
%Local and Niche Sites: The Advantages of Being Small 41
among media analysts, who since the late 1990s have pointed to financial news
as a rare example of the sort of information that people, and advertisers, will pay
for online.
A 2006 report on ``value creation'' in journalism, from Harvard's Kennedy
School of Government, put the lesson bluntly: ``specialize or localize.''^{\href{#endnotes-chapter-3}{10}}As the
report explained, ``Because of the increasing range of information sources, greater
abilities to access material from anyplace at anytime, and requirements to create
tight bonds that lead to loyal consumers, news organizations will have to move
away from the unfocused, something-for-everyone, one-size-fits-no-one news
products characteristic of the second half of the twentieth century.''
Of course, not every news outlet can be the Financial Times or the Wall Street
Journal, with long-cultivated expertise in a valuable and time-sensitive brand of
information. For most, the clearest path to ``adding value'' lies in paying closer attention
to their immediate community. The Harvard report put this drily: ``To be
competitive and create economic value, media will need to increase their differentiation,
and thus exclusivity. The most effective way to do so is to create value
through local coverage that is linked to the lives, aspirations, and understanding
of individuals in the locations in which they live. It is this kind of coverage that
other news providers cannot do well.''
That's the theory. But TBD, Patch, the hyperlocal sites launched by the New
York Times and the Washington Post, and many others like them have yet to
produce a commercially viable proof-of-concept. The list of success stories in
local online news hasn't changed much in recent years; it contains mainly small,
grass-roots community sites. If nonprofit ventures with significant foundation
funding, such as MinnPost and the Voice of San Diego, are removed from the
list, most of what's left are Baristanet, Alaska Dispatch, The Batavian, West Seattle
Blog and a few others.
These ventures vary in their business models and the kind of journalism they
produce. What they have in common is limited resources, a narrow coverage
footprint and no claim to the corporate efficiencies of their larger peers.
* * *
%The Story So Far: What We Know About the Business of Digital Journalism

Alaska Dispatch is a statewide news site launched in 2008 by the husbandand-
wife reporting team of Tony Hopfinger and Amanda Coyne. In 2009,
philanthropist and former U.S. News & World Report executive Alice Rogoff
bought a majority share; the founders stayed on as editors, and the Dispatch
was relaunched with a mandate to build up a newsroom and dedicate itself to
serious political journalism.
Today the site has a full-time editorial and Web staff of 10, up from just two
at launch. It also uses paid freelancers. Total staff costs run in the neighborhood
of \$650,000 per year. The Dispatch saves money by avoiding print and delivery
costs, which is an especially serious expense for dailies in Alaska. In late 2008 the
Dispatch's print rival, the Anchorage Daily News, ended rural air delivery to the
state's remote outposts.
Rogoff says the Dispatch doesn't stint on the costs of covering the Alaskan
frontier. The newsroom is attached to an airplane hangar in Anchorage, and access
to Rogoff 's airplane has made it easier to cover distant events like the Iditarod
sled race. (The Dispatch also drew praise for its reporting on the 2008
Point Hope caribou massacre and subsequent trial.11) Tony Hopfinger says the
site focuses on statewide political news and analysis—exactly what many small
dailies have cut in favor of covering murders and car wrecks.
The site's founders say that commercial success is integral to the mission of the
Alaska Dispatch. Even the ``About Us'' page repeats the message: ``Because the
owners of the Alaska Dispatch believe that journalism must and will ultimately
pay for itself, the site is a for-profit enterprise, relying on online advertising and
sponsorship.'' When she became publisher, Rogoff committed to backing the site
until it turns a profit, which she expected to happen in three years. Now, about
two years in, she won't disclose financial details but says the Dispatch is on track
to meet its goals.
The Dispatch does appear to have found a niche in Alaska's news ecosystem.
Roughly 125,000 unique visitors generate more than 1 million page views each
month, impressive statistics in a state of just over 700,000 inhabitants. According
to Hopfinger, the site has about 30 to 40 advertisers at a time; its ad rates run
from \$150 to \$1,550 per month, with a guaranteed minimum of 75,000 impres
%Columbia
%Journalism School | Tow Center for Digital Journalism
%Local and Niche Sites: The Advantages of Being Small 43
sions. Though pricing is by the month, the site will oblige advertisers who would
rather buy by the amount of traffic—hence the guarantee. And the Dispatch's
modest entry-level rates compare favorably with print alternatives.
One of the Dispatch's biggest challenges has been to forecast the amount of
ad space, or inventory, it will have each month, because this fluctuates greatly
depending on how much traffic the site gets. On occasion, the site has had to
turn advertisers away. To deepen and diversify its inventory without taking on
too much risk in increased editorial costs, Hopfinger plans to bring established
Alaskan blogs into the Dispatch under a revenue-sharing agreement.
At first glance, the statewide profile of the Dispatch seems to set it apart from
smaller local sites. But Rogoff and Hopfinger stress that Alaska's agenda-setters
constitute a kind of village, one spread among Anchorage, Juneau, Washington
and Houston. They share a narrow and well-defined set of interests: federal funding,
government regulation, the oil industry (hence the Houston link), transportation,
Sarah Palin and so on. One of the site's most successful features is its ``Bush
Pilot'' blog, focused on the small-scale aviation so critical to Alaskan life.
``It's like a small town,'' Hopfinger says. ``The flipside of that is there are fewer
people and fewer businesses.'' But he adds that the Dispatch has benefited from
the kind of boosterism that smaller community sites enjoy. ``You get an opportunity
as an underdog. People want to see us succeed,'' he says.
An emphasis on a small and well-defined community sets apart most of the
online-only local news outlets that began to dot the Web about five years ago.
These run, generally, from professionally staffed hard-news outlets such as the
Dispatch to news-oriented community blogs like Baristanet and West Seattle
Blog. Wherever each site falls along that spectrum, none of these grass-roots
ventures has either the assets or the built-in costs of local sites backed by established
newspapers or television stations. (For a detailed discussion of costs,
see Chapter 7.)
The grass-roots sites also face a different set of problems than do large-scale,
networked hyperlocal ventures such as Patch, and, to a lesser extent, the original
TBD. Patch is hardly a legacy newsroom. But it does have to succeed on a
scale that justifies AOL's vast editorial, infrastructural and ad-sales investments
%The Story So Far: What We Know About the Business of Digital Journalism

(and that compensates for the company's declining income as an Internet service
provider.) If only a few of the 700-plus Patch sites take root and thrive in their
communities, that won't be enough for the enterprise to succeed; for the business
to make sense, the bulk of them have to work.
These distinctions suggest two axes for plotting the local news ventures working
online, depicted in the chart shown here. The vertical axis distinguishes online-
only outlets from those that also have traditional print or broadcast assets.
The horizontal axis arrays organizations according to their editorial footprint:
single-site, hyperlocal outlets covering a community or neighborhood; sites or
small site networks covering a cluster of communities; and site networks with a
regional or national scale.
Landscape of local online news
Local and ``hyperlocal'' news sites vary both in their coverage footprint and in their
affiliation with traditional print or broadcast outlets. Many of the success stories in
online journalism appear in the bottom left quadrant: small, grass-roots ventures
without corporate backing or ties to established newsrooms.

%Columbia Journalism School | Tow Center for Digital Journalism
%Local and Niche Sites: The Advantages of Being Small 45
The local news sites on the top half the chart are tied to substantial legacy
operations, whether they are based in one city or spread across a chain of newsrooms
that share back-end resources. These sites enjoy the same advantage in
serving their local audiences that the New York Times site does in delivering
national news online: access to the editorial resources of established professional
newsrooms. But that editorial product published online yields only a tiny fraction
of the ad revenue that it does in print or broadcast. The health of these sites
is effectively wedded to the health of their traditional parents.
The hyperlocal networks in the bottom right quadrant don't have legacy
newsrooms to draw on. They must either build an editorial staff from scratch, like
Patch and Main Street Connect, or cull local information from public sources
and other sites, as do EveryBlock (now owned by MSNBC) and Outside.in.
Their key asset lies, potentially, in uniting hundreds or thousands of hyperlocal
channels with back-end infrastructure for selling and serving advertising.
It is easy to understand the argument that these networks ought to occupy
the sweet spot for hyperlocal news. Like a stable of trade publications or a chain
of small newspapers, Patch can pull together a large audience out of many small
ones. Its size should confer advantages unavailable to local competitors in the
individual markets where it operates: lower costs, better technology, access to bigger
advertisers and so on. And as noted above, those markets have been carefully
selected for their commercial potential.
The bottom left quadrant is the source of the most surprising lessons about
building commercially viable journalism online. The independent, locally grown
news sites that populate this quadrant would seem to be at a clear disadvantage.
They lack the editorial backing from established newsrooms that many competitors
enjoy. Their infrastructure costs—bandwidth, content management, ad serving
and so on—are fixed and cannot be shared across a network. They lack what
has been considered a crucial element of success in the media business: scale.
That several of these grass-roots sites have nevertheless built viable businesses
raises two questions. The most obvious one is how they have managed to make
ad revenue align with expenses. But just as important—and perhaps still to be determined—
is whether their model can support serious accountability journalism.
%The Story So Far: What We Know About the Business of Digital Journalism

* * *
The city of Batavia, N.Y., is unlikely to show up on AOL's carefully calculated
list of promising Patch sites. It is a Rust Belt community of just 16,000 people in
the western end of the state, about 50 miles from Buffalo. The prison system is
one of the few growth sectors in what was once a thriving industrial center, and
Batavia's downtown merchants have struggled to compete with big-box retailers.
When The Batavian's website went live in May 2008, the local paper, the
Batavia Daily News (owned by the Johnson Newspaper Corp.) didn't have a
website. (It does now.) Gatehouse Media, which publishes small dailies, weeklies
and ``shoppers'' around the country, launched The Batavian as an experiment in
online-only publishing. In addition to hiring two reporters, who are no longer
there, Gatehouse provided its in-house digital guru, Howard Owens, who became
the new site's publisher. In early 2009 Gatehouse laid Owens off and he
assumed ownership of The Batavian, which runs as an independent site with no
editorial or business ties to other publications.
Three years after going live, The Batavian, according to Owens, is profitable. It
offers a promising example of local online journalism. The site has grown from
fewer than 2,000 unique visitors per day in 2008 to roughly 6,000 now, generating
close to 600,000 page views each month. Owens won't say what it costs to
run, but The Batavian operates with a skeleton staff: Owens, his wife, and two
part-time employees, in addition to freelancers who are paid a small sum per
story. The site posts about five short reported stories per day, and additional bulletins
or photo pieces.
Most impressively, The Batavian has about 100 advertisers at any time—up
from just three in 2008—and pulled in between \$100,000 and \$150,000 in ad
revenue in 2010, Owens says. He aims to double ad income in 2011 and to hire
one or two full-time employees. ``I don't really have to sell anymore—they call
me,'' he says of local advertisers. ``It's driven by word of mouth.''
One factor driving that word of mouth is The Batavian's modest advertising
rates. The site eschews pricing by traffic completely; instead it charges a flat fee
of \$40 to \$260 per month (though one premium package runs as high as \$400).
%Columbia Journalism School | Tow Center for Digital Journalism
%Local and Niche Sites: The Advantages of Being Small 47
Owens estimates that a month's run on his site would buy just one day's placement
in his print competitor, the Daily News. ``I wanted it to be an easy decision,''
he says. ``What's another \$200?''
Online, of course, the supply of space available to sell to advertisers increases
with traffic, because ``impressions'' are the unit of measurement. A small site like
The Batavian would be hard-pressed to support 100 advertisers with the top-ofpage
banner model used by many large metro dailies. With one major banner per
page, each sponsor would get just 6,000 impressions per month, or \$30 worth at
a fairly generous \$5 cost-per-thousand. (In 2010 comScore found12 that average
newspaper CPMs were \$7 nationwide, though its analysis included the largest
newspapers and newspaper chains in the country; small local outlets tend to have
lower rates.) At that rate, the site would earn \$36,000 a year. If there were two big
banners on each page, annual revenue would rise to \$72,000.
Instead, Owens runs the site like a ``pennysaver''—every advertiser appears on
each page, in long columns running down both sides of the site. Their positions
rotate during the day to make sure every merchant spends some time near the
top. And to encourage scrolling, every article appears in full on the site's front
page, with the most recent items at the top. It is possible to absorb all of the day's
news without ever clicking beyond the home page. Owens doesn't have to worry
about driving traffic to various corners of the site to deliver impressions to different
advertisers. He designed this approach based on his experience at three
newspaper companies, with access to online data for more than 100 local papers.
``I saw that it's very hard to get people to move past the home page,'' he explains.
``So I decided to base my business on that.''
A national brand probably would not place its ad alongside 100 others. But
Batavia's merchants have the sense that they are sponsoring a popular local resource
at a reasonable price. Local boosterism makes a difference, Owens insists:
``Some advertisers just want to support community.'' His advertisers rarely ask
about click-through rates, though Owens says some are pleased to learn that, say,
a total of 80 people clicked on their ad over the course of a month.
%The Story So Far: What We Know About the Business of Digital Journalism

A useful metric for evaluating this approach is not CPM, but RPM—revenue
per 1,000 impressions. Assuming The Batavian earned \$125,000 in 2010 (the
middle of the range Owens claims) and averaged 600,000 page views per month,
it achieved an RPM of \$17, an impressive figure for a news site serving a small
and far-from-affluent community.
* * *
A similar formula applies at Baristanet, one of the most successful local news
sites in the country. Baristanet has one decisive advantage: its audience of affluent,
media-savvy professionals in the retail-rich bedroom communities of suburban
New Jersey, anchored by the towns of Montclair and Maplewood. (The area
scores well on AOL's algorithms—six of the seven towns Baristanet serves have
their own Patch sites.)
Baristanet, launched in 2004, keeps costs radically low. Everyone involved with
the site has another job—even the two founders, Liz George and Debbie Galant.
``Everyone's freelance,'' George explains. She and Galant act as top editors, giving
the final word on every article; one other editor is paid by the month. The rest
of the site's dozen or so freelancers—many of whom moonlight from salaried
jobs—are paid by the piece, usually about \$50 each. ``We don't want long articles,''
George says. ``If they spend half the day on a story, that's too long.''
Then there are people who write for free, submitting opinion pieces, comments,
bulletins and photos. Baristanet offers roughly the mix of content that a
community weekly would; one Friday in March 2011, for instance, some political
news about local budgets was sandwiched between pieces on ``weekend
highlights'' for kids and a major markdown at the local cheesemonger. George
explains that many smaller items require no reporting at all, just a photo and a
blurb. The combination of paid articles, opinion, aggregation, and ``things that
come in over the transom,'' yields more than enough material to keep the site
fresh, she says.
%Columbia Journalism School | Tow Center for Digital Journalism
%Local and Niche Sites: The Advantages of Being Small 49
Altogether, editorial costs run to \$5,000 or \$6,000 per month—higher than at
The Batavian, but still fairly modest for a site that runs about eight longer articles
per day and, according to George, attracts 80,000 unique visitors monthly. Most
important, costs are far below the roughly \$20,000 in advertising that George
says Baristanet pulls in each month. For several years, according to George, the
site's profits have provided a sizable second income for the two founders and
their hired editor.
Baristanet also eschews cost-per-thousand pricing in favor of a simple calendar
model, and, though rates run higher than at the Batavian, an advertiser can get
on the site quite cheaply. Merchants pay from \$150 to \$1,600 per month (weekly
rates are also available) depending on their ad's size, placement and frequency
of rotation. George says businesses in the area have no interest in buying by the
impression or by the click, though Baristanet does report such statistics to them.
Because Baristanet rotates ads across its available inventory, a merchant's exposure
is limited by the amount of traffic the site gets. According to George, consultants
have advised against ad rotation, but so far the hospitals, car dealerships,
real estate agents, restaurants and other businesses that advertise on the site don't
seem to mind. Merchants occasionally call wondering where their ad is; George
advises them to refresh the page a few times until it appears.
As a result, Baristanet achieves an enviable ratio of revenue to traffic. With an
average monthly volume of about 475,000 page views, the site enjoys an RPM
in the neighborhood of \$42—many times the revenue it would get if it used a
standard CPM model. Just as The Batavian's revenue would collapse if merchants
complained about being stacked together on a single page, Baristanet could not
do the business it does each month if it had to guarantee a hard number of impressions
to each advertiser.
It seems fair to assume that the site's appeal to advertisers is not tied to such
narrow statistics. George suggests that merchants are paying relatively little to be a
part of a one-of-a-kind community resource that enjoys wide recognition in the
towns it serves. Baristanet claims to have 53 percent household penetration in its
core market of Montclair and has logged more than 300,000 comments since its
%The Story So Far: What We Know About the Business of Digital Journalism

inception—roughly one for every other person in all of Essex County(though
many come from repeat commenters). ``It's been an easy sell,'' George says. ``Everybody
wants to partner with us.''
* * *
That local online journalism can succeed in such different environments—
prosperous suburban New Jersey and Rust Belt upstate New York—is an indication
that it can be viable elsewhere. A 2010 survey of 66 ``promising local news
sites'' around the country, conducted by the Reynolds Journalism Institute at the
University of Missouri, found that the top objective of these sites was to ``produce
original news'' and that on average nearly half of their content came from
paid staff, rather than, for instance, aggregation or reader contributions. ^{\href{#endnotes-chapter-3}{13}}Advertising was far and away the most important revenue source for these sites,
accounting for 45 percent of revenue on average. (Foundation grants came next,
at 17 percent of revenue, and reader donations followed at 12 percent.) For 28 of
the sites surveyed, advertising supplied three-fourths or more of annual revenue.
Fifty-six percent of the sites operated as for-profit ventures, and of these, half
reported making a profit the previous year. (It is important to remember these
results are entirely self-reported.)
Clear lessons emerge from the experiences of Baristanet, The Batavian and the
Alaska Dispatch. First, all three sites have embraced calendar-based advertising
pricing systems that yield more revenue than they could expect pricing strictly
by the number of impressions. Low prices, anecdotal successes and a sense of
community engagement allow local merchants to find value on terms that a
national advertiser might reject out of hand. The sites have managed to appeal to
local advertisers by selling in terms that work for them. ``A lot of advertisers don't
understand CPMs,'' says Victor Wong, CEO of PaperG, a company that helps
publishers attract local ads. ``They don't understand what a page view means, they
don't know when the page ran, they don't trust CPM measurement.''
%Columbia Journalism School | Tow Center for Digital Journalism
%Local and Niche Sites: The Advantages of Being Small 51
But it would be a mistake to see in these examples a formula that any local
venture could replicate just by asking merchants for a few hundred dollars each
month. Each of these sites filled a vacuum when it launched and has remained
popular even as new competitors have appeared. Their real feat is having built
sizable audiences on the cheap. The same is true of niche or ``vertical'' sites that
aim for a particular demographic segment or ``community of interest,'' rather
than a geographic area.
Henry Blodget's Business Insider (reviewed in detail in Chapter 7) offers a
good example: The financial news site reached break-even last year by building
a monthly audience of 6 million unique visitors, on a yearly budget of about \$5
million. An even more dramatic example is DailyCandy, the decade-old trendsurfing
email newsletter that occupies roughly the same journalistic space online
that Lucky magazine does in the print world.
DailyCandy was launched in March 2000 from the kitchen table of Dany Levy,
then a young editorial-side veteran of New York magazine and Lucky. Levy's
venture offered one of the most bare-bones editorial propositions imaginable: a
short daily email alerting readers to something hot—a new cupcake shop, a shoe
style—in New York's (and the Internet's) fast-changing retail culture.
``One simple thing in your e-mail inbox that told you one thing you needed to
do that day,'' Levy explained to a Harvard Business Review blog in 2009. ``It was
meant to save people time and keep them plugged in. Not everyone can afford to
eat at Mario Batali's new place, or some other hot, new restaurant, but this kind
of knowledge is cultural currency. It's water-cooler conversation.'' ^{\href{#endnotes-chapter-3}{14}}That interview came after her company had been bought by Comcast, in the
summer of 2008, for a reported \$125 million. By the time of the sale, Daily-
Candy had grown from a one-person shop to a company with 55 employees,
running 12 editions across the country and reaching a total audience of 2.5 million
people—most of them women, and two-thirds of them younger than 35.
Financial details were scarce, but an internal email from early investor (and veteran
of MTV and AOL) Bob Pittman reportedly said the company would reach
\$25 million in revenue in 2008, with profits of \$10 million.^{\href{#endnotes-chapter-3}{15}}Analysts had been
speculating eagerly about what the company might be worth since 2006, when
the Wall Street Journal reported it was on the auction block at \$100 million.^{\href{#endnotes-chapter-3}{16}}%The Story So Far: What We Know About the Business of Digital Journalism

Those numbers put DailyCandy in a different league financially from the local
news ventures profiled in this chapter. But the dynamic that makes DailyCandy
work was visible years earlier, when the newsletter was a grass-roots venture with
much smaller ambitions. Levy launched her business with \$50,000 in savings and
\$250,000 raised from family and friends. The first edition went out to just 700
people, mostly friends or colleagues of Levy, then readership grew explosively. In
2001 the newsletter was already paying for itself, with tiny ads in each emailed
edition as well as separate sponsored emails straight from advertisers.
By 2003 the subscriber list had grown to 285,000—more than 400 times its
starting audience, a stunning ratio for so-called organic growth achieved with
minimal outside support. It was on the basis of these numbers that Pittman
made his initial investment in the business, reported to be ``in the single-digit
millions,'' which in turn fueled the newsletter's expansion into new markets
and new editions. ^{\href{#endnotes-chapter-3}{17}}In a broad sense, the experience of successful local and niche sites bears out
the received wisdom that media ventures in today's hypercompetitive landscape
must ``specialize or localize.'' But only a fraction of online news outlets that pursue
this strategy ultimately succeed. Defining and attracting a desirable audience
is necessary, of course, but not by itself sufficient; acquiring that audience on a
tight budget is what sets successful grass-roots ventures apart from the also-rans.

%Columbia Journalism School | Tow Center for Digital Journalism

\chapter{The New New Media: Mobile, Video and Other Emerging Platforms}
News organizations can be forgiven for feeling that they're in an endless cycle
of Whac-A-Mole.
They've had 15 years to get onto the Internet, and for much of that time the
experience was limited largely to words and photos on a Web page, accessed on
a personal computer. But more recently, journalism has been blessed and bedeviled
by a stream of follow-on innovations. As a result, most organizations have
tried to develop new ways to report and distribute stories, and many are making
substantial investments so their work will appear on attractive new devices. Their
hope is that these new kinds of digital journalism will enhance companies' earnings;
their fear is that if they don't adapt, they will lose audiences' attention and
the revenue it brings.
It hasn't been easy. Video has been seen as a great way to get more sustained
engagement, but many news organizations have found it to be expensive and difficult
to produce. And even though ad rates are three to five times what regular
display ads bring, video often doesn't get enough traffic to attract substantial revenue.
Mobile devices, meanwhile, provide consumers with greater access to news,
but the small screen size can be a nightmare for designers and a poor display space
for advertisers. Tablets—particularly the iPad—have looked like a more immersive
experience for readers, and a more likely venue for subscriptions and higher
ad revenue. But their luster has dimmed as the dominant manufacturer, Apple, has
insisted on charging high fees and controlling economically valuable information
about customers. Each new device brings an additional level of complexity and
expense. Not long ago, ``convergence'' was the keyword in news production, as
television, newspaper, magazine and pure online sites all started to look the same.
Now comes a new ``divergence,'' in which online journalism organizations must
distribute news into distinctly different modes of presentation.
%The Story So Far: What We Know About the Business of Digital Journalism

The iPad has been a hit: Expectations are that about 30 million devices will
have shipped by the end of 2011.^{\href{#endnotes-chapter-4}{1}}But while analysts expect the iPad to capture
more than 90 percent of the tablet market in 2011, competitors are entering
the fray. Tablet manufacturers have announced that more than 25 brands will be
available in 2011, at screen sizes ranging from five to 11 inches.^{\href{#endnotes-chapter-4}{2}}Audiences are fragmenting in other ways, too—in their interests and habits.
The conflict is evident in the behavior of Michael Harwayne, vice president of
digital strategy and development at Time Inc. Harwayne lives in Manhattan and
takes the subway to work. He likes to read the Wall Street Journal on his commute,
but lately the question has been: Which format works best?
Harwayne likes the Amazon Kindle. When the Journal became available on
that device in the spring of 2009, he decided to pay an extra \$10 per month for
the convenience, though he was already paying \$363 per year for the print and
web editions. The Kindle price rose to \$15 per month a year later.
But there was no connection between his print/online subscription and the
Kindle edition, and Harwayne found it annoying to be billed separately. Then,
in May 2010, came the Wall Street Journal iPad app, which he got as part of
his overall subscription. It's not that it was ``free,'' but because he didn't have
to pay a separate fee, it felt free. Still, he said, ``I didn't like carrying it and
reading it on the subway, so I never actually canceled the Kindle until I had a
lightweight alternative.''
In the fall of 2010, he was introduced to a Samsung tablet. He liked it more
than the iPad, so he switched to the Journal's Android version (which is also
included in his core subscription to the paper) and finally canceled his Kindle
subscription. But Harwayne wondered, ``Why shouldn't I be able to read the
paper on any device I have?''
In March 2011, he learned that he'll eventually have to pay an additional
\$17.29 a month to keep his iPad or Android version of the Wall Street Journal.
Through it all, the one thing that hasn't changed is that Harwayne likes to read
the print version of the Journal after a long day at work. Still, it ``seems like a very
strange consumer strategy,'' he says.
%Columbia Journalism School | Tow Center for Digital Journalism
%The New New Media: Mobile, Video and Other Emerging Platforms 57
Early data about tablets appeared to show great promise for news organizations.
A few months after the first iPad went on the market in the spring of 2010, the
Associated Press reported that Gannett Co. was getting \$50 per thousand page
views for iPad advertisements—or five times the price it was getting on its websites.
^{\href{#endnotes-chapter-4}{3}}Condé Nast initially said visitors were spending an hour with each of its
iPad issues—far more than the three or four minutes per visit its websites draw
and close to the overall print magazine average of 70 minutes.^{\href{#endnotes-chapter-4}{4}}David Carey, president
of Hearst Magazines, said in March 2011 that the company would end the
year with ``several hundred thousand subscriptions in total'' sold through digital
publisher Zinio, Barnes & Noble's Nook e-reader and Apple's iPad. He predicted
that as much as 25 percent of his company's subscribers will be on tablets ``in the
next five years.'' ^{\href{#endnotes-chapter-4}{5}}But making predictions based on early and volatile sales is tricky. The data on
usage of tablets and smartphones come from products—and a competitive environment—
that are in transition. Many buyers are early adapters to technology, a
group whose behavior does not reliably predict the greater population who will
eventually buy the gadgets. (Of the 66 million smartphone users in the U.S., only
about a third have used the browser or downloaded an app, according to audience
measurement company comScore.^{\href{#endnotes-chapter-4}{6}})
Wired, a magazine with a tech-savvy readership, sold 100,000 single copies via
the iPad in June 2010, but that number dropped to 22,000 by October; in 2011,
its single-copy iPad sales have averaged 20,000 to 30,000 per issue.^{\href{#endnotes-chapter-4}{7}}Wired's average
monthly circulation of 800,000 still consists mostly of print subscriptions
and single-copy sales; a small number (27,000) are sold as PDF-based digital
replicas.^{\href{#endnotes-chapter-4}{8}}It's likely that the high price and one-issue limit of Wired's iPad version
have hindered sales; one copy costs \$4.99—the same as a copy sold at the newsstand—
while an annual print subscription starts at around \$12 a year. And the
froth has settled throughout the company: In April 2011, a Condé Nast publisher
told AdAge that the company's iPad strategy was slowing: ``They're not all doing
all that well, so why rush to get them all on there?''^{\href{#endnotes-chapter-4}{9}}One issue is Apple's own pricing strategy. The company announced in early
2011 that it wants a 30 percent cut of any subscriptions paid through the iTunes
store.^{\href{#endnotes-chapter-4}{10}}More important, when the user pays with a credit card stored in iTunes—
%The Story So Far: What We Know About the Business of Digital Journalism

and Apple had about 200 million registered users in 2011 ^{\href{#endnotes-chapter-4}{11}}—the user's name and
address don't have to be shared with the publisher, unless the customer agrees.
Without this information, publishers have a handicap: They can't find out the
particulars of their subscribers' reading behavior. Google is pushing an alternative
tool for subscriptions called One Pass, in which the company will charge
publishers 10 percent of revenue and share subscribers' names and information.
Some publishers, such as Time Inc., have built their own payment and collection
systems for selling their own apps from their own websites so they don't have to
share any information or pay any fees.^{\href{#endnotes-chapter-4}{12}}For most magazines, neither the replica digital copies nor the iPad versions of
their magazines count in the ``rate base,'' which is the number of readers publishers
guarantee to deliver to advertisers. So, for now, publishers tout it as ``bonus''
circulation they can't really charge for.
Some news organizations are optimistic about the economics of mobile devices.
In March 2011, Dow Jones announced that it had 200,000 paying subscribers
who access to the Wall Street Journal via some sort of mobile device.^{\href{#endnotes-chapter-4}{13}}The
company did not say how much additional revenue this brought in, so many of
these readers could be like Michael Harwayne—digital subscribers who signed
up for mobile access. (The Wall Street Journal's total reported average daily paid
circulation is about 2 million copies—1.6 million copies in print and 430,000
electronic copies.^{\href{#endnotes-chapter-4}{14}})
Time Inc. announced plans in February 2011 to give Time, Fortune, People
and Sports Illustrated subscribers the ability to access those magazines' content
on multiple platforms.^{\href{#endnotes-chapter-4}{15}}Sports Illustrated has been particularly aggressive in digital
expansion; to introduce its digital package as widely as possible, it has given
access to all 3.15 million of its current print subscribers. For new customers, it is
promoting an ``All Access'' subscription plan, which includes the print magazine,
plus access via tablet, web and smartphone; in March 2011, the price for All Access
(including a bonus windbreaker) was \$48 per year.^{\href{#endnotes-chapter-4}{16}}A digital-only package
with no magazine and no jacket costs the same. That pricing scheme helps protect
the print edition and provides the biggest possible digital audience.
Some publishers are willing to invest a lot to gamble on an unknown future
and avoid sitting on the sidelines.
%Columbia Journalism School | Tow Center for Digital Journalism
%The New New Media: Mobile, Video and Other Emerging Platforms 59
In February 2011, News Corp. launched The Daily, a tablet-only newspaper.
First offered on just the iPad, though there are plans to extend it to more tablets,
The Daily announced a start-up budget of \$30 million, which let it hire enough
journalists, designers and technicians to create 100 pages of content per day. Integrated
into the Daily are features that seem to shine on the iPad platform, such
as social-media links, audio and video. Greg Clayman, publisher of the Daily, said
hundreds of thousands of users have downloaded the app, but he's not ready to
reveal how many use it on a regular basis.^{\href{#endnotes-chapter-4}{17}}George Rodrigue, managing editor of the Dallas Morning News, says the iPad
``may be the thing that helps people read enterprise journalism. We used to say
you have to be platform agnostic. I don't think that's right anymore. You have
to be platform specific.'' But the transition isn't cheap. ``We have to build a staff
for the iPad—two people plus an assignment editor,'' he says. ``We're going to
handcraft little stories summarizing every story—55 words max. Every reporter
will write the summary themselves. Every section front will be a summary of the
news page.'' At the Miami Herald, Raul Lopez, the interactive general manager,
estimates the paper's total digital page views to be 30 million per month; about 2
million are mobile, and half of those are via the iPhone.
* * *
Meanwhile, video has become an essential element of the digital experience.
According to comScore, about 89 million people in the U.S. watched at least one
online video, or video advertisement, daily by the end of 2010.^{\href{#endnotes-chapter-4}{18}}For journalism sites, broadband access has made video distribution more feasible.
But persuading users to watch news video isn't easy. Online video journalism
is becoming a world of haves and have-nots. Among the haves, CNN.com
reigns supreme.
``In any given month, well over half our unique visitors watch video,'' says
K.C. Estenson, senior vice president and general manager of CNN.com. ``The
percentage has gone up every year.'' When CNN.com redesigned its site in 2009,
%The Story So Far: What We Know About the Business of Digital Journalism

the network ``anchored it on video,'' Estenson says. It isn't unusual for the website
to have 15 or more video links on its home page, including at least three or four
in high-profile featured positions.
CNN.com is different than other video-rich sites because of the size and expectations
of its audience. The company says it delivers between 60 million and
100 million video streams a month. In contrast to local-broadcast competitors,
CNN.com can match the costs of substantial technology and newsgathering
with a massive audience. ``If you do not have scale, you do not have a business,''
says Estenson. CNN.com has also launched a free iPad application in which
video is integrated into text. And CNN is hungry for more viewers. Estenson
says CNN can sell ``almost every single video impression we create. So we would
like more video consumption.''
CNN has brought its broadcast expertise onto the Web. ``We program by times
of day,'' says Estenson. ``The bell curve of visitors to our site peaks between 11
a.m. and 2 p.m. on regular news days. We've noticed that social media links go up
a lot at night.'' Sometimes CNN makes programming decisions based on a mix
of demographic and editorial priorities. Estenson notes that users under the age
of 25, who ``are disproportionately on social media,'' tend to be more interested
in entertainment and features rather than in hard news. So, while ``CNN is all
about trust and reliability,'' Estenson says, ``for CNN.com, entertainment is one
of the biggest sections of the site.''
Estenson sees contrasts between the content on CNN's cable channel and
CNN.com. Online, producers have more freedom to experiment and expand,
and the content can be more daring. ``Going online is a private experience, versus
watching in the living room. When people are online, they seem to gravitate to
things that are more provocative than they would if they were in a room with
their friends.''
How profitable is CNN online? According to a presentation the network made
to analysts in mid-2010, revenue for CNN overall (consisting of the U.S. and international
divisions, Headline News and digital) was about \$500 million; digital
advertising and content sales accounted for about 10 percent of total revenue.^{\href{#endnotes-chapter-4}{19}}CNN.com's profit margin isn't broken out publicly by either its division, Turner
Broadcasting System Inc., or by the parent company, Time Warner Inc. Estenson
%Columbia Journalism School | Tow Center for Digital Journalism
%The New New Media: Mobile, Video and Other Emerging Platforms 61
does say the digital operations have been profitable for the last seven years—even
including corporate costs and just under 100 ``dedicated digital'' people on the
editorial side. All of CNN benefits from a centralized publishing platform; sales
and administrative expenses are spread across the three television divisions and
their online properties.
CNN.com is dependent mostly on ad revenue that is sold directly by a CNN
sales force. But the digital business also gets some direct corporate support. Estenson
says a ``select portion'' of the subscription money that cable and satellite companies
pay to carry CNN's programming is used for research and development
activities related to new technologies, such as smartphones, tablets and televisions
connected to the Web. Thus, benefits from CNN's digital investments flow both
ways, from the traditional to the digital and back.
Estenson believes the site feeds viewers and value back to the legacy network.
``Digital platforms are the entry point for the brand. More and more people will
discover the brand through them,'' he says. And he envisions some of the distinctions
between the two platforms becoming less relevant, especially as the Internet
gets a bigger foothold in living rooms.
CNN.com is something of an exception in its success with online video. Most
local TV stations' websites have far less traffic and revenue. In interviews with executives
at a station based in one of the top five metropolitan areas of the country,
a more difficult picture emerges. (The station was willing to share metrics on the
condition it not be identified.) In October 2010, the station's website attracted
about 7 million page views. But it delivered only 622,000 video streams that
month. The station's general manager said that when video became feasible on
the Web several years ago, executives believed they would have a natural competitive
advantage. ``We thought having video was the key to becoming a popular
website. But only 10 percent of the visitors look at video.'' And partly as a result
of the low video usage online, only 1 percent of the station's total advertising
revenue comes from its website.
The general manager has seen statistics that show similarly paltry results at
other local television stations' sites. One problem, he says, is that other sites are
also producing video, so broadcast stations face more competition for views.
Sports fans appear to be particularly interested in newspaper sites' video.^{\href{#endnotes-chapter-4}{20}}He has
%The Story So Far: What We Know About the Business of Digital Journalism
%62
thought about doing more consumer research but can't justify the expense. ``No
one is buying the site, really,'' he says, ``so it's not worth spending more money to
figure it out.''
Somewhere between this station's frustration and CNN.com's success is LIN
Media, a company that owns 32 local television stations in markets ranging
from Springfield, Mass., to Albuquerque, N.M. LIN delivered 116 million video
streams in 2010, and has built its business on shared operations and costs, as well
as long-term investment in branding and marketing.^{\href{#endnotes-chapter-4}{21}}LIN is in 17 markets, and the company has multiple stations in several cities. It
is in small or medium-sized ``DMAs''—that is, ``designated market areas'' that are
defined and ranked by Nielsen, the audience-rating company. The markets range
from Indianapolis (25th largest in the country) to Providence, R.I., (53rd) to
Lafayette, Ind. (191st).^{\href{#endnotes-chapter-4}{22}}LIN controls costs by having one building, one staff, and
one newsroom per market, with costs shared by all its stations in that market. For
non-local topics such as health, LIN produces stories that serve all of its markets.
Companywide, LIN has about 200 digital employees in a workforce of 2,000.
(Four years ago, it had nine digital employees in a workforce of 2,300.) Robb
Richter, LIN's senior vice president for new media, says the video the company
produces for broadcast is its competitive advantage. ``We have mounds of video
we can use''—something most other sites lack.
Print-based media are still building video resources and expertise. Even those
with successful sites have had a hard time winning a video audience. Michael
Silberman, general manager of New York Magazine's popular site, says nymag.
com has tried to integrate video into its editorial content but that the pace of
video and the commitment to watch it still aren't working for most visitors. The
site features videos it produces in-house as well as material from around the Web
that is ``curated'' by online editors, based on subject matter, relevance and news
value. So, for example, nymag.com has a food video page with categories beyond
news, such as restaurants, chefs and recipes. But it's hard to build an audience,
Silberman says, adding, ``If your site isn't about video, people don't click on it.''
For nymag.com, less than 10 percent of unique users go to video. At Huffington
Post, no more than 5 percent of unique visitors clicked on a video throughout
most of 2010.^{\href{#endnotes-chapter-4}{23}}%Columbia Journalism School | Tow Center for Digital Journalism
%The New New Media: Mobile, Video and Other Emerging Platforms 63
Lewis DVorkin, chief product officer of Forbes Media, agrees. ``Video on the
web is hard. It's very stressful,'' he says. Since joining Forbes in the spring of 2010,
DVorkin has been coming up with new ways to create and distribute Forbes
content. But video has him stumped: ``We had a difficult video strategy. It was
conceived on the broadcast model—produced, highly expensive, and it involved
lots of people.'' He foresees moving to outside contributors more in video, as he
has done elsewhere on Forbes.com. DVorkin and nymag's Silberman both say
that until they unlock the video puzzle, they are losing opportunities for ad revenue.
Silberman says, ``Our ad demand outstrips our audience demand.''
The Wall Street Journal's site has drawn significant traffic and revenue to its
video offerings. The site serves around 8 million streams a month, says Alan Murray,
deputy managing editor and executive editor, online, for the Journal. And the
ad rates are healthy—\$30 to \$40 per thousand views (or CPMs). The site features
live videos before and after the market closes, and they're often displayed prominently
on the home page. Unlike much of the rest of the site, viewing the videos
doesn't require a subscription.
The key to making video profitable, Murray says, is controlling costs. WSJ.com
has about 16 people devoted to video production, and the company has trained
many of its print reporters in basic video techniques. ``If you go to Bahrain and
need a satellite truck, that's \$25,000,'' Murray says. ``All we need is a \$200 iPhone
4.'' WSJ.com has also managed to get viewers to watch video during work hours,
something that many other sites have found difficult.
The Miami Herald recently noted that its video traffic grew by 25 percent in
2010.^{\href{#endnotes-chapter-4}{24}}Videos that the Herald produces and hosts on its site get about 200,000
streams a month, says Lopez, the interactive general manager. That is a relatively
small number, given that the site gets more than 6 million visitors per month.
The Herald has bolstered its video presence with segments produced by the
Associated Press and other organizations. It also tried to distinguish itself by producing
longer videos of newsmakers talking about various topics, but those are a
tough sell online. ``There's a reason that television does two-minute stories,'' the
Herald's managing editor, Rick Hirsch, told Poynter.org. ``Unless something is
super compelling, people's attention span is relatively short, and it's even shorter
%The Story So Far: What We Know About the Business of Digital Journalism
%64
on a small screen.'' And it's not easy to make the advertising numbers work. Lopez
says Herald-produced videos earn just \$4,000 per month—that is, 200,000
streams a month at \$20 per thousand viewers.
Cynthia Carr, senior vice president of sales at the Dallas Morning News, also
doubts video will become a significant profit driver any time soon. ``We're not
monetizing video. I don't see that bringing big revenue,'' she says. Dallasnews.
com got an average of 186,000 video streams a month in 2010, clicked on by 2
percent of the unique visitors coming to the site.
And video traffic is hard to anticipate. Like Hollywood, there are hits and
flops—as the Detroit Free Press found in January 2011, when it ran a dramatic
video of a shootout in a local police station.^{\href{#endnotes-chapter-4}{25}}It got nearly 714,000 streams, or
nearly half the total traffic to video on Detroit's site for a three-month period.
When it launched, the video was preceded by a short commercial. But within
seven hours and 70,000 streams, a reader who went by the name ``HartlandRunner''
posted this comment: ``I am glad you are willing to tell this story by showing
the video, but why the ad beforehand? Brutality brought to us by UnitedHealth
Care? ... Very, very tacky, even in an online world. … This tape was paid for by
taxpayers and shows graphic real violence … and you guys put an ad on it. That's
unbelievable and I hope you change it.''
Nancy Andrews, Detroit Free Press's managing editor for digital media, said, ``I
saw the comment and checked with the vice president of advertising. She said
let's take the ad off.'' So they removed the ad, and forfeited some revenue—but
kept readers happy. And it did help editors understand even more about how
news video works on the Internet. ``People are interested in the raw video content,''
Andrews says. ``Show me what happened. … You don't necessarily need the
context in video form, too. … Think of it more like a picture that talks than a
full story.''

%Columbia Journalism School | Tow Center for Digital Journalism

\chapter{Paywalls: Information at a Price}
``Information wants to be free. Information also wants to be expensive. Information
wants to be free because it has become so cheap to distribute, copy,
and recombine—too cheap to meter. It wants to be expensive because it can be
immeasurably valuable to the recipient. That tension will not go away. ... Each
round of new devices makes the tension worse, not better.''
—Stewart Brand, The Media Lab, 1987 ^{\href{#endnotes-chapter-5}{1}}``The Internet is the most effective means of giving stuff away for free that
humanity has ever devised. Actually making money from it is not just hard, it
may be fundamentally opposed to the character and momentum of the net.''
—John Lanchester, London Review of Books essay, 2010 ^{\href{#endnotes-chapter-5}{2}}When the Wall Street Journal decided to charge for its online edition in 1996,
the company did so without a great deal of deliberation. Rather, as Peter Kann,
who was then the chief executive officer of the Journal's parent company, Dow
Jones, would later recall, ``I didn't know any better. I just thought people should
pay for content.''^{\href{#endnotes-chapter-5}{3}}That was a novel idea at the time—that people should pay for news they got
on the Web. Today, after years of declining print circulation and disappointing
online ad revenue, many news organizations have begun pondering whether to
institute a subscription system for their online sites.
Pondering is still all most companies have done, though increasingly they are
warming to the idea of charging for at least some of their digital content. Their
hesitation stems from several concerns. Some are fearful they will lose so much
Web traffic that their online advertising revenue will fall significantly; others are
daunted by the technological hurdles involved in getting a new online subscriber
system to work in tandem with the one that has served print customers for years.
%The Story So Far: What We Know About the Business of Digital Journalism
%68
Also, subscription revenue has historically been such a small factor in the addriven
media business that many news organizations wonder if they would ever
get much return on the investment.
Publishers usually cite three reasons to charge for online products. One, of
course, is to increase subscription revenue. Another, less obvious, is to stanch the
erosion in legacy operations: That is, since their readers now get the content they
want for free online, why would they pay for a print subscription? If you start
charging for digital access, shouldn't that protect your more profitable print business?
Finally, there is evidence that a paying audience is more valuable to advertisers
because it demonstrates deeper commitment by those readers.
A few online-only news organizations have tried pay schemes, usually to
charge for premium content beyond their free websites. Politico launched its
``Pro'' version in early 2011, charging \$2,495 a year for in-depth coverage of such
topics as energy or health care.^{\href{#endnotes-chapter-5}{4}}That puts Politico into competition with older
publications like Congressional Quarterly, now owned by the Economist Group,
and newcomers like Bloomberg Government. At a much lower price, ESPN.
com offers access to its ``Insider'' site, with exclusive blogs, videos and tools, at
prices ranging from \$30 to more than \$70 a year. And to ensure there's a bundle,
online subscribers also get ESPN The Magazine, a biweekly print publication.
The paywall issue is especially acute for newspaper sites. In the months leading
up to publication of this report, most of the attention of journalists was directed
at the New York Times' new digital subscription service. Before that, though, the
conversation about paywalls in the U.S. has focused on two staunch believers in
the digital subscription business: the Wall Street Journal, which began charging
in 1996 shortly after its website launched, and the Arkansas Democrat-Gazette,
which started imposing online subscriptions in 2002.
Walter Hussman, publisher of the Arkansas paper, has portrayed his site's paywall
as a way to protect the more lucrative print edition. The online subscription
service ``does not justify itself as a revenue stream,'' Hussman has said.^{\href{#endnotes-chapter-5}{5}}Print subscribers
get the online edition for free.
%Columbia Journalism School | Tow Center for Digital Journalism
%Paywalls: Information at a Price 69
The Wall Street Journal sees it differently and has consistently charged print
subscribers extra for digital access. And the difference between those strategies
is manifested in the publications' number of digital subscribers: WSJ.com has
around 1.1 million subscribers (including those who also get the print edition),
or a bit more than half of its print base. The Democrat-Gazette has around 4,400
subscribers to its ``electronic edition''—about 2 percent of its daily circulation
base. Its print circulation, though, has remained remarkably steady while that of
other papers has declined precipitously. In 2006, the Democrat-Gazette's daily
circulation was 176,910. Daily circulation now is listed at 186,962, though some
of that strength is due to a merger of operations with some small Arkansas papers
whose subscribers are now counted in the Democrat-Gazette's total.^{\href{#endnotes-chapter-5}{6}}But how replicable are these two models? The Wall Street Journal provides
content geared toward financial decision making and reaches a more elite and
affluent audience than most news organizations. The Arkansas paper is the dominant
news organization in its state.
To see how news executives figure out whether to charge online, we examined
the decision-making processes at two large metro newspapers—the Dallas
Morning News and the Miami Herald. Each thought about the same issues,
relied on similar data—and then embarked upon completely different strategies.
Both papers have histories as journalistic powerhouses in their home markets.
The Herald, which has been owned by McClatchy since 2006, has won 20 Pulitzer
Prizes, on subjects ranging from local election fraud to the Iran-Contra
scandal. The Morning News has won nine Pulitzers and has dominated the Dallas
market since its parent company, Belo, bought and closed the rival Times
Herald in 1991.
But both have experienced significant declines in their print circulation, and
both had reason to believe that their free websites might be partly to blame.
At the Herald, circulation had been steadily declining for years. The Herald and
its Spanish-language sibling, El Nuevo Herald, fell from a combined daily circulation
of 393,382 in 2005 to 261,657 in 2009. Most of the decline was outside
the Herald's ``city zone''—its core in Miami-Dade County. The Herald has also
cut back discounted bulk circulation to schools, hotels and other institutions. ^{\href{#endnotes-chapter-5}{7}}%The Story So Far: What We Know About the Business of Digital Journalism
%70
The trend has been similar in Dallas, where the Morning News has dropped
from 373,586 daily circulation in 2007 to 264,459 in 2010. In part, that is also
because the paper began to focus on its most loyal print subscribers a few years
ago. The News trimmed back most of its delivery beyond a 100-mile radius of
Dallas, though it still circulates in Austin, the state capital, which is 200 miles
away. ``We reduced footprint in the state,'' says John Walsh, senior vice president
for circulation at the Morning News. ``Advertisers were saying they're not interested
outside the core market.'' That helped eliminate some extraneous expenses.
It took so long to get newspapers to Odessa, about 350 miles west of Dallas, that
the delivery person ``had to spend the night in a motel after delivering the paper,''
Walsh says. ``It was like the Pony Express.'' The Morning News also eliminated
much of its single-copy sales effort, removing 9,000 of its 10,000 newspaper
racks around the metro area.
A few years ago, the News began doing studies about the price sensitivity of
its subscribers. Executives wanted to know if the remaining readers were now a
core of the faithful who would be willing to pay much higher prices for home
delivery. One study indicated that a 40 percent hike in the price of a subscription
would result in a loss of around 12 percent of its subscribers, says Publisher James
Moroney. That emboldened executives to raise the price of a monthly subscription
aggressively, from \$21 to \$30 in May 2009 and then to \$33.95 in 2011—one
of the highest prices for any metropolitan paper in the country.
In those days, Moroney was convinced that free digital access was the way
to go. In May 2009, he told a U.S. Senate committee holding hearings on the
state of the newspaper business that ``if The Dallas Morning News today put up
a paywall over its content, people would go to the Fort Worth Star-Telegram.''^{\href{#endnotes-chapter-5}{8}}Within a few months, though, Moroney began reconsidering his aversion to a
paywall. In remarks in the fall of 2010 to a small group at the Carnegie Corporation,
Moroney provided this analysis: ``The Morning News does 40 million page
views a month. If we could sell out three ad positions on every page every day at
a \$7 CPM, we would yield \$10 million'' a year.^{\href{#endnotes-chapter-5}{9}}That, he noted, would cover less
than a third of his editorial costs—even as those costs have dropped as newsroom
staffing has fallen from 660 at its peak to around 400.
%Columbia Journalism School | Tow Center for Digital Journalism
%Paywalls: Information at a Price 71
More fundamentally, Moroney had concluded that a focus on volume—either
in the form of cheap print subscriptions or of Web traffic that generated insufficient
revenue—had damaged the news industry's economic vitality. ``What I
most fear about this obsession with volume is it underlies the persistent belief
that if we will just grow sufficiently large audiences online, then eventually we
will sell enough advertising to be sustainably profitable,'' he told the group. He
added, ``There is more supply [of online ads] than there is demand. And the explosive
growth of social media only ensures this imbalance of supply and demand
will persist for a considerable period of time.''
Others at the Morning News noticed that traffic to the Web site had grown
as print subscription rates rose. Why pay more for print, some readers seem to
have reasoned, when you can get the same news free online? ``We found when
we raised the price of the paper, a lot of people migrated to dallasnews.com,'' says
Executive Editor Bob Mong.
The News launched an aggressive pricing scheme for its digital content in
February 2011. People who don't subscribe to the paper must pay \$16.95 a
month to get access to the Web, iPad and iPhone versions of the Morning News.
Print subscribers already paying \$33.95 a month get unlimited access to any
digital edition.
It is a paywall, but not an absolute one. Stories that strike the editors as ``commodity''
journalism—such as breaking news, or weather and traffic updates that
could easily be found elsewhere—are free to all. More proprietary or exclusive
journalism requires a subscription. (Currently, about half of the stories on the
site's home page have open access.)
When Moroney announced the pay plan, he and his staff were predicting that
page views would drop by 40 to 50 percent. ``I'm not confident we're going to
succeed,'' he told Nieman Journalism Lab. ``But we've got to try something.''^{\href{#endnotes-chapter-5}{10}}In
an interview a few weeks before the paywall launched, he portrayed the strategy
as a way to help return journalism to one of its former, and highly profitable,
roles as a one-stop storehouse of local news. ``At least for a period of time, you
can restore the bundle,'' he said.
%The Story So Far: What We Know About the Business of Digital Journalism
%72
In late April 2011, six weeks after the pay plan launched, the News did see
traffic declines—though less, so far, than Moroney had predicted. Unique visitors
were down 17 percent, and page views declined 28 percent, compared to the
same period in 2010. Mark Medici, director of audience development for the
News, declined to disclose how many new digital subscribers had signed up, but
did say that 27 percent of print subscribers had enrolled for digital access.
Traffic declines were also on the minds of Miami Herald executives when they
debated whether to institute a pay plan.
The Herald did a survey on its site in October 2009 to determine users' willingness
to pay for its content. It was a voluntary and thus unscientific poll; nevertheless,
the results didn't inspire a great deal of confidence. Fifteen percent said
they'd pay for unlimited access; an additional 23 percent said ``maybe.'' The dollar
amounts weren't meaningful, though; less than 5 percent said they would spend
more than \$10 a month.
Another survey question asked readers if they would make a ``voluntary financial
payment'' to support the Herald's site. Nearly a third said they were very or
somewhat likely to do so, and so a few weeks later, the Herald's site instituted a
``tip jar,'' attaching this plea to many pages on the site: ``If you value The Miami
Herald's local news reporting and investigations, but prefer the convenience of
the Internet, please consider a voluntary payment for the Web news that matters
to you.'' Says Armando Boniche, the Herald's circulation director: ``We got about
\$1,000 to \$2,000 total. McClatchy [the Herald's parent company] had us pull it
after six weeks.''
Meanwhile, the Herald increased print subscription prices, though not to the
extent that Dallas did, and stopped discounting the paper in Broward County, just
north of its home market. And the Herald made a few smaller price-enhancing
moves, such as charging 50 cents a week for an insert with TV listings and \$1
extra for the ad-filled Thanksgiving Day newspaper. (Still, old habits die hard. In
January 2011, the Herald was offering six months of seven-day delivery for just
77 cents a week—a whopping 83 percent discount from its stated price.)
%Columbia Journalism School | Tow Center for Digital Journalism
%Paywalls: Information at a Price 73
The Herald also did some paywall calculations, modifying formulas provided
by the Newspaper Association of America. In 2009, when the study was prepared,
Miamiherald.com was attracting around 3.88 million unique visitors and
25.2 million page views a month. Its advertising mix was typical of many news
organizations of its size. The Herald's own ad department sold 42 percent of the
total space available on the site, at prices averaging slightly over \$13 per 1,000
views. An additional 36 percent of the available advertising space on the site was
sold as ``remnant''—very cheap—ads, under \$1 CPMs. And 22 percent of the ad
inventory on the site went unsold altogether.
The Herald first modeled what would happen if it imposed what Boniche calls
a ``10-foot wall'' that would require a 99-cent monthly subscription for anyone
to read anything on the site. The company predicted page views would fall by
91 percent, and total revenue from the site would drop by 76 percent. In other
words, new subscription revenue wouldn't come close to compensating for the
ad dollars that would vanish as the audience contracted.
Herald executives mapped out several scenarios in which they could institute
a paywall and match the results they were getting with a free site whose
income was entirely from advertising. But all of the ideas required substantial
leaps of faith.
One scenario, charging just 99 cents a month for digital access, would require
the Herald to attract 335,000 subscribers—about 30 percent more than the combined
daily print circulation of the English- and Spanish-language newspapers.
Another option: The Herald could make do with only 50,000 digital subscribers,
but it would have to charge them nearly \$120 a year—almost as much as a Wall
Street Journal online subscription. Or, the site could enroll 50,000 subscribers
at a more reasonable price (99 cents a month), but the paper would have to get
advertisers to pay an impossible six to ten times its current rates for online ads.
Given how remote any of those possibilities seemed, the Herald analysis suggested
that the most sensible approach to a paywall would be a hybrid model
with 1 percent of users—about 38,000—paying \$1.99 a month for unlimited access,
and nonsubscribers getting a great deal of access as well. That would preserve
the site's traffic and advertising. But the revenue boost from digital subscriptions
%The Story So Far: What We Know About the Business of Digital Journalism
%$74
would be less than \$1 million a year, and that sum, which represents less than 1
percent of the company's overall revenue, didn't seem worth the investment in
time, marketing and other costs.
* * *
One publisher whose digital subscription base has grown substantially is the
Financial Times.
The FT started charging for access in 2001 and had a modest number of online
subscribers for many years, getting to 126,000 online subscribers in 2009, slightly
less than a third of its print subscription base.^{\href{#endnotes-chapter-5}{11}}Subscriptions leapt to 207,000
in 2010, or more than half the number of print subscribers. And digital access
isn't cheap—the FT charges \$259 a year for a standard subscription and \$389 for
premium access to more content deep within the site.^{\href{#endnotes-chapter-5}{12}}The growth is tied to a change in strategy. Nonsubscribers used to be able to
come to FT.com and read 10 free stories without registering; after registering,
they could get 30 more stories a month before the subscription requirement
kicked in. (This is similar to the ``metered'' approach that was put into effect in
2011 by the New York Times.) The FT toughened its policy in 2007 by preventing
nonsubscribers from getting any stories without registration and limiting
them to 10 stories a month before the paywall rises.
So, the wall has become less permeable. But Rob Grimshaw, managing director
of FT.com, says there is a more fundamental change at work: Managers ``used to
approach it as newspaper marketing;'' now they realize they ``are direct Internet
retailers.''
That means using behavioral targeting to determine which of the nearly 3
million nonpaying, registered users are most likely to subscribe and directing appeals
to them. ``What topics are people reading? We developed a dynamic model
to determine readers' propensity to subscribe''—one that is constantly shifting,
with changes being made ``on a daily basis,'' Grimshaw says. ``We're spending the
same amount on marketing as we used to, but we more than doubled our rate
of acquisition.''
%Columbia Journalism School | Tow Center for Digital Journalism
%Paywalls: Information at a Price 75
The FT has also been aggressive about shutting down ``leakage,'' as Grimshaw
puts it—that is, unauthorized copying of stories. And when it comes to offering
free content, ``we're more controlled than WSJ.com,'' which offers free access to
most of its stories via Google News and many stories at no charge on its home
page.
The FT's approach is a testament to the possibilities of paid content, but it also
demonstrates how hard it is even for a premium publisher to extract revenue
from digital advertising. When the FT's parent company, Pearson, reported results
in early 2011, it noted that for the FT Group, 55 percent of its revenue comes
from ``content/subscriptions'' while 45 percent comes from advertising.^{\href{#endnotes-chapter-5}{13}}A decade
ago, the FT earned 74 percent of its revenue from ads, and only 26 percent
from subscriptions.
``The outlook for the ad business online is quite bleak,'' says Grimshaw. ``There's
just not enough money there.'' As a subscription site with a select audience,
FT.com can charge higher rates for ads than general-interest sites. ``We can create
scarcity in a marketplace that has no scarcity,'' he says. ``In that light, subscriptions
and ads are complementary.'' But given that FT.com doesn't use networks to fill
up unsold ad space at discount prices, Grimshaw says ``I'd be surprised if we sell
50 percent'' of the site's inventory.
* * *
After years of internal debate, the New York Times has entered the realm of
pay-for-access. If its audacious and complex plan succeeds, that will likely encourage
many other publishers to follow suit.
This isn't the first time the company has tried online subscriptions. In 2005,
the Times launched its TimesSelect service, charging those who didn't get the
print edition \$49 a year to access opinion pieces. After a fast start, with more than
120,000 subscribers signing up in two months, the plan stalled, and the Times
closed it down two years later; executives said the \$10 million a year the service
was generating wasn't enough to compensate for the lost traffic and ad revenue.^{\href{#endnotes-chapter-5}{14}}%The Story So Far: What We Know About the Business of Digital Journalism
%76
So why would the Times take a new gamble to charge for digital access? Part
of the answer lies in how dramatically the company's revenue mix has changed
in recent years.
In 2005, the New York Times Media Group, which is composed primarily of
the Times' paper and website, generated nearly \$1.9 billion in ad and subscription
revenue; about a third of that came from circulation. Five years later, ad revenue
had dropped by nearly \$500 million, while circulation revenue had increased
because of aggressive price hikes for home delivery and newsstand sales. Today,
circulation revenue for the group almost equals advertising revenue.
The Times' website is tremendously popular, but digital ads have been growing
unevenly and don't come close to making up for the shortfall in print ad sales.
Indeed, the site, with more than 30 million monthly unique users in the U.S.,
contributes less than 20 percent of the Times' overall revenue.
%Columbia Journalism School | Tow Center for Digital Journalism
%Paywalls: Information at a Price 77
So the Times devised a pay scheme that it hoped would be porous enough to
allow occasional readers (around 85 percent of the total) to browse the site for
free, but priced aggressively enough to generate significant revenue from its most
devoted readers.^{\href{#endnotes-chapter-5}{15}}It was a difficult plan to execute—requiring more than 14
months, and reportedly costing tens of millions of dollars.^{\href{#endnotes-chapter-5}{16}}When the Times introduced the plan in March 2011, many found it to be
unnecessarily complex. Users are supposed to be limited to 20 stories a month
before they hit the wall. But because there are so many exceptions depending
on how one accesses the site—for example, via Google, Twitter or a blog—even
some experts are befuddled by the plan. Staci Kramer, editor of paidcontent.org,
which covers the digital media industry, wrote that ``the logistics are far more
complex than anything should be that doesn't require a degree in quantum physics.''
^{\href{#endnotes-chapter-5}{17}}There are different rates for online, smartphone and tablet access, ranging
from \$195 to \$455 a year for the full package. Consumers can't get annual, or
even monthly, subscriptions, because everything is priced in four-week increments.
And the price led one commentator to headline his blog post, ``The New
York Times is Delusional.''^{\href{#endnotes-chapter-5}{18}}The Times is unusual among big publishers in that it doesn't require print subscribers
to pay anything to access its digital editions. Both WSJ.com and FT.com
have long charged everyone for online access, on the theory that digital editions
offer utility, archives and tools that the print edition can't. The Times is offering
its print readers a sweet deal: Weekend-only subscribers can pay as little as \$327
a year, and in the bargain get a digital package worth almost a third more. Times
executives insist this isn't an effort to prop up the company's more lucrative
legacy revenue. ``We didn't make this decision to bolster print,'' Janet Robinson,
the Times' company CEO, said shortly after the pay plan debuted. ``We made
this decision to create a new revenue stream.''^{\href{#endnotes-chapter-5}{19}}But given the way the offer is
structured, it's hard to argue that the two aren't closely tied. The pricing—which
is higher for tablets than for the Web—also reflects Apple's decision to take a 30
percent cut of subscriptions purchased through iTunes.
A few weeks after the Times instituted the pay plan, Robinson reported that
more than 100,000 people had signed up for digital subscriptions.^{\href{#endnotes-chapter-5}{20}}Most of
those were enrolled for the introductory offer of 99 cents for the first four weeks,
%The Story So Far: What We Know About the Business of Digital Journalism
%78
according to a person close to the situation, so it isn't clear how that will play
out when those subscribers start getting billed up to \$35 for every four weeks of
unlimited access.
In the Times' own story on its plan, a senior editor called the plan ``essentially
a bet that you can reconstitute to some degree the print economics online.''^{\href{#endnotes-chapter-5}{21}}In
fact, though, it is as much an effort to restore print economics to the print edition,
by providing extra value to subscribers and giving them one less reason to
forgo the lucrative newspaper for the digital edition.
* * *
And then there is the Newport Daily News, a 12,000-circulation newspaper
in Rhode Island.
In 2009, the News decided that it was almost impossible to make money
from digital ads. ``The people we hired to sell advertising on the Internet just
never did very well,'' the paper's then-publisher, Albert ``Buck'' Sherman, told
Nieman Journalism Lab.^{\href{#endnotes-chapter-5}{22}}So the News took an unusual step: Print subscriptions
were priced at \$145 a year, print/online combos at \$245 and online-only
access would cost \$345.
In other words, by forgoing the paper, a digital subscriber was on the hook for
an additional \$100. And Sherman wasn't coy about the rationale: ``Our goal was
to get people back into the printed product.''
Some online-only content, such as videos and blogs, is outside the paywall; the
same goes for columns like ``Clergy Corner'' and ``Advice on Pets.'' But anyone
who wants access to the electronic edition, which reproduces the day's paper,
must pay. The company also operates a free site, newportri.com, designed to appeal
to tourists and others looking for recreational or entertainment information.
In early 2011, the News dropped the price for print and online to \$157, or
a dollar a month above the print-only fee. But online-only access remains at
\$345—a price that current publisher William Lucey III says, in an interview, ``is
more of a deterrent.'' The amount was based on a scenario in which, ``if everyone
wanted only a digital product, this is what it would cost.''
%Columbia Journalism School | Tow Center for Digital Journalism
%Paywalls: Information at a Price 79
The paper's site, newportdailynews.com, gets around 80,000 visitors a month.
Especially with online ad rates ``dropping 20 percent a year,'' that's not enough
to sustain the operation, which includes a newsroom of 22 people, Lucey says.
Indeed, online ad revenue accounts for only 2 to 3 percent of total advertising
for the paper.
After the change was put into effect, ``our single-copy sales went up about 300
a day''—a bit less than 10 percent of overall single-copy sales. As the economy
improves, ``print is coming back. February [2011] was up 35 percent over last
year'' in ad sales.
And even with AOL's free Patch site moving into town, Lucey says there are
no regrets. ``We found our comfort zone, and we stopped agonizing about it.''
AOL, which launched the Patch site in Newport in July 2010, is sanguine: ``The
Newport Daily News does great work and has been a staple in Newport County
for generations,'' says spokeswoman Janine Iamunno. ``There is room for all of us.''
* * *
So, which approach is best, free or paid?
Pay proponents often put it this way: High-quality journalism costs a great deal
to produce, so users ought to pay to get it. Pay opponents have a counterargument:
Paywalls cut sites off from ``the conversation'' online and will deprive them
of the attention they need from blogs, aggregators and social media.
We prefer to frame it as a business issue—and in that respect, it's possible that
neither side has it exactly right. In fact, pay plans may have little immediate impact
on sites that are just getting into the business. The reason is that most companies
are likely to have only small streams of online circulation revenue, which
could roughly match advertising declines from lower traffic. Digital subscriptions
may pay off in the years to come, but only if media companies can persuade consumers
using new platforms—like smartphones and tablets—to adopt a pay plan.
Even before the Internet, subscription revenue didn't amount to much for
most news organizations. Print publications often underpriced subscriptions because
they believed they could lose money on circulation and make it up on
%The Story So Far: What We Know About the Business of Digital Journalism
%80
advertising from larger audiences. Broadcast TV and radio were free, and fees
for cable stations like CNN are buried in bills that make it impossible to discern
the true costs of content. So in the old world, Americans weren't used to paying
much for news; in the digital world, news organizations have spent 15 years training
their consumers to be freeloaders.
As a result, most people are happy to pay nothing at all for news, even as they
have come to accept paying for other forms of digital content. A 2010 study of
1,000 adults commissioned by AOL showed that about four in 10 people pay for
``online content''—but that was a broad definition, including music and video. ^{\href{#endnotes-chapter-5}{23}}Only 4 percent said they pay for online news. A Pew Project for Excellence in
Journalism survey in January 2010 found little interest in paying: Among ``loyal
news consumers, only a minority (19 percent ) said they would be willing to
pay for news online, including those who already do so and those who would
be willing to if asked.''^{\href{#endnotes-chapter-5}{24}}Another Pew study in January 2011 showed 23 percent
of respondents who say they would pay \$5 a month to get full access to a local
newspaper online; that dropped to 18 percent when asked if they would pay \$10
per month.^{\href{#endnotes-chapter-5}{25}}Some say such surveys miss the point. Porous pay systems like the New York
Times' are being erected precisely so they will capture only the most devoted users.
And a hypothetical question in a poll might not capture true sentiment. ``Don't
survey based on what people say they would pay,'' says Aaron Kushner, an investor
who is mounting a bid to buy the Boston Globe from the New York Times Co.
``No one expects to pay for news, so why would they answer differently?''
But even if pay schemes attract users, it's hard to charge enough to produce a
great deal of revenue. Kushner argues that most publishers are making the same
mistake now that they made years ago. ``The problem is they're basing the price
on cost or history rather than value. Forget pricing on cost,'' he says. If anything,
he says, digital editions should be more valuable because of their archives and
interactivity. ``Figure out what is the value of the product and then price against
it. Publishers have been undervaluing their product for too long.''
There is, in some publishers' pay plans, an aura of frustration over the inability
to convert large online audiences into advertising revenue. Moroney, of Dallas,
is simply being more candid than most when he notes that much of the News'
%Columbia Journalism School | Tow Center for Digital Journalism
%Paywalls: Information at a Price 81
online ad space goes unsold, and so a cut in traffic to the site will have little financial
impact. Others, such as Albert Sherman of the Newport Daily News, frame a
paywall as a way to protect the print edition, but at most papers, some circulation
has already been lost because of free alternatives.
The best chance to make headway with pay schemes is likely with a device that
people can hold in their hands. For most mobile phones and tablets, a commerce
system is already in place, and the transaction is straightforward. Moreover, consumers
have shown a willingness to pay for content on mobile devices, whether
that involves ringtones or sports videos. So, if publishers really hope to expunge
the ``original sin'' of giving away content free online, they may be best positioned
to do so not on the computers where they first gave away their wares, but on
mobile devices that offer a more welcoming environment.


\chapter{Aggregation: `Shameless' – and Essential}
A group of middle school students at Brooklyn's Urban Assembly Academy
of Arts and Letters got a special treat one March afternoon in 2011. Just five
weeks after the announcement of the \$315 million deal in which AOL acquired
Huffington Post, AOL's chief executive, Tim Armstrong, and Arianna Huffington,
HuffPost's co-founder, came to the school to teach a class in journalism.
The lesson—or what one could see of it in the short, treacly video account
that ran on the Huffington Post—may have told more about the future of the
news business than what either Huffington or Armstrong intended.^{\href{#endnotes-chapter-6}{1}}A few moments
after the video begins, an official of the program that arranged the visit
speaks to the camera: ``We are delighted that Arianna Huffington and Tim Armstrong
are going to be teaching a lesson on journalism.'' What the video showed,
though, wasn't a lesson in how to cover a city council meeting, or how to write
on deadline. Instead, the teacher in the classroom told her students, ``We're going
to give you headlines that we pulled from newspapers all over the place, and you
guys are going to place them and decide what type of news they are.''
This, then, was a lesson in aggregation—the technique that built Huffington's
site up to the point that AOL wanted to buy it.
In just six years, Huffington has built her site from an idea into a real competitor—
at least in the size of its audience—with the New York Times. The Huffington
Post has mastered and fine-tuned not just aggregation, but also social media,
comments from readers, and most of all, a sense of what its public wants. In the
process, Huffington has helped media companies, new and old, understand the
appeal of aggregation: its ability to give prominence to otherwise unheard voices
and to bring together and serve intensely engaged audiences, as well its minimal
costs compared to what's incurred in the traditionally laborious task of gathering
original content.
%The Story So Far: What We Know About the Business of Digital Journalism

HuffPost's model has provoked sharp criticism from, among others, Bill Keller,
executive editor of the New York Times, who, like Captain Renault in ``Casablanca,''
appears shocked that aggregation is going on.^{\href{#endnotes-chapter-6}{2}}``Too often it amounts
to taking words written by other people, packaging them on your own website
and harvesting revenue that might otherwise be directed to the originators of
the material,'' Keller wrote. ``In Somalia this would be called piracy. In the mediasphere,
it is a respected business model.'' He wrote this even as the Times' own
site has demonstrated the power of aggregation in many ways, notably in a blog
called The Lede, which has deftly captured the tempo and texture of such ongoing
stories as the protests in Iran and the upheaval in Egypt by blending Times
reporting with wire reports and original material from outside sources.^{\href{#endnotes-chapter-6}{3}}In fact, almost all online news sites practice some form of aggregation, by linking
to material that appears elsewhere, or acknowledging stories that were first
reported in other outlets. An analysis of 199 leading news sites by the Pew Project
for Excellence in Journalism found that most of them published some combination
of original reporting, aggregation and commentary and that the mix differed
considerably depending on the management strategy, the site's history and—to
be sure—its budget.
Pew categorized 47 of the sites it surveyed as aggregators/commentators and
152 as primarily producers of original content. In the aggregator/commentary
group, fourth-fifths of the sites were online-only; of the original-content group,
four-fifths were connected to traditional media.^{\href{#endnotes-chapter-6}{4}}Traffic is highly concentrated at
the top of the list, with the top 10 sites accruing about 22 percent of total market
share. Seven of the top 10 sites are ``originators.''^{\href{#endnotes-chapter-6}{5}}What is surprising is that consumers use these different kinds of sites quite
similarly. Original-content sites do marginally better at keeping visitors for longer
stretches and leading them to more Web pages, but it is hard to imagine that
this slightly higher engagement is enough to help cover the costs of original
production and reporting.
%Columbia Journalism School | Tow Center for Digital Journalism
%Aggregation: 'Shameless' – and Essential 85
Huffington often says that aggregation benefits original-content producers as
much as it does the aggregators.^{\href{#endnotes-chapter-6}{6}}The story of a recent blog post on New York
Magazine's site makes for a good illustration.
At about the time that Huffington and Armstrong were visiting the school
in Brooklyn, Gabriel Sherman, a contributing editor to New York Magazine,
was nailing down a scoop. Under the headline, ``Going Rogue on Ailes Could
Leave Palin on Thin Ice,'' Sherman reported that Roger Ailes, the head of Fox
News, had warned his paid commentator, former Alaska Gov. Sarah Palin, not to
go forward with her video accusing the media of ``blood libel'' in the way they
portrayed conservatives after the shooting of Arizona Rep. Gabrielle Giffords^{\href{#endnotes-chapter-6}{7}}The story required at least three days of reporting and editing work. The facts
had to be bulletproof. The post went live on nymag.com's Daily Intel column at
7:57 p.m. on March 13.
The next morning, an editor for the Huffington Post spotted the item and wrote
a rendition of it for that site, publishing at 8:27 a.m^{\href{#endnotes-chapter-6}{8}}Huffington Post played by
the rules: It credited Sherman by name and gave nymag.com a link at both the
beginning and the end of the item. What Huffington Post took from Sherman's
post—237 words, or about half the original length—would be justifiable under
almost any definition of copyright.
The power of aggregation soon became clear: The original Sherman post
drew nearly 53,000 readers on nymag.com, and about 17,500 of them came directly
from the links on Huffington Post. Smaller numbers of readers came from hotair.com, which is part of a network of conservative websites and publications;
and from Andrew Sullivan's popular blog, The Daily Dish. All told, three-fourths
of the traffic to Sherman's story came from other sites. The item also drew more
than more than 130 reader comments on nymag.com, which is far higher than
what the typical blog post gets.
The real winner, though, was Huffington Post. Its aggregated version of the
item got more than 2,000 comments. Comments are not a perfect proxy for
traffic, but it appears that the Huffington Post item got a much bigger audience
for its post than the original New York Magazine item, for a fraction of the cost.
As this example shows, links from other sites or search engines are among the
cheapest and most efficient ways to bring in new users. Even the largest news
suppliers, such as Time.com or CNN.com, appreciate what top billing on You-
Tube^{\href{#endnotes-chapter-6}{9}}or Google News can do to increase traffic and advertising revenue. ``There
are ways you can deliver better ad results, but you can't do it if you focus on your
own content only and not others,'' Scout Analytics Vice President Matt Shanahan
says. He adds that when sites promote each others' content, they create more engaged
audiences through additional page views and commentary. ``The advertiser
wants the audience,'' he says. ``And the audience wants the audience.''
* * *

%Columbia Journalism School | Tow Center for Digital Journalism
%Aggregation: 'Shameless' – and Essential 87
News organizations have always blended material from a variety of sources
by combining editorial content from staff, news services, and freelancers; adding
advertising; and then distributing the package to consumers. In the digital
world, news aggregation is not so different. It involves taking information from
multiple sources and displaying it in a readable format in a single place.^{\href{#endnotes-chapter-6}{10}}Digital
aggregation businesses can be successful when they provide instant access
to content from other sources, and they generate value by bringing content to
consumers efficiently.
The cheapest way to aggregate news is through code and algorithms, with little
or no human intervention. The way an individual story is displayed throughout
the day is determined automatically, typically according to how recently the article
was published and how popular it becomes. Aggregation is slower and more
expensive when it becomes ``curation,'' involving humans in the filtering and
display processes.
Google News belongs to the most basic aggregation category, called ``feed'' aggregation,
in which an algorithm sorts news by source, topic or story and displays
the headline, a link and sometimes a few lines from the original story. The costs
are low.
Yahoo News is an enhanced aggregation feed; it has always had some level of
editorial management in the selection and placement of stories—though it posts
up to 8,000 stories a day, so editorial involvement is fairly minimal. Like Google
News, Yahoo News aggregates from across the Web, but it gives preference to the
approximately 200 media companies from which it licenses content—such as the
Associated Press, Reuters and ABC News. In return for the content, Yahoo News
gives the partners a share of its ad revenue—in addition to sending them traffic.
Traffic to the news sections of Yahoo and Google is relatively small compared
with the total traffic of these companies' sites. For example, in one week in April
2011, Yahoo News represented about 6.5 percent of the total traffic to all Yahoo
sites as determined by the online audience measurement company Hitwise.^{\href{#endnotes-chapter-6}{11}}But
Google News and Yahoo News are the first stop of the day for significant numbers
of users, and that is considered a good predictor of multiple visits and customer
loyalty. Yahoo and Google also let individual users customize their home
pages by personal preferences—according to topic or news source. In the latest
%The Story So Far: What We Know About the Business of Digital Journalism

refinements, Yahoo has introduced a recommendation engine for stories called
LiveStand, while Google introduced ``News for You,'' which keeps track of what
stories a user has clicked on and provides related content.^{\href{#endnotes-chapter-6}{12}}Large media companies
such as the New York Times and Washington Post and independent companies
like Flipboard are doing much the same thing—developing programs that
can recommend stories and videos based on a user's previous choices. Because of
the wide variety of topics and the enormous volume of stories posted, this is a far
more difficult problem than creating the algorithms that Amazon or Netflix use
for recommendations. Some companies are also working on adding friends' and
networks' reading choices to the recommendation engine.^{\href{#endnotes-chapter-6}{13}}On the other end of the spectrum, Huffington Post starts with algorithmic
selections but puts them into the hands of human editors who set priorities for
sections and then condense, rewrite or bring several organizations' versions of the
same story together. HuffPost turbocharges the formula with a mix of social media,
dynamic packaging, and photos and charts. These techniques lead to praise,
criticism—and parody. Comedy Central's Stephen Colbert told viewers that, to
retaliate for HuffPost's republishing without permission the entire contents of his
show's website, he would create ``The Colbuffington Repost,'' that is, the entire
Huffington Post, just renamed. Its re-re-packaged content would make him the
owner of the ``Russian nesting dolls of intellectual theft.''^{\href{#endnotes-chapter-6}{14
}}Newser.com, the aggregation site co-founded by Michael Wolff, represents
much of what legacy media companies hate about the Web: It has little original
reporting, and its stories are short rewrites of information from several other sites,
with a design that emphasizes graphics. Wolff, a media critic who is now editor of
Adweek, has spent much of his career playing provocateur—and driving people
in the media business a little crazy. This effort is no different. Andrew Leonard,
a writer for Salon, wrote a story called ``If the Web doesn't kill journalism, Michael
Wolff will.'' Leonard says that Newser displays a ``truly precious degree of
shamelessness. … Even the slide shows are repackaged, rewritten and abbreviated
versions of content originated by other publications.''^{\href{#endnotes-chapter-6}{15}}Newser says out loud in its slogan what many aggregation sites hope their users
will infer: ``Read Less, Know More.'' Its co-founder and executive chairman,
Patrick Spain, says the site aims to limit its stories to 120 words. ``The most time
%ColumbiaJournalism School | Tow Center for Digital Journalism
%Aggregation: 'Shameless' – and Essential 89
consuming part of editorial is identifying which stories we are going to carry,''
he says. ``And we have to identify the one, two, or three major sources to use to
write the story.'' Wolff asks, ``If you are a consumer, why would you go to a single
source?'' The New York Times, he says, ``used to be seen as a broad view of the
news'' but is now regarded as ``parochial and limited.'' Newser publishes about 60
stories, or digests of stories, per day, though it has at times published as many as
100. ``Cost is less of a driver than the effect we are looking for,'' says Spain. ``If you
have hundreds of articles, it is not an editorial function; it is a fire hose function.''
Newser has business offices in New York and Chicago, but its writers are freelancers.
They live in the U.S., Europe and Asia, and generally work from home.
There are four full-time and about 15 part-time staff members who perform
editorial duties, working at rates of \$20 to \$40 per hour. Spain says ``this is a
gigantic edit staff compared to Digg'' (a site where story placement depends on
readers' votes). ``They have no editorial people. But this is tiny compared to the
New York Times.''
For its other functions, Newser has eight full-time employees who work on
marketing, administration and management, and technology. Its total operating
costs are about \$1.5 million per year, for a site with 2.5 million unique visitors
a month.^{\href{#endnotes-chapter-6}{16}}Spain and Wolff have both said that in 2011, they expect to break
even—that is, to get to the point where advertising revenue is high enough to
cover operating costs, though not to start paying back the initial investors.
Nymag.com does original reporting, as in the case of the Ailes/Palin story, but
since 2007 it has also had a strategy of growing through four blogs that use thirdparty
content combined with original reporting: Grub Street (on food), Daily
Intel (political and media news), Vulture (culture) and The Cut (fashion). ``These
niches need editorial authority to be successful,'' said Michael Silberman, general
manager of nymag.com. In a given week, the site publishes only about 35 articles
from the print magazine but 450 to 500 blog posts and thousands of photos. As a
result, only 14 percent of the site's page views are of content from the magazine.
``Every time we increase the frequency of the blog posts, we can drive up the
numbers of audience,'' Silberman says. And since 2007, nymag.com's audience
has grown from 3 million unique users to 9 million.
%The Story So Far: What We Know About the Business of Digital Journalism

The site has been particularly adept at going beyond its local roots. About 30
percent of the print magazine's audience comes from outside New York, but 70
percent of the website's readers live beyond the home market, which helps the
site attract attracting national advertising. Its restaurant section, Grub Street, expanded
in 2009 and is now in six cities.
The evolution of nymag.com's cultural news site, Vulture, from a small feature
to a destination with 2.5 million unique users demonstrates how powerful aggregation
strategies can be. When it started in February 2010, Vulture was getting
700 to 800 unique visitors daily. After its official launch in September 2010,
Silberman found that it ``filled an editorial hole in the marketplace.'' One of its
most popular features is ``clickables''—a stream of 20 short posts per day, featuring,
among other things, viral videos and music albums leaked ahead of official
release. Vulture now has 10 full-time editorial employees and get lots of support
from New York Magazine back-office departments in finance, human resources
and technology. The magazine has decided to spin off Vulture as its own site with
a separate Web address sometime in 2011—a move that Silberman believes will
help generate sales of entertainment and other national ads to companies that feel
that a close tie to New York City can be an impediment.
* * *
There are few secrets on the Internet, and even fewer barriers to entry. Each
innovation that works instantly attracts imitators and improvers. (LinkedIn, the
professional networking site, launched LinkedIn Today in March 2011 to curate
content not just by topic but also by what people in a user's network or industry
are reading.^{\href{#endnotes-chapter-6}{17)}} Because aggregation is so much cheaper than original content,
it has an automatic economic advantage, but the attractiveness of aggregation
brings more and more competitors into the field. So merely being an aggregator
is hardly a guarantee of economic security.
A few publishers have successfully sued sites that steal their content outright.
That has led others to toy with the idea of getting news sites to unite and deny
aggregators access to their content. Even if that kind of cooperation were legal—
and it might not be—it would be impossible to sustain or enforce. There are just
%Columbia Journalism School | Tow Center for Digital Journalism
%Aggregation: 'Shameless' – and Essential 91
too many sites producing original content. The economic benefits of aggregation
and being aggregated are significant, even if they differ widely from one site to
the next.

\chapter{Dollars and Dimes: The New Cost of Doing Business}
Journalism is expensive and good journalism especially so, but the newsroom
usually is not the costliest part of running a news organization. The Commerce
Department has estimated that printing and delivery account for up to 40 percent
of a newspaper's costs;^{\href{#endnotes-chapter-7}{1}}a 2007 study found that the newsroom accounts for
only about 15 percent of a newspaper's expenses.^{\href{#endnotes-chapter-7}{2}}(One publisher reports that
``infrastructure'' costs—everything other than editorial and marketing—typically
make up two-thirds of a newspaper's expenses.^{\href{#endnotes-chapter-7}{3}}) Still, editorial production has
been a major expenditure, with newsroom costs running into the tens of millions
of dollars even at mid-size news organizations, and far higher at major national
news outlets. In 2008 the New York Times reported that its newsroom budget
was more than \$200 million per year.^{\href{#endnotes-chapter-7}{4}}
Such robust newsroom spending has been made possible, of course, by the
information and advertising dominance that news media traditionally enjoyed.
Outlets with 20 percent profit margins had the luxury of not having to think
about the cost of each story produced. For decades, big-city newsrooms provided
a wealth of resources to support top-tier reporting: newswires, clipping services,
transcription services, research and library staff, and so on. A major investigative
series might take months to prepare and cost tens of thousands of dollars, or even
much more. (A New York Times editor estimated that the paper's collaboration
with ProPublica on euthanasia in New Orleans hospitals after Hurricane Katrina
cost \$400,000; he later clarified that's what it would have cost if the Times had
undertaken it alone.^{\href{#endnotes-chapter-7}{5}}) Even in a bare-bones newsroom, serious accountability
journalism is not cheap, as the new generation of foundation-supported, onlineonly
newsrooms can attest.
Consider CT Mirror, a nonprofit, online-only newsroom founded in 2010 to
cover politics and government in Connecticut. It is hard to imagine a more sober
outlet: CT Mirror focuses squarely on news about such topics as education, the
economy, human services and the budget, with almost no human interest stories
or even crime reporting. It is also hard to imagine a lower-cost outlet for serious
%The Story So Far: What We Know About the Business of Digital Journalism

reporting. CT Mirror is based in a modest Hartford office, and most of the sevenperson
editorial staff works from home or on the road. The site has no printing
or delivery costs, and it has minimal sales or marketing expenses, and a simple,
straightforward website; fully 75 percent of its foundation-provided budget goes
to editorial salaries. ``There's no room for fat,'' says founder and editor Jim Cutie.
Because CT Mirror is foundation-backed, it makes no secret of its budget.
In its first 15 months of operation, the newsroom ran on \$1.1 million in contributions
and produced about 2,400 news stories, for an average cost per story
of around \$450. Similarly, the Gotham Gazette, a foundation-funded, nonprofit
politics-and-policy news site based in New York City, has four employees and a
\$350,000 annual budget. It produces between four and eight original stories per
week, for a per-story cost of more than \$1,000.
Of course, not every item in a general-interest newspaper requires the same
level of investment. One print daily, in a mid-size market with a relatively low
cost of living, recently did an analysis to determine how much various articles
cost. The newspaper, which shared its numbers on the condition it not be identified,
found that the salary cost of reporting and writing stories ran from \$190
to \$430. The most expensive stories came from the opinion section, and the
cheapest came from features. (The average cost per staff-written story was \$227;
articles by stringers cost \$85.) Those figures don't include editing, production or
distribution costs, which could easily triple the cost.
Meanwhile grass-roots local news sites such as Baristanet, in suburban New
Jersey, and The Batavian, in upstate New York, operate on shoestring budgets.
They keep costs at just thousands of dollars per month with tiny reporting staffs
and almost no infrastructure. (See Chapter 3.) To replicate their cost structure
would be difficult for a hard-news site like CT Mirror and all but inconceivable
for a traditional, bricks-and-mortar newsroom working online as well as in
broadcast or print.
This points to a central paradox of the online news economy: In an environment
of sharply constrained ad revenue, the media's traditional economies of
scale break down. What look like powerful editorial and business assets for online
%Columbia Journalism School | Tow Center for Digital Journalism
%Dollars and Dimes: The New Cost of Doing Business 95
journalism—like established brands and well-staffed newsrooms—are turning
out to be liabilities, because they are accompanied by a severe reduction in pricing
power for circulation and advertising. The pressure on costs is intense.
One local TV station, interviewed for this report on the condition it not be
identified, illustrates this paradox well. The station is a successful local broadcaster
operating in one of the top four U.S. television markets. It has a 150-person news
staff and is a leading source of local news in its market. For several years it has also
operated a website, run by three dedicated producers who do original reporting
as well as post stories and video drawn from newscasts.
On paper, the site has the assets to be a top online outlet for news about its city.
It can draw on its sizable reporting resources and the promotional power of the
station's broadcast operation. And it has free access to a large supply of valuable
``rich media'' assets in the video and audio segments produced by its parent.
The station has built a large online audience over the last several years, growing
from a monthly average of 550,000 unique visitors in 2008 to about 2.5 million
at the end of 2010. Still, the station's general manager has struggled to make the
site break even. He shaved expenses substantially by outsourcing software and
site maintenance and by cutting back on reporting from the field; not counting
salaries, the site now costs roughly \$500,000 per year to run. He has also tried a
number of different sales strategies, including revenue-sharing partnerships and
small, dedicated sites—``microsites''—custom-built for particular advertisers. Still,
the site accounts for just 1 percent of the station's overall revenue. ``Forget local
being the holy grail,'' he says. ``National sites are making money, but we don't
have the scale locally to do so.''
The cost pressure can be even more severe for local newspapers following their
audiences to the Internet. A newspaper has enviable assets for putting news on
the Internet—because it produces so much news text every day—and in theory
it can also achieve enormous savings as it makes the switch to digital distribution,
which does not require ink, paper and delivery trucks.
John Paton, chief executive officer of the Journal-Register Company, based in
Yardley, Pa., has made reducing legacy costs the centerpiece of what he calls a
``Digital First'' strategy. The company, which came out of bankruptcy in 2009,
%The Story So Far: What We Know About the Business of Digital Journalism

owns 18 local dailies and scores of other ``multi-platform products'' across the
Northeast and upper Midwest. Since becoming CEO in 2010, Paton has reduced
expenses by consolidating printing facilities and outsourcing a wide range of
noneditorial functions, from delivery to advertising design. He promises to have
reduced infrastructure costs by 50 percent in three years.
Another prong of ``Digital First'' is to wean the publisher from its dependence
on print advertising revenue, which Paton calls the ``crack cocaine'' of the business.
He predicts that by the end of 2011, more than 15 percent of ad revenue
will come from the digital side and that most of that will be purely online revenue—
not ads sold in print-online package deals.
Paton has been a popular figure on the future-of-journalism circuit, appearing
frequently at industry conferences to extol his company's digital gains. He
frequently cites growth percentages for the Journal-Register site—for instance,
he says the company's ``digital audience'' grew 75 percent in 2010, reaching 8.8
million unique visitors in March 2011. He also recently announced bonuses for
employees and declared the company had made a \$41 million annual profit.^{\href{#endnotes-chapter-7}{6}} It
isn't clear, however, how that profit figure is calculated, because the company
does not provide data on revenue, costs or other metrics as a publicly traded firm
would. Paton says his investors don't want to disclose too much.
But Paton does acknowledge that moving revenue online amounts to ``trading
dollars for dimes''—or perhaps, if he's successful, quarters. The gamble is
that the Journal-Register Co. will be able to cut costs and increase its online
audience quickly enough to compensate for the lower revenue that online
advertising brings.
The notion of ``trading dollars for dimes'' captures the impact of digital distribution
on the economics of the business. Newspapers, magazines and broadcasting
are all characterized by high fixed and low variable costs; it's quite expensive
to produce the first copy of a newspaper, but it's far cheaper to produce the
second copy—or the millionth. A local broadcaster faces much the same set of
costs whether it reaches 100 viewers or 100,000. Hence the traditional media's
profound economies of scale. News outlets that could not build a large enough
base of readers or viewers to cover their steep fixed costs have tended to collapse
%Columbia Journalism School | Tow Center for Digital Journalism
%Dollars and Dimes: The New Cost of Doing Business 97
in a few years, mired in debt. But those that surpassed that break-even point and
went on to establish a mass audience could become immensely profitable—and
those steep fixed costs created a natural barrier to competition.
Online, the equation changes dramatically. Observers sometimes underestimate
the expense involved in running a high-quality, high-traffic online publication.
But the barriers to entry are radically lower than in print or in broadcast. While
a number of aspects of the online ad market have favored advertisers over publishers,
simple audience fragmentation goes a long way toward explaining why
news outlets have seen their revenue squeezed so tightly. Today, someone wanting
political news or movie reviews has dozens of alternatives to choose from or
stumble across.
Thus both revenue and costs are lower online. To be more precise, the cost
curve has been stretched out. The steep initial investment required to launch
a media business is gone, and that has opened up opportunities for low-cost
local or topical sites that aim to build an audience in the thousands or tens of
thousands. This is the niche occupied by many moderately successful blogs as
well as community sites like Baristanet, with modest ad income and even more
modest expenses.
At the other extreme one finds large-scale media properties that have substantial
technology or editorial costs but that amass enough sheer traffic to turn a
profit. The dominant example here is Google, whose 175 million monthly users
in the U.S.—generating billions of page views per month—allow it to capture
more than 40 percent of the entire U.S. online advertising market.^{\href{#endnotes-chapter-7}{7}} Even considering
only ``display advertising'' (that is, excluding search ads), Google accounts
for 13 percent of ad spending. Yahoo and Facebook, the display ad leaders, each
claim an additional fifth of the market.
However, most legacy news producers operate in the large and difficult middle
of the cost curve, with traffic too low to compensate for the fixed expenses
of news production, despite the savings that come from publishing online. In
2010, total operating expenses at the New York Times Co. ran to \$2.1 billion,
about two-thirds of the \$2.9 billion total for Yahoo Inc.^{\href{#endnotes-chapter-7}{8}} Of course, somewhere
between a third and a half of the newspaper's expenses would disappear if it no
longer printed a paper edition. But Yahoo has many times more monthly visitors
(roughly five times as many, if one counts traffic only to nytimes.com.). And
while monthly visitors to all Times properties generate fewer than 2 billion page
views, Yahoo serves out a staggering 100 billion pages each month.^{\href{#endnotes-chapter-7}{9}}
Justin Smith, president of Atlantic Media Co., argues that these dynamics explain
both the opportunities the Internet affords and the stark challenge it has
posed to established news providers. ``There is a whole wave of new journalism
models that have been developed at a fraction of the cost of traditional media,''
he says. ``Traditional media players are way too set in their ways for reducing cost.
They can't sustain the revenue to support their costs.''
That is not to say, however, that any online venture that falls between a small
community blog and Google is doomed to fail. In addition to running the Atlantic,
Smith is a founder of Breaking Media, a collection of sites aimed at specific—
and affluent—professional communities. Its properties cover law (Above
the Law), Wall Street (Dealbreaker), fashion (Fashionista), green transportation
(AltTransport) and accounting (Going Concern). Above the Law is the most
%Columbia Journalism School | Tow Center for Digital Journalism
%Dollars and Dimes: The New Cost of Doing Business 99
successful of these sites, with a monthly audience of more than 700,000 unique
visitors. Smith won't say when the company might turn a profit, but his formula
depends on pulling together sizable audiences at minimal cost—each site has just
two journalists, with ad sales and administration centralized.
Henry Blodget's Business Insider site is pursuing a similar strategy, on a much
larger scale. Blodget has disclosed financial details about his media company, reproduced
below, in what was an unusual move for a private firm.^{\href{#endnotes-chapter-7}{10}} ``We're a private
company, and we've never disclosed any of that stuff, either. But I'm honestly
not sure why,'' he explained on March 7, 2011. ``So we're going to try an experiment.
We're going to disclose that stuff. Then we're going to see if something
horrible happens to us.''
The statistics tell an interesting story. With 45 full-time employees, including
25 in the editorial department, Business Insider is hardly a grass-roots effort. In
2010 it cost almost \$5 million to run. But unlike many sites of similar size, Business
Insider managed to turn a tiny profit in 2010—about \$2,127, or as Blodget
put it, about enough to buy a MacBook Pro.
One factor accounting for Business Insider's survival is that the site targets
investors and financial professionals rather than a general-interest audience. But
that also means it must fight for readers and advertisers with the rest of the financial
press, including giants like Bloomberg and Reuters and lower-cost sites like
paidcontent.org and Breaking Media's.

%The Story So Far: What We Know About the Business of Digital Journalism

The biggest difference between Business Insider and a similar site that doesn't
break even is traffic: Blodget's venture managed to pull in 6 million unique visitors
a month by the end of 2010, about double its audience of a year earlier. By
March 2011, Business Insider had almost 8 million visitors, which represents a 30
percent jump in just three months. If the site becomes truly profitable, it will be
by virtue of having continued that growth—getting to an audience of 15 million
to 20 million visitors each month while keeping expenses flat.
The example of Business Insider suggests a provocative comparison with the
old-media world. In the newspaper industry, a rule of thumb has been that every
1,000 additional readers justifies an additional newsroom employee. Going by
Blodget's numbers, the comparable figure in online news media is closer to one
person on the editorial staff per 150,000 readers.
As far as costs are concerned, then, the real advantage of digital-only operations,
from The Batavian to Business Insider, is that they don't have to ``trade
dollars for dimes''—they are natives of the dime economy. By contrast, legacy
news outlets must navigate a tricky cost transition when they go digital, cutting
expenses and boosting online revenue while minimizing the damage to the traditional
advertising that still sustains them.
This process begins by learning how to get the most out of their newsrooms
in each medium. At the Atlantic, Justin Smith says, cost efficiencies depend on
employees working across the digital/print divide. ``There are very few employees
who don't do both print and digital work,'' he says. ``Maybe a couple of factcheckers
and one senior editor who concentrate on long pieces. We have about
60 people in editorial, and 99 percent of them are completely integrated.'' In
advertising sales, all salespeople sell print and digital.
The Atlantic got a lot of attention in 2010 for having become profitable (``a
tidy profit of \$1.8 million''^{\href{#endnotes-chapter-7}{11}}) for the first time in decades. But the numbers disclosed
were for the company as a whole, including print, digital and events. Still,
Smith insists that the company's move to profitability depended on containing
costs on the print side. ``We were brutal about shifting resources away from print,''
he says. The company made layoffs in both the editorial and ad sales departments.
%Columbia Journalism School | Tow Center for Digital Journalism
%Dollars and Dimes: The New Cost of Doing Business 101
Does this mean the Atlantic makes money online? The company reported that
digital advertising revenue rose 70 percent, and print advertising revenue 27 percent.
^{\href{#endnotes-chapter-7}{12}} ``The digital version of the Atlantic is definitely profitable,'' Smith says.
``And it is a source of growing profit.'' Smith says this is the case even as the
company continues to increase staff and resources on the digital side. The New
York Times' report on the Atlantic had the magazine's overall revenue doubling,
to \$32.2 million in 2010, with advertising revenue accounting for about half of
that. Digital ads supplied 40 percent of ad revenue.
For a large metro newspaper, the calculus of cost-cutting is tougher. Even in
the face of declining readership and ad revenue, executives often fear that major
cuts will only accelerate their slide in circulation.
Detroit is an exception. Newspaper executives there met early in 2008 to
consider their options, none of which was very attractive. The Detroit area was
on its way to a dead-last rating in a survey of 363 cities' job growth for the first
decade of the 21st Century.^{\href{#endnotes-chapter-7}{13}} While most U.S. cities had one newspaper, Detroit
had two—the Free Press and the News—bound in a joint operating agreement
and dividing diminished circulation and ad revenue. (The agreement means that
the two newsrooms compete, but their partnership handles both papers' ads, circulation
and printing. The Free Press is owned by Gannett and the News by MediaNews
Group; as part of a revised agreement, Gannett must pay MediaNews
around \$45 million over a 20-year period, according to Crain's.^{\href{#endnotes-chapter-7}{14}})
Then-publisher Dave Hunke assembled his team, and they kicked around ideas
on how to cut costs. The ideas ranged from publishing a pocket-sized newspaper,
to arming 200 citizen journalists with cameras, to a ``dinner in a bag'' promotion
that would give special consideration to readers when they pick up meals at a
local grocery store. The newspaper executives also considered deep reductions in
staff or space devoted to news.
In the end, they came up with a radical idea: eliminating home delivery of
both papers on Monday, Tuesday, Wednesday and Saturday. The reasoning was
that those four days were responsible for only 23 percent of the papers' print ad
revenue. The cutbacks went into effect in March 2009.
%The Story So Far: What We Know About the Business of Digital Journalism

The four non-home-delivery days are now responsible for only 7 percent of
the papers' print ad revenue. Those days' papers are smaller and are still sold as
single copies in the city; several thousand are picked up by independent contractors
who deliver them to houses, generally in Detroit's wealthier suburbs. The
papers also have a same-day edition available by U.S. mail that reaches about
4,000 subscribers.
In October 2009, the dailies doubled their newsstand weekday price from 50
cents to \$1. The cost of home delivery for Thursday, Friday and Sunday is now
\$13 a month, slightly less than what subscribers used to pay for seven days a week.
As part of the change, the companies launched an electronic edition for subscribers—
basically a replica of that day's papers, available online. It loads slowly,
though that has been improved since the early going. Access to the papers' websites
remains free.
As a cost-cutting measure, executives say the move to three-days-a-week home
delivery met their goals and helped stabilize their journalistic efforts. The company
says the delivery change enabled it to trim its overall costs by 15 percent.
``It ensured our survival,'' says Paul Anger, editor and publisher of the Free Press.
Joyce Jenereaux, executive vice president of the Detroit Media Partnership, adds,
``If we hadn't done it, we'd be putting out horrible products.'' But the bleeding
hasn't entirely stopped. In November 2010, unions representing 900 employees
got to look at internal financial data for the papers; after doing so, they agreed
to pay cuts, a two-year wage freeze and increased health insurance payments.^{\href{#endnotes-chapter-7}{15}}
One thing that isn't clear is whether the Detroit strategy was successful in
getting readers to move from print to digital platforms. As executives expected,
circulation of the Detroit papers declined. In the months before the delivery
change, the papers' combined weekday circulation was 436,238; in early 2011,
that circulation was 230,876. Neither of those figures includes the e-edition,
which has weekly traffic of more than 100,000. But the weekly e-edition number
doesn't represent that many individual readers; someone who logs on five
days in a row would be counted five times. Daily figures show that about 20,000
people visit the e-edition on each of the four days the paper isn't delivered, and
that a third that many use it on days when the paper is delivered.^{\href{#endnotes-chapter-7}{16}} Engagement is
%Columbia Journalism School | Tow Center for Digital Journalism
%Dollars and Dimes: The New Cost of Doing Business 103
substantial, as e-edition users spend about 18 minutes per visit. But the e-edition
doesn't reach nearly as many people as used to get home delivery on the days
that have been eliminated.
As for the papers' websites, they had significant growth in 2008—a year before
the change in home delivery. That was a big news year which included the collapse
of the auto industry, the historic Obama election and a scandal involving Mayor
Kwame Kilpatrick's affair with his chief of staff. (The Free Press won a Pulitzer
Prize for its coverage of the mayor's woes.) The papers grew 31 percent in unique
users. After the cuts in home delivery, the number of users on the sites continued to
grow, but more slowly: by 6 percent in 2009 and 10 percent in 2010.
Similarly, the year of big increases in time spent per visit on the Web was 2008,
when it went from about 8 minutes to more than 13 minutes. Why? One reason,
executives say, is that 2008 was the year that reader commenting was enabled on
the site. ``There was not a discernible bump when we made the model [delivery]
change,'' says Patricia Kelly, senior vice president at Detroit Media Partnership.
Kelly also notes that digital ad revenue is up 65 percent since 2005. Print advertising,
meanwhile, is down 50 percent in the same period. As a result, digital is
expected to represent an estimated 19 percent of total ad revenue in 2011.
The changes in the delivery model have affected the culture of the newsrooms
to some extent. The papers' news staffs are designed to operate daily, just as before,
without paying much attention to whether they are publishing in print or
online on a particular day. Anger says, ``It's everyone's responsibility to be invested
in digital publication.'' But, says another editor, ``There's still a feeling that, if you
want people to see something, you're going to shoot for Thursday, Friday or
Sunday. … We haven't divorced ourselves from the idea that the big story should
run on a day we publish a paper.''
The deep cost-cutting measures by Detroit's dailies illustrate the challenge
legacy news providers face as they adapt to the economics of online media. The
strategy worked as a survival mechanism in a tough market; the damage from
cutting home delivery was checked by the fact that print revenue is falling so
quickly anyway. But it hasn't transformed a great number of print readers into
%The Story So Far: What We Know About the Business of Digital Journalism

new digital users. The sites' main growth occurred before the delivery change,
and appears to have been fueled by big, well-covered stories, and improved functionality
that got readers more involved.

\chapter{New Users, New Revenue: Alternative Ways to Make Money}
``The basic point about the Web is that it is not an advertising medium, the
Web is not a selling medium, it is a buying medium. It is user-controlled.''
—Jakob Nielsen, Web usability expert, 1998 ^{\href{#endnotes-chapter-8}{1}}
The journalism business these days often seems like a strange new world.
As Jack Sweeney, who has been the publisher of the Houston Chronicle since
2000, puts it, ``I thought I knew this business, and I did. But this business-model
blowup is totally different.'' Sweeney has had a long career with newspaper advertising
departments in Washington, D.C., Trenton, N.J., and Boston,^{\href{#endnotes-chapter-8}{2}} and he
could have been talking about a number of the efforts news organizations are
making to grow—like a Utah TV station's classified-ad service that has turned
into a community resource, or a national media company selling ads that never
appear on its own sites, or his own initiative as a service provider to small businesses.
The tactics differ, but they share a common strategy: News companies
are developing new businesses, not just propping up the old ones. And in doing
so, they are challenging some of the orthodoxies that had slowed their transition
to the digital world.
For most of these companies, the revenue from such new initiatives is modest;
it doesn't begin to replace the dollars lost in the traditional business. But there
are encouraging signs. The process of finding new readers and dollars is forcing
media companies to redefine who they are and what business they are really in.
* * *
Sweeney makes his point with a spreadsheet that shows how much some of
Houston's biggest retailers spent in 2010 on ads in the Chronicle. The numbers
aren't what they used to be, and he knows they're not coming back. Even
though the Chronicle remains the country's 10th-largest newspaper, its circulaThe
Story So Far: What We Know About the Business of Digital Journalism
106
tion dropped by more than 10 percent in 2010 over the year before, to 343,952.
A decade ago, it was nearly 550,000.^{\href{#endnotes-chapter-8}{3}} Moreover, online display advertising, which
was just under \$28 million in 2010, won't make up the difference on its own, as
it represents 12 percent of total ad revenue.
So the Chronicle, with guidance from its owner, Hearst Corporation, is looking
to a sector often ignored by big media—the small fry. ``We didn't used to go
after mom-and-pop businesses,'' Sweeney says. ``Houston has 310,000 businesses
with 10 employees or less. The potential is huge. As department stores have consolidated,
we needed something new.''
What makes the Chronicle's approach interesting is that it isn't based on selling
ads that appear on the pages of the site or in the newspaper. Instead, the
Chronicle is launching a consulting business—selling a host of Internet services,
from website design to improving businesses' rankings on search engines. And
when the Chronicle does sell ads as part of this outreach, those are just as likely
to appear on Yahoo or Facebook as on chron.com.
To get started, Sweeney hired about 30 employees, some of whom who knew
the world of small businesses from having worked at Yellow Pages. The Chronicle
also retrained some of its own staff. The sales pitch it makes to businesses is this:
The Chronicle evaluates their websites, improves their rankings on search-results
pages and helps them write press releases that are posted on the chron.com site
to give traffic a boost.
The Chronicle charges \$500 and up a month for the service, asking its clients
to sign one-year contracts. As of April 2011, the fourth month of the program, it
had enrolled nearly 500 businesses and booked more than \$2.5 million in contracts.
Sweeney's goal is to reach around \$7 million in annual revenue.
Others are also in this business. One firm, ReachLocal Inc., signed up nearly
17,000 advertisers, booking nearly \$300 million in revenue in 2010.^{\href{#endnotes-chapter-8}{4}} McClatchy
has partnered with Webvisible, a California-based Web services firm that says it
has more than 10,000 clients.
%Columbia Journalism School | Tow Center for Digital Journalism
%New Users, New Revenue: Alternative Ways to Make Money 107
But even if the effort is as successful as the Chronicle hopes, Sweeney figures it
would do no more than match the current revenue from one of the paper's biggest
advertisers. In other words, it will be a big help but is not, in itself, a replacement
for the old business model. ``This has become a nickel-and-dime business,''
he says. ``And you need a lot of nickels and dimes.''
* * *
A decade ago, KSL, a local TV station in Salt Lake City, came up with what
was then a novel idea: It would start its own classified-ads section on KSL.com
and end its relationship with a company that was already providing that service
for the site.
``We were making like \$300,000 a year [in revenue] on the partnership, which
back then was a lot of money online,'' says Clark Gilbert, president and chief
executive officer of Deseret Digital Media. But the station, an NBC affiliate, saw
the change as a way to build traffic to the site. Its classifieds service would also be
a way to showcase the moral standards of its owner, the Church of Jesus Christ
of Latter-day Saints. Deseret Digital Media runs the online properties for the
church's TV station and its newspaper, the Deseret News, along with sites more
obviously of the church, MormonTimes.com and DeseretBook.com.
Today, KSL.com is a powerhouse on the Web. The site has more than 4 million
unique users and generates an astounding 250 million page views a month, says
Gilbert. (KSL's sister property, the Deseret News, has a more typical audience of
about 2.5 million unique users and 30 million page views; the website of a competing
newspaper, the Salt Lake Tribune, has roughly the same size audience.) In
a recent study of Web traffic data in major markets, a company called Internet
Broadcasting found that KSL.com reaches 48.8 percent of its local market. That
is more than any local media outlet in the survey but one, the Minneapolis Star
Tribune's site. And it is far beyond the Web footprint of the top local TV stations,
which average under 20 percent market share.^{\href{#endnotes-chapter-8}{5}}
%The Story So Far: What We Know About the Business of Digital Journalism

Still, all that traffic didn't keep Deseret Digital Media from announcing layoffs
last year at the Deseret News^{\href{#endnotes-chapter-8}{6}} that, despite the sunny headline that announced
the news (``Deseret News set to lead, innovate''), resulted in a cutback
of 43 percent of the newspaper's workforce and consolidation of some newsgathering
operations with KSL-TV.^{\href{#endnotes-chapter-8}{7}} (The Salt Lake Tribune announced the
layoffs with some competitive schadenfreude: ``Tribune to press ahead in face
of News changes.''^{\href{#endnotes-chapter-8}{8}})
KSL.com's strategy relies partly on its worldwide audience of church members,
but it also offers useful lessons for news organizations seeking untraditional ways
to build a digital audience.
The classifieds themselves are mostly free, though advertisers can pay up to \$10
a day to get prominent placement. The classifieds pages also host other ads, and
more importantly, they are responsible for about 70 percent of KSL.com's total
traffic, so they provide tremendous benefits to the rest of the site. The pages carry
prominent links to news stories and videos on KSL.com, which helps to generate
70 million to 80 million page views a month for content that isn't classified
ads. ``The main route to the site is still the news page,'' Gilbert said. ``We haven't
tried to make 'KSL.com/classifieds' our bookmark. That made the [KSL] news
site bigger than any other news site in the market.''
Gilbert adds that there is another benefit: ``Here's something hard for old-media
people to accept. … Our news content gave a level of trust to the classifieds,
and classifieds drove relevance back to the news.'' Or, put another way, the fact
that readers have come to rely on the classifieds under the KSL brand helped to
build relevance and credibility in the news as well.
KSL.com had some important advantages. First, it started early, shortly before
Craigslist came to Salt Lake City. And because it was a TV station's website, it
wasn't perceived as competitive by its existing staff; there was no classified-ads
manager to complain about giving away a lucrative revenue stream. ``KSL didn't
have legacy products that were competing with this service,'' says Chris Lee,
general manager of DeseretNews.com. ``If they wanted to do cars, there wasn't
someone saying, 'But we're already doing cars!'''
%Columbia Journalism School | Tow Center for Digital Journalism
%New Users, New Revenue: Alternative Ways to Make Money 109
The site also demonstrated a keen sense of its audience—which shouldn't be
a surprise, given the church ownership. Managers tried to be especially vigilant
about keeping the site clean (``no way were we going to allow prostitute or massage
ads,'' Gilbert says) and detecting fraud.
KSL.com also committed to ``letting our users develop the product with us,''
Gilbert said. For instance, in the spring of 2011, KSL.com asked its readers what
kinds of firearms they thought the site should allow to be sold. It also asked them,
``How often do you believe people are using the KSL Classifieds Firearms and
Hunting section to circumvent firearm laws?''^{\href{#endnotes-chapter-8}{9}} Users help police the site for bad
actors. Anonymity isn't allowed: ``Sellers had to have an identity,'' Gilbert said.
The classifieds give KSL.com an unusually high level of engagement. According
to Mike Petroff, vice president of new media sales, the site gets around 10
million page views from 250,000 users on an average weekday, for a stunning
daily rate of 40 page views per visitor. (The ads don't appear on the newspaper's
site; the Deseret News shares business, but not newsgathering, operations with
the Salt Lake Tribune, owned by MediaNews Group.)
Most ads expire after 30 days. Even with such a short lifespan, there were more
than 206,000 listings on a typical day in March 2011, in categories ranging from
goats to muzzleloaders, from paintball equipment to bands seeking members.
Gilbert came to Salt Lake City in late 2009 after a career that included a professorship
at Harvard Business School, where he worked closely with Clayton
Christensen, author of ``The Innovator's Dilemma,'' a well-known book about
disruptive change.^{\href{#endnotes-chapter-8}{10}} The two of them collaborated on the ``Newspaper Next''
project, a 2006 study sponsored by the American Press Institute to encourage
innovation.^{\href{#endnotes-chapter-8}{11}} Gilbert was hired by Deseret Management Corp.'s president and
chief executive officer, Mark Willes, who had a troubled reign as CEO of Times
Mirror (1995-2000) and publisher of the Los Angeles Times (1997-1999).^{\href{#endnotes-chapter-8}{12}}
One of Gilbert's main goals was reflective of a tenet of Christensen's philosophy:
``Business units don't evolve; corporations do.''^{\href{#endnotes-chapter-8}{13}} So Gilbert separated
the digital sales force to enable it to take more risks. He said that KSL.com had
been ``run through the mainline channel—the TV. The [ad] sellers would have
%The Story So Far: What We Know About the Business of Digital Journalism

an afterthought to also sell Web. They'd throw it in if you also bought TV.'' The
new company ``created a profit-sharing relationship with the legacy organization.
They'd benefit from our growth—but they didn't control it.''
KSL.com's revenue grew 75 percent from 2009 to 2010, executives say, though
they don't spell out numbers.^{\href{#endnotes-chapter-8}{14}} Gilbert says his company will continue to push
on both the cost and the revenue sides of the equation: ``News is expensive,'' he
says, and audience loyalty is key. ``You can't get two clicks and expect to pay off
on that investment.''
* * *
For decades, there has been a connection between the journalism that news
organizations provide and the advertisements that generate most of their revenue.
Whether it's a glossy spread that runs before the table of contents in a fashion
magazine, or the anchorman's ``more after this message'' assurance on the local
Eyewitness News, ads and content have always been closely linked in the stream
that appears before the consumer.
That linkage is breaking down, and news organizations are scrambling to replace
it with something else. That may mean selling ads on sites they don't own
or control. ``Creating content doesn't ensure a well-sized audience,'' says Chris
Hendricks, vice president of interactive media at newspaper chain McClatchy
Co. ``We're accepting of the fact that the two may be disengaged.'' He then adds
something one wouldn't have heard a few years ago from a media executive:
``The longstanding premise of content and advertising being inextricably linked
has clearly fallen apart.''
McClatchy and other companies are turning toward selling advertising space on
other sites, including Facebook and Yahoo. ``It's almost like we are a sales and distribution
company that decided we're going to fund journalism,'' says Hendricks.
Salespeople at McClatchy's 30 daily newspapers, as well as those at many other
news organizations, sell ads on Yahoo as part of their pitch to local advertisers.
For a worldwide company like Yahoo, ``it's very difficult and expensive to set
up a local sales force of size,'' Hendricks says. In the 1990s, Microsoft tried and
%Columbia Journalism School | Tow Center for Digital Journalism
%New Users, New Revenue: Alternative Ways to Make Money 111
failed to crack the market with a venture called Sidewalk, which was designed
to produce city guides and sell local ads. Hendricks notes that Yahoo's rates for
local ads tend to be higher than for national ads—but Yahoo needed people who
knew the communities and businesses. ``So we became their local sales force selling
their inventory.''
Because Yahoo has such broad reach, the relationship opens a big market for local
news organizations. ``The typical paper has 15 percent penetration in the local
market,'' Hendricks says, speaking of online operations. ``When we partner with
Yahoo, it takes us up to 80 percent.'' And because many Yahoo ads are ``behaviorally
targeted''—meaning they are more closely geared to readers' interests, based
on Web usage habits, geography or demographics—the rates are much higher.
But those ads need a lot of viewers to ensure that the subsections of the audience
are big enough to interest advertisers. ``It's almost impossible to sell behaviorally
targeted ads with 15 percent penetration,'' Hendricks says. ``With Yahoo's scale
you can.'' McClatchy averages an \$18 cost per thousand views for targeted ads,
Hendricks says. That's about twice the average for its usual display ads, though it
has to share the proceeds with Yahoo.
There are longstanding examples of moves to sell inventory beyond a company's
sites, including careerbuilder.com, an employment-classified site that Mc-
Clatchy jointly operates with Gannett and Tribune companies.^{\href{#endnotes-chapter-8}{15}} But there can
be difficulties. It isn't easy to persuade traditional ad departments to sell inventory
that is not their own. ``The gravitational pull of print is very strong. As soon as
you get away from distribution and content adjacency, the harder it gets,'' Hendricks
says. And ad sales on other sites represent only a small revenue stream
so far. Hendricks says McClatchy sold about \$15 million of Yahoo ads in 2010
and expects to increase that to as much as \$19 million in 2011. To put that into
perspective, as dismal as 2010 was for McClatchy, the company still sold a billion
dollars of advertising that year.^{\href{#endnotes-chapter-8}{16}}
* * *
%The Story So Far: What We Know About the Business of Digital Journalism

One of the issues in selling others' ad space is that a publisher must adjust to a
variety of pricing schemes. For example, the Houston Chronicle tells advertisers
that it can help them reach, say, local men between the ages of 25 to 64 on
Facebook. But Facebook users would need to see an ad 2,500 times before the
advertiser could be assured it would generate a single click, according to the
Chronicle. That's a key reason that the price of ads on Facebook is low: \$1,500
for 1.875 million impressions on Houston's rate card, or a CPM of just 80 cents,
less than a tenth of what most news sites get. Others have calculated Facebook's
effective CPMs as even lower, below 20 cents.^{\href{#endnotes-chapter-8}{17}} By comparison, for targeted ads
on Yahoo Sports or Finance, the Chronicle expects to charge up to \$4,400 per
200,000 impressions, for a CPM of \$22.
Actual ad costs often vary from what appears on a rate card as a result of bargaining
between buyer and seller. Nevertheless, it's noteworthy that the Chronicle
charges nearly 28 times as much for ads on Yahoo as on Facebook. The price difference
is a result of several factors, including the more prominent display space
on Yahoo and the problems that social-media sites like Facebook have getting
users to see or click on ads. In November 2010, the Wall Street Journal reported
that 24 percent of all online display ads in the U.S. now appear on Facebook, but
that they are responsible for less than 10 percent of total display-ad revenue.^{\href{#endnotes-chapter-8}{18}}
Why do Facebook ads get such low rates? And what does that mean for the rest
of the market? It could be that the standard ways of valuing advertising—that is,
by whether it will impel a consumer to buy a product, visit a store or feel better
about a brand—simply don't work very well in a world where people using
social media aren't looking to be sold something.
In a prescient 2008 AdAge column, Matthew Creamer summed up an issue
that runs throughout this discussion: ``The Internet is too often viewed as inventory,
as a place where brands pay for the privilege of being adjacent to content.
... The presumed power of that adjacency has provided the groundwork for the
media industry for decades.''^{\href{#endnotes-chapter-8}{19}} Companies today have faster and cheaper access
to consumers. ``The marketer, once at the mercy of a locked-up media landscape,
can now be a player in it,'' he adds.
%Columbia Journalism School | Tow Center for Digital Journalism
%New Users, New Revenue: Alternative Ways to Make Money 113
Inevitably, as Creamer notes, the discussion becomes one of how marketing is
shifting to ``earned'' media rather than paid. One analyst defines the distinction
this way: ``'Earned media' is an old PR term that essentially meant getting your
brand into free media rather than having to pay for it through advertising,'' writes
Sean Corcoran of Forrester Research. ``The term has evolved into the … wordof-
mouth that is being created through social media.''^{\href{#endnotes-chapter-8}{20}}
If marketers believe they can reduce their advertising costs by engaging consumers
directly, that almost certainly cuts revenue for news organizations. Although
some firms are trying to capitalize on the trend by assisting advertisers
with their social-media strategies, that is a labor-intensive business that is outside
the expertise of many media companies.
And there are journalistic problems that go beyond the economic loss represented
by the decline of old-fashioned advertising relationships. A Florida company,
Izea, explicitly sets up arrangements so people who blog or tweet favorably
about a company can get compensated in cash, travel or in other ways. The
company insists that its writers adhere to Federal Trade Commission guidelines,
enacted in 2009, requiring disclosure of `` 'material connections' (sometimes payments
or free products) between advertisers and endorsers.''^{\href{#endnotes-chapter-8}{21}}
But a 2010 study by Izea found that many people engaged in this ``social media
sponsorship'' weren't aware of the FTC guidelines or had been offered compensation
without a requirement to disclose it. The survey respondents also priced a
``sponsored tweet'' from a personal Twitter account at an average of \$124 and a
``sponsored blog post'' at \$179—around the same amount a small news organization
will pay for a story and far more than an average blog post would ever get
from display ads.^{\href{#endnotes-chapter-8}{22}}
* * *
None of these ventures comes close in potential payoff to the online coupon
craze, pioneered most successfully by Groupon. The company was launched in
Chicago in November 2008, offering its customers daily discount deals on serThe
Story So Far: What We Know About the Business of Digital Journalism
vices ranging from nail salons to restaurant meals. For the deals to kick in, a
minimum number of users must sign up; the system encourages users to spread
the word and to take advantage of social media.
In a little over two years, the company expanded to more than 500 markets in
44 countries and turned down a \$6 billion takeover offer from Google. Forbes
called it ``the fastest-growing company in Web history.''^{\href{#endnotes-chapter-8}{23}}
The company's model is a repudiation of much of what has driven online revenue
for media companies. ``Banner ads seem such a relic of the 19th century,''
Groupon founder Andrew Mason told Wired.com. ``If God created man and the
Internet on the same day, we would see more stuff like Groupon.''^{\href{#endnotes-chapter-8}{24}}
The company has drawn complaints, particularly from retailers like some Chicago
restaurant owners who said many Groupon customers either came only for
the discount and didn't return, or gamed the system by copying coupons and
using them repeatedly.^{\href{#endnotes-chapter-8}{25}} Groupon has also spawned a host of competitors.^{\href{#endnotes-chapter-8}{26}} And
media companies have wavered between joining with Groupon or competing
with it.
The Minneapolis Star Tribune launched a coupon service, called STeals.
Cox Media has started DealSwarm. McClatchy is trying to have it both ways:
It announced a deal with Groupon in July 2010.^{\href{#endnotes-chapter-8}{27}} Less than a year later, Mc-
Clatchy said it would launch its own deal service, while continuing to work
with Groupon. As AdAge noted, ``McClatchy gets 15\% of revenue from Groupon
deals ... on its own, McClatchy could collect as much as 50\% from deals
it sells and distributes.''^{\href{#endnotes-chapter-8}{28}}
But it may be too late for news organizations to get substantial revenue from
this business. Groupon is reported to be considering an initial public offering
that could raise as much as \$15 billion.^{\href{#endnotes-chapter-8}{29}} With so much capital, Groupon could
compete on price and breadth in ways that would overwhelm ordinary competitors.
And Groupon has its own challenges. It is possible that so many competitors'
coupons will flood the market that consumers and businesses will begin to tune
them out, which would diminish the value of the idea.
%Columbia Journalism School | Tow Center for Digital Journalism
%New Users, New Revenue: Alternative Ways to Make Money 115
* * *
Many media companies are trying to raise revenue through more untraditional
means. Wired Magazine opened a physical ``Wired Pop Up Store'' in New York
City during the winter holidays, where it holds events like a ``Geek Dad Family
Party.''^{\href{#endnotes-chapter-8}{30}} The store sells gadgets and paraphernalia. New York Magazine sponsors a
wedding showcase event every year, selling tickets to the public, and sponsorships
to national fashion brands; it also caters to local disc jockeys, dress stores, bakeries
and other enterprises in the wedding business.
Such events may be good for branding, but tend not to bring in a great deal of
new revenue. The Atlantic is different. It is involved in running about 75 events
a year, the most ambitious of which is the Aspen Ideas Festival. ``Most magazines
do events for advertisers,'' says Justin Smith, president of the Atlantic Media
Group. ``We use the Atlantic brand and editorial prowess for attracting people.''
The business, called Atlantic LIVE, also runs events with such names as the Green
Intelligence Forum and the Food Summit. They usually include partnerships
with organizations tied to the topic. Coverage of the event may appear on the
Atlantic's site or in the pages of the magazine.
Atlantic LIVE is run separately from the magazine and website and has its own
sales group and editors who run the events. It has become a significant source of
income for the company. Of the \$32 million reported as revenue by the company
in recent publicity, as much as \$6 million comes from these events.
* * *
If the old formula of ``adjacency''—selling ads and commercials alongside content—
is fading, what will replace it? There are many possibilities, but few are
likely, on their own, to provide the stream of dollars that advertising and circulation
once did.
%The Story So Far: What We Know About the Business of Digital Journalism

It may be most useful to resist the temptation to think about digital journalism
economics in terms of moving an old business model to a new realm. The
common thread in the strategies described in this chapter is that they demonstrate
an embrace of the Internet, rather than an attempt to subjugate it to
legacy business models.
When viewed that way, the Internet isn't a friend or an enemy. It's reality.


%Columbia Journalism School | Tow Center for Digital Journalism

\chapter{Managing Digital: Audience, Data and Dollars}
Although all digital news organizations live in a brutally competitive environment,
some companies do much better than others because their managers
respond more deftly to opportunities.
Arianna Huffington is in that category, and the Huffington Post's growth in
audience^{\href{#endnotes-chapter-9}{1}} and influence^{\href{#endnotes-chapter-9}{2}} is an example of a sustained idea and management attention.
The venture capitalist Eric Hippeau was an early investor and was CEO
of the site for several years, until its sale to AOL in February 2011. He was struck
by the conviction of the founders—Huffington and Ken Lerer, a corporate communications
executive turned venture capitalist—that much of U.S. society had
lost trust in authority and in journalism. When Huffington Post launched in
2005, blogs were resonating with consumers. ``They didn't have to go through
gatekeepers—journalists,'' Hippeau said. ``Blogs could democratize news.'' Logically
flowing from this idea was a focus on encouraging reader commentary, and
HuffPost hired people to help ensure that the conversation would be democratic
and open.^{\href{#endnotes-chapter-9}{3}} ``It is expensive to moderate,'' Hippeau said. ``We have 25 full-time
in-house moderators.''
What HuffPost's founders didn't know at the beginning was how rapidly social
media were going to grow. After all, in 2005, YouTube was just getting started and
Facebook was still confined to colleges and universities. But HuffPost's management
quickly realized that the social media trend fit with their original convictions.
``As the audience embraced social media, we followed, '' said Hippeau. And
that attention to engaging readers—who now contribute 4 million comments a
month on the site—led them to spend more on technology and less on content.
Huffington Post also developed an ability to respond quickly to the data that it
was getting on traffic and usage—something that is a crucial component of success
in digital journalism. Indeed, data analysis has moved from being a required
skill in media companies' finance departments to being an essential part of the
résumé for editors, writers and designers.
%The Story So Far: What We Know About the Business of Digital Journalism

At CNNMoney.com, ``Everyone on the staff has access to page-view and traffic
data,'' says Executive Editor Christopher Peacock. The staff gets daily emails
listing the top 50 stories by section as well as by the entire site. And, he says, ``We
have real-time metrics. We have a proprietary system to tell us how engaging our
headlines and home page are.''
LIN Media, with its 32 local TV broadcast stations, has an integrated content-
management system that distributes content (and allocates costs) across all
its markets and platforms. Its daily report on the previous day's metrics is sent
around to business and editorial departments each morning. ``Sometimes this
report affects broadcast TV decisions as well,'' says Robb Richter, senior vice
president for new media. ``It's like having a great focus group all day long.''
Forbes Chief Product Officer Lewis DVorkin writes a blog about the company's
evolving business practices, often noting the integration of previously independent
departments, functions and platforms at the company. DVorkin sees
this integration as essential to Forbes' digital growth. ``The Web and social media
turned everything upside down. Knowledgeable content creators, audience
members and marketers, too, now possess tools to independently produce and
distribute text.''^{\href{#endnotes-chapter-9}{4}} Expanding readership, once the job of circulation experts, is
now done by business and editorial employees who develop ``audience growth
strategies,'' which shape coverage. When they decide which topics (say, college
tuition) are likely to attract more readers, or different readers, that affects the recruitment
of bloggers, and the efforts of staff and contributors to find followers
and fans.
Forbes also encourages its largest advertisers to contribute content directly
to the magazine and the site as part of their advertising buy. The companies are
given tools to publish content—text, video and photos—on their own page
on the site. This might startle journalists who expect strict separation between
the editorial and business sides, but DVorkin sees this effort as a logical way to
bring in advertisers who know they can create digital content elsewhere, through
websites and email. Labeling the material as coming from advertisers helps inoculate
the company from violating the church-state divide, DVorkin says, adding
that Forbes' approach allows marketers not to be confined in the ``ghetto''
of freelance-written advertorial. The advertisers' material is not edited by Forbes
%Columbia Journalism School | Tow Center for Digital Journalism
%Managing Digital: Audience, Data and Dollars 121
and appears online and in the magazine as ``Forbes AdVoice.''^{\href{#endnotes-chapter-9}{5}} (If it's for the print
edition, Dvorkin reads it for tone, but says he does no more than that.) The print
AdVoice column—limited to one per issue—appears in the table of contents
and may run next to a related story. An online column is featured near relevant
editorial content.
Giving advertisers direct access to an audience without previously approving
the message is a big departure for media companies. The American Society of
Magazine Editors' standards, revised in January 2011, are strict about separating
ad content visually from editorial content, but they are silent on the access issue.^{\href{#endnotes-chapter-9}{6}}
Eric Hippeau, who has gone back to being a venture capitalist, calls this approach
``turning your customers into publishers.'' Advertisers, he says, will not
only create content that will increase traffic, but this will represent ``a great diversification
of revenues'' away from advertising sold by the page view. Before Hippeau
left the HuffPost, the company had just launched a program that charged
flat fees and gave advertisers the opportunity to ``have a conversation'' with the
site's audience through posts and responses. He believes that once companies start
interacting with the audience in this environment, they will be hooked. ``Once a
brand starts that process, they are not going to stop. This is a great benefit to the
media companies.''^{\href{#endnotes-chapter-9}{7}}
Managing digital journalism properties often means stepping away from roles
and job descriptions that were found in traditional operations. At AOL, executives
have decided that content areas such as business or technology should become
their own business units, or ``towns'' in the AOL patois.^{\href{#endnotes-chapter-9}{8}} And editors are
increasingly responsible for determining the revenue potential of stories.
An explicit rendition of AOL's strategy can be found in a 57-page internal
PowerPoint called ``The AOL Way,'' which was leaked to Business Insider in February
2011. The handbook outlines AOL's plans to lower the cost of creating
content while increasing revenue, with explicit targets; it was written a few weeks
before AOL announced its deal to buy the Huffington Post.^{\href{#endnotes-chapter-9}{9}}
The company's rule of thumb is that the cost of acquiring a story should be
no more than half the amount of ad revenue expected to come from that story.^{\href{#endnotes-chapter-9}{10}}
An editor who wants to pay for a premium freelancer must also estimate the size
%The Story So Far: What We Know About the Business of Digital Journalism

of the audience for the assignment—in other words, the editor must cost-justify
every story.^{\href{#endnotes-chapter-9}{11}} In the chart below, the company shows average costs and revenue
by story type. The first column lists the department or source of the story, and
the next column states the average cost of these stories. These figures are then
matched with the ``eCPM''—or effective cost (to advertisers) per thousand page
views—and from that, AOL estimates the number of page views needed to break
even on that story.^{\href{#endnotes-chapter-9}{12}}
One method of building traffic is to hop onto hot topics, the document advises.
``Use editorial insight and judgment to determine production,'' the document
says, offering as an example that if ``Macaulay Culkin and Mila Kunis are trending
because they broke up,'' someone should ``write a story about Macaulay Culkin
and Mila Kunis.'' And editors are told to always keep expenses in mind. The cost
of content can run from \$25 for a freelance article that needs 7,000 page views
to break even, to a \$5,000 video that will require a half-million streams to recover

%Columbia Journalism School | Tow Center for Digital Journalism
%Managing Digital: Audience, Data and Dollars 123
its costs.^{\href{#endnotes-chapter-9}{13}} Catchy headlines, such as ``Lady Gaga Goes Pantless in Paris'' (from
AOL site StyleList.com) are important to entice readers from search.^{\href{#endnotes-chapter-9}{14}} Similarly,
an article headlined ``Benadryl for Dogs'' should cost \$15, because its revenue
potential is around \$26.^{\href{#endnotes-chapter-9}{15}} ``We are heavily invested in analytics as this is the way
to empower our editors and journalists,'' Neel Chopdekar, vice president at AOL
Media, said in an interview shortly before ``The AOL Way'' was made public. He
calls this ``bionic journalism—the best of man and machine.''
Paying freelancers by performance is not as unusual a practice as paying editorial
staff that way. About.com, the general information website founded in 1995
and now owned by the New York Times, pays its expert writers, or ``guides,'' by
performance.^{\href{#endnotes-chapter-9}{16}} USA Today announced in early April that it is considering paying
bonuses to writers based on page views.^{\href{#endnotes-chapter-9}{17}}
Digital companies, which lean heavily on part-time contributors or unpaid
commenters, are constantly on the lookout for cheap labor. And mainstream
news companies have long offered psychic, rather than financial, rewards to its
reporters and editors.
Forbes' DVorkin is experimenting with pay schemes for blogging ``contributors,''
whom Forbes compensates with a flat monthly fee. On top of that, Forbes
pays a bonus if a writer reaches a certain target of unique visitors. (DVorkin declined
to give details about pay at Forbes, but he did say that at True/Slant—the
Web company he owned before coming to Forbes—contributors would typically
earn about \$200 per month, and some would get twice that much, counting
their bonuses. A ``few'' earned several thousand dollars a month.) DVorkin said
he'd like to add more metrics to the calculations—for example, Twitter followers
or repeat visitors.
``In the newsroom, we are trying to develop different currencies to value success,
“ says CNNMoney.com's Peacock. Journalists typically feel rewarded when
their stories run on the front of a print publication, or lead the evening news.
“We are trying to develop different kinds of 'front page' experiences for the
journalists,'' Peacock says. But the new standards, such as the number of page
views or comments, are often beyond an editor's control and are just as likely to
be determined by readers, aggregators or bloggers.
%The Story So Far: What We Know About the Business of Digital Journalism

Digital executives also must constantly decide when to deploy staff to work
on new or experimental products that don't meet the new productivity tests.
Tumblr, the social media microblogging platform, gets billions of monthly page
views overall, but its value to most media companies is still negligible. Some
journalists are fascinated with what it might become in terms of driving traffic
or buzz—and their employers let them spend time with the platform, if only to
be sure they don't miss out on something that might turn into the next Twitter.
For example, GQ has a Tumblr site with just 12,000 followers—a tiny fraction of
the print magazine's monthly circulation of 800,000. GQ's senior editor, Devin
Gordon, says ``Tumblr is a side project, but I care a lot about it.'' He and an
editorial assistant limit themselves to no more than two hours per week posting
Tumblr content.
Managers of digital operations must also deal with journalists who are able to
establish a following on the basis of their own talents rather than the prestige or
reach of the news organization. Andrew Sullivan's Daily Dish was responsible
for 1 million monthly unique visitors, or about 20 percent of the traffic, at the
Atlantic's site. But it was Sullivan's audience, not the Atlantic's; the blogger owned
the brand equity. So when he moved to The Daily Beast in April 2011, he took
his unique users with him. This phenomenon isn't entirely new, of course. Columnists
like Walter Winchell would change employers in the glory days of the
1920s tabloid wars.^{\href{#endnotes-chapter-9}{18}} But in digital journalism, audiences can follow stars with
great ease, and conceivably journalists with big individual followings could begin
to keep, and try to make money from, data about their readers, rather than leaving
that to their employers.
DVorkin says he is changing the way he judges the quality of a reporter. ``It
used to be a question of how they develop their sources. Now it's how they develop
their sources and their audience.'' He expects Forbes journalists not just to
cover news, but to be ``maestros'' of comments and of followers. And they have to
be recruiters. ``When we used to hire a reporter, we'd say, 'Show me some clips.'
Now I say, 'Who is in your orbit? Who are your sources? Who do you know?
Who can you convince to contribute?' ''
%Columbia Journalism School | Tow Center for Digital Journalism
%Managing Digital: Audience, Data and Dollars 125
The brutally competitive nature of digital journalism also extends to advertising
sales, and many traditional media companies have a hard time justifying
a large commitment to the effort simply because the returns, at least initially,
can be so small. Consider the case of a company that publishes family-oriented
magazines and a website. The company (which asked for confidentiality in return
for providing its data) runs a profitable monthly print magazine, with free
distribution of 400,000 copies within a top-five metro area. But its attempts to
replicate its success on the Web haven't worked out. The site associated with the
publication gets 200,000 unique users and 1.5 million page views per month.
The expenses associated with the site amount to only around \$181,000 per year,
but that isn't quite covered by its ad revenue. Indeed, digital advertising accounts
for just under 4 percent of the company's \$4.68 million in annual ad revenue.^{\href{#endnotes-chapter-9}{19}}
Of the company's 1,500 ad clients, 100 are online, and of those, only about 10
advertisers are exclusively digital
LIN Media sells an estimated \$30 million in advertising from its Web and mobile
efforts; that represents about 7 percent of total revenue, or a significantly larger
percentage than many other local broadcasters claim.^{\href{#endnotes-chapter-9}{20}} But to put that figure
into perspective, compare it to automobile advertising, which typically accounts
for about 20 to 25 percent of total ad revenue for local broadcast companies.
These companies face an ongoing dilemma. If they didn't make an effort to
sell digital advertising, they wouldn't lose much income—for now. But they believe
that digital delivery of their content is bound to grow over time, so they
are investing in working out pricing and customer relations even though the immediate
return doesn't justify the effort. Whether they can play out these digital
advertising calculations successfully depends on the quality of their management.

\chapter{Conclusion}
``Here's the problem: Journalists just don't understand their business.''
That's the diagnosis from Randall Rothenberg, a former New York Times media
reporter who heads the Interactive Advertising Bureau, a trade group representing
publishers and marketers.
Whether or not you agree with his sweeping characterization, it's clear that
many sectors of the traditional news industry have been slow to embrace changes
brought on by digital technology. They also have been flummoxed by competitors
who invest minimally in producing original content but have siphoned off
some of the most profitable parts of the business.
At the same time, digital journalism has created significant opportunities for
news organizations to rethink the way they cover their communities. And in
several organizations, old and new, we see promising signs that a transformed
industry can emerge from the digital transition—one that is leaner, quicker and,
yes, profitable.
We do not believe that legacy platforms should be disregarded or disbanded.
It simply is not reasonable to assume that any company would cast aside the part
of its business that generates 80 to 90 percent of its revenue. But we do think
that companies ought to regard digital platforms and their audiences as being in
a state of constant transformation, one that demands a faster and more consistent
pace of innovation and investment.
To that end, we offer these recommendations:
\begin{itemize}
\item Digital platforms have been treated too often by traditional news organizations
as just another opportunity to publish existing content. Many sites are
filled with ``shovelware''—content that amounts to little more than electronic
editions of words and pictures from traditional platforms. But, as we
have seen, publishers can build economic success by creating high-value,
less-commoditized content designed for digital media. New York Magazine's
successful site gets little traffic from print-edition stories; KSL.com's classThe
Story So Far: What We Know About the Business of Digital Journalism
130
fied ads are not part of its broadcast program; and the Dallas Morning News
provides online football coverage that would be impossible to replicate in its
sports section.
\item Media companies should redefine the relationship between audience and
advertising. They have spent a great deal of time and resources building
masses of lightly engaged readers. And the industry has turned online ads
into what Rothenberg calls low-value ``direct-response advertising—a.k.a.,
junk mail.'' That kind of advertising is dependent on volume—a game publishers
will never win when competing with behemoths like Facebook and
Google. This is not a goal that can be accomplished just by the business side.
Journalists must make a fuller commitment to understanding the audiences
they have and the ones they want, and to revamping their digital offerings to
ensure deeper loyalty.
\item Media companies ought to rethink their relationships with advertisers. This
doesn't mean allowing them to dictate coverage or news priorities. It does
mean understanding that advertisers now have many more ways to reach
customers than they used to and that some of these methods, such as social
media, can be cheap and effective. News organizations have their own
strengths: They produce journalism that is geared to their communities, and
they employ sales forces who know their markets—both of which should
give them a competitive advantage. They can act as guides to the digital era,
helping companies produce new-media ads, place them online for maximum
impact and learn such digital fundamentals as getting better positioning
on search engines.
\item News and marketing companies should develop alternatives to the impression-
based pricing system (that is, pricing by CPM, or cost per thousand)
that dominates online advertising. Small publishers have been successful
selling ads by the week or month rather than by volume. Many large advertisers
and ad agencies will insist on paying by the impression, but news
organizations need to build upon their current pricing schemes by combining
digital ads more effectively with broadcast or print, social-media
outreach and other methods. Moreover, media companies must come up
with ways to build content value into digital display ads; as others have also
%Columbia Journalism School | Tow Center for Digital Journalism

noted, too many of them are relics of a decade ago—boxes on a page that
convey little of the information or appeal that historically made advertising
valuable to consumers.
\item News organizations must be vigilant about outright theft of their content,
but they should also realize that most aggregators operate within the bounds
of copyright law and are generating value for readers. This means news sites
must do more than simply insert links (most of which are never clicked)
within stories, and instead develop a thoughtful approach to understanding
what topics best lend themselves to aggregation and how best to engage
their readers in the effort.
\item It is asking a lot to expect a legacy division—in news or ad sales—to embrace
such a radically different world as digital. Retraining gets you only so
far. Small, traditional news organizations may find it impossible to set up separate
divisions. But bigger companies should analyze the potential in creating
separate digital staffs, particularly on the business side. We did find successful
companies with integrated digital and legacy departments, but others have
demonstrated that they can compete more effectively by deploying committed
digital-only teams that adapt to rapidly changing circumstances.
\item Journalists must be prepared for continued pressure on editorial costs. There's
an old rule of thumb in the newspaper world, that every 1,000 readers supports
one newsroom staffer. That kind of thinking isn't going to hold in the
digital world. We are likely to see a world of more, and smaller, news organizations,
the most successful of which will leverage their staffs and audience
by using aggregation, curation and partnerships with audiences to provide
content of genuine value.
\item Mobile digital devices represent a special challenge for news companies; for
every successful new product or new platform, there will be others the company
tried that didn't work. If a company can place small bets on many ventures,
the probability increases that one will win.
\item Any news site that adopts a pay scheme now should have very limited expectations
for its success—at least on the Web. In the case of a print publication,
requiring digital readers to pay may help to slow circulation losses, but that
%The Story So Far: What We Know About the Business of Digital Journalism

is hardly a long-term solution. A pay plan merged with an ambitious strategy
to improve users' experience on mobile platforms has a much better chance
to succeed.
\end{itemize}
We restate the bias we offered at the beginning of this report: We believe
the public needs independent journalists who seek out facts, explain complex
issues and present their work in compelling ways. We also believe that while
philanthropic or government support can help, it is ultimately up to the commercial
market to provide the economic basis for journalism. The industry has
realized many of the losses from the digital era. It is time to start reaping some
of the benefits.

\chapter{Acknowledgements}
We owe a great debt to many people who contributed to this report. While we
can't name them all here, we wish to thank some of those most deeply involved.
Nicholas Lemann, dean of Columbia's Journalism School, hatched the idea
for the report and has consistently guided our efforts with wisdom and skill.
Jeffrey Frank and Marcia Kramer carefully edited our copy and saved us from
verbosity—or worse. Janice Olson took germs of our ideas and turned them into
wonderful graphics. Emily Bell, director of the Tow Center for Digital Journalism,
and Elizabeth Fishman, who oversees communications for the Journalism
School, provided expert advice throughout.
Our families, friends and colleagues tolerated our absences with patience and
provided wonderful insight that helped shape our findings.
Most importantly, we would like to thank the dozens of publishers, journalists
and salespeople who opened their doors and books to us and dealt with our
many questions. For all the difficulties that journalism faces these days, we were
deeply impressed and encouraged by their commitment to this business, and to
ensuring that citizens will continue to get the information they need to lead
their lives.

\chapter{Endnotes}
\section{Endnotes: Introduction}
1 Apple's split-adjusted stock price was about \$10 (from Yahoo Finance, http://yhoo.it/
ecaGAO), while Knight-Ridder's was around \$60 (Grain Market Research, http://bit.ly/
dGf5F5). Apple had far more shares outstanding, leading to valuations that are within 1 percent
of each other.
2 Entire contents of “The Reconstruction of American Journalism” can be found on CJR.org,
http://bit.ly/eP6Fjl
3 “All the News That's Fit to Subsidize,” op-ed from WSJ.com, Oct. 21, 2009.
http://online.wsj.com/article/SB10001424052748704597704574486242417039358.html
4 Media commentator Alan Mutter addressed this in a post, “Non-profits can't possibly save the
news,” Reflections of a Newsosaur, March 30, 2010, http://bit.ly/dMp86C. He calculated the
news media would need an endowment of \$88 billion to produce enough revenue to support
current models.
5 Stewart Brand, The Media Lab (Penguin, 1988).
6 James Hamilton, All the News that's Fit to Sell (Princeton University Press, 2003).
7 Carl Shapiro and Hal R. Varian, Information Rules (Harvard Business School Press, 1999).


%<h3 id="endnotes-chapter-1">Endnotes: Chapter 1</h3>
\section{Endnotes: Chapter 1}
1 Rick Edmonds, ``An Online Rescue for Newspapers?'', Poynter.org, Jan. 27, 2005.
http://bit.ly/gphsCR
2 Figures from Newspaper Association of America data. http://bit.ly/h4dxxf
3 ``State of the News Media 2011: Network by the Numbers,'' Pew Research Center's Project
for Excellence in Journalism. http://bit.ly/eH71Ld
4 Carl Sessions Stepp, ``State of the American Newspaper, Then and Now,'' American Journalism
Review, September 1999. http://bit.ly/eDev0Y
5 Vin Crosbie, ``The Placebo Called Convergence,'' June 9, 2010. http://bit.ly/ft8f8b
6 ``Internet Gains on Television as Public's Main News Source,'' Pew Research Center for the
People & the Press, Jan. 4, 2011. http://bit.ly/ia45aw
7 ``State of the News Media 2011: Mobile News and Paying Online,'' Pew Research Center's
Project for Excellence in Journalism. http://bit.ly/fsVAWf
8 ``McClatchy Reports Fourth Quarter 2010 Earnings,'' Feb. 8, 2011. http://bit.ly/hsfERQ
9 Edwin Diamond, ``Trouble in Paradise,'' New York Magazine, March 3, 1986, page 52.
http://bit.ly/hPZgZP
10 Remarks made at Borrell Associates Local Online Advertising Conference, March 3, 2011.
The Story So Far: What We Know About the Business of Digital Journalism
20
11 Lucas Graves et al, ``Confusion Online: Faulty Metrics and the Future of Digital Journalism,''
September 2010. http://bit.ly/hBPwt7
12 A study of local online media has this to say: ``Content is king, but not the content most
people think. News and information sites do indeed generate revenue, but the top five local
online companies derive all their content from their own advertisers.'' From ``Benchmarking
Local Online Media: 2010 Revenue Survey,'' Borrell Associates.
13 Alan D. Mutter, ``Mission Possible? Charging for Web Content,''

\section{Endnotes: Chapter 2}
%<h3 id="endnotes-chapter-2">Endnotes: Chapter 2</h3>
1 In its results for the second quarter of 2010, http://bit.ly/h31e8V, the Times Co. says 26
percent of its total ad revenue comes from online. But for the New York Times Media Group,
which includes the namesake property and the International Herald Tribune (print and
online), circulation revenue is almost as significant as advertising. Thus, digital certainly
represents less than 20 percent of total revenue for the NYT's paper and site, though the
company doesn't break the results out in more detail.
2 ``Morgan Stanley's Meeker Sees Online Ad Boom,'' Bloomberg Businessweek, Nov. 16, 2010.
http://buswk.co/dP8wQU (full presentation available at http://slidesha.re/dHqdrC). A March
2011 study by eMarketer, http://bit.ly/htZ3Mw, put the time spent vs. ad spending disparity
at 25.2 vs. 18.7 percent for Internet and 8.1 vs. 0.5 percent for mobile.
3 According to a comScore study released in March 2011, ``Lessons Learned, Maximizing
Returns with Digital Media,'' 30 percent of all U.S. Internet users delete their cookies, up to
six times a month. That can result in a 250 percent overcounting of unique visitors to a site.
Slide 6 of http://bit.ly/hhIf0y
4 Jimmy Orr, ``Latimes.com Has Record Page Views in March,'' latimes.com, April 8, 2011.
http://lat.ms/i435ob
5 ``Nielsen Analysis,'' State of the News Media 2010, Pew Research Center's Project for
Excellence in Journalism. http://bit.ly/gcIRQ2 The study also notes that the average visitor
spends 10 minutes a month on newspaper or local TV sites, while cable news sites get close
to 24 minutes per month.
6 ``Newspaper Engagement,'' submission to Newspaper Association of America Marketing
Conference, Feb. 23, 2006. http://bit.ly/i2b6Eg
7 Scout Analytics is actually measuring devices, not humans, but there is reason to believe the
numbers even out in some fashion. Many people use more than one device in a month to
access a site, but also some devices, especially home computers, are used by more than one
person in the same period of time.
8 ``Importance of Analyzing Unit Cost of Engagement in Advertising,'' Digital Equilibrium blog,
Nov. 29, 2010. http://bit.ly/ihY3nc
9 ``Engagement as the Unit of Monetization,'' Digital Equilibrium blog, Oct. 25, 2010.
http://bit.ly/ejvcJp
10 Remarks at Borrell Associates Local Online Advertising Conference, March 3, 2011.
The Story So Far: What We Know About the Business of Digital Journalism
11 Erin Pettigrew, ``Strengthening Our Core (Readership),'' Gawker Media, March 5, 2010.
http://bit.ly/hnQnLv
12 Felix Salmon, ``The New Gawker Media,'' Reuters.com, Dec. 1, 2010. http://reut.rs/gBg6lt
13 Eric Peterson and Joseph Carrabis, ``Measuring the Immeasurable: Visitor Engagement,''
Web Analytics Demystified, 2008. http://bit.ly/hvFuio PBS later adapted the formula to
designate its most loyal users as those who view at least 3.2 pages per visit; stay at least 2.57
minutes on the site; have visited the site within the past two weeks; and visit the site at least
three times a month.
14 Examiner articles at http://exm.nr/gZSbnU and http://exm.nr/fI2JEJ
15 Edmund Lee, ``Does Who Creates Content Matter to Marketers in a 'Pro-Am' Media World?'',
AdAge, June 7, 2010. http://exm.nr/fI2JEJ
16 Examiner gallery and article at http://exm.nr/fHdRPW and http://exm.nr/eonVdC
17 Gawker's Nick Denton wrote that ``clickthroughs are an indicator of the blindness, senility
or idiocy of readers rather than the effectiveness of the ads.'' From ``Why Gawker is moving
beyond the blog,'' Lifehacker blog, Nov. 30, 2010. http://lifehac.kr/gVWcuF For more on the
inutility of clicks, see comScore study, op. cit. Slide 4 of http://bit.ly/hhIf0y which reports that
84 percent of all U.S. Internet users never click on an ad in a given month, and that there are
50 percent fewer clickers in 2011 than in 2007.
18 comScore press release, ``U.S. Online Display Advertising Market Delivers 22 Percent Increase
in Impressions vs. Year Ago,'' Nov. 8, 2010. http://bit.ly/ezZAYa
19 ``Online: Key Questions''


\section{Endnotes: Chapter 3}
%<h3 id="endnotes-chapter-3">Endnotes: Chapter 3</h3>
1 See, for instance, Laura McGann, ``Six reasons to watch local news project TBD's launch next
week,'' Nieman Journalism Lab, Aug. 6, 2010. http://bit.ly/dRnAxQ
2 Alan D. Mutter, ``Hyperlocals like TBD: More hype than hope,'' Reflections of a Newsosaur,
Feb. 24, 2011. http://bit.ly/fVi6M0
3 Paul Farhi, ``Allbritton Communications slashes staff at reorganized TBD.com,'' Washington
Post, Feb. 23, 2011. http://wapo.st/g8XcRa
4 Rafat Ali, ``Politico Crushing It On Revs, Profits In Fiscal '09; Changes Ownership Structure,''
paidcontent.org, Jan. 4, 2010. http://bit.ly/eMimy7
5 Michael Wolff, ``Politico's Washington Coup,'' Vanity Fair, August 2009. http://bit.ly/eyx79Y
6 Paul Farhi, ``TBD.com making its move into the crowded market of local news,'' Washington
Post, Aug. 7, 2010. http://wapo.st/hTwEAG
7 Main Street Connect, ``Community News.'' http://bit.ly/e4nfiB
Columbia Journalism School | Tow Center for Digital Journalism
Local and Niche Sites: The Advantages of Being Small 53
8 David Kaplan, ``AOL Buying Hyperlocal News Aggregator Outside.in; Will Align With Patch,''
paidcontent.org, March 4, 2011. http://bit.ly/fZBjsA; and Dan Frommer, ``AOL Buys Outside.
In, Less Than \$10 Million,'' Business Insider, March 4, 2011. http://read.bi/eBKRaJ
9 Ken Auletta, ``You've Got News,'' New Yorker, Jan. 24, 2011. http://nyr.kr/eLd7Qn
10 Robert G. Picard, ``Journalism, Value Creation and the Future of News Organizations,'' Joan
Shorenstein Center on the Press, Politics and Public Policy, Spring 2006. http://bit.ly/ep2ufT
11 David Saleh Rauf, ``Dispatches from the Last Frontier,'' American Journalism Review,
December/January 2011. http://bit.ly/fpMpLF
12 comScore, ``The New York Times Ranks as Top Online Newspaper According to May 2010
U.S. comScore Media Metrix Data,'' June 16, 2010. http://bit.ly/ecqJ37
13 Michele McLellan, ``Block by Block: Building a new news ecosystem,'' Reynolds Journalism
Institute. http://bit.ly/kIjVwv
14 Anthony Tjan, ``DailyCandy's Accidental Entrepreneur: An Interview with Dany Levy,''
Harvard Business Review, Oct. 14, 2009. http://bit.ly/fmnRzD
15 Peter Kafka, ``Comcast Buys DailyCandy For \$125 Million,'' Business Insider, Aug. 5, 2008.
http://read.bi/fg6zNh
16 Dennis K. Berman and Julia Angwin, ``Former AOL Official Pittman Puts Web Firm Daily
Candy Up''


\section{Endnotes: Chapter 4}
%<h3 id="endnotes-chapter-4">Endnotes: Chapter 4</h3>
1 Bryan Chaffin, ``Analyst Ups Q4 iPad Shipments to 6.3 Million,'' The Mac Observer, Dec. 2,
2010. http://bit.ly/fVdrQD; Yenting Chen and Joseph Tsai, ``Apple expects shipments of 6-6.5
million iPads in 1Q11,'' Digitimes, March 2, 2011, http://bit.ly/fEiSkb; Pascal-Emmanuel
Columbia Journalism School | Tow Center for Digital Journalism
The New New Media: Mobile, Video and Other Emerging Platforms 65
Gobry, ``iPad Shipments Will Hit 65 Million In 2011, Says Analyst,'' Business Insider, Dec. 29,
2010. http://read.bi/hDFy5C; Anna Johnson, ``iPad Sales To Grow By 127 Percent in 2011,''
Kikabink News, Dec. 15, 2010. http://bit.ly/fQXWNy
2 Mobile DTV and ITV presentation by Pearl consortium at Borrell conference, March 3, 2011.
3 Andrew Vanacore, ``Publishers see signs the iPad can restore ad money,'' Associated Press,
June 3, 2010, http://usat.ly/f9wnqi.
4 Joe Pompeo, ``iPad Owners Spend An Hour Or More Reading A Single Magazine On The
Device,'' Business Insider, June 7, 2010. http://read.bi/htam4c
5 Christopher Hosford, ``Tablets, contextual ads drive discussion at summit,'' B2B,
March 14, 2011. http://bit.ly/eNlYji; David Kaplan, ``Hearst's Carey: Tablets Will Provide
25 Percent Of Magazines' Circulation'' paidcontent.org, March 9, 2011. http://bit.ly/ibBilx
6 ``comScore Reports January 2011 U.S. Mobile Subscriber Market Share,'' comScore,
March 7, 2011. http://bit.ly/gPjCfz
7 John Koblin, ``Memo Pad: iPad Magazine Sales Drop,'' WWDMedia, Dec. 29, 2010.
http://bit.ly/gSJg66 Also see Peter Kafka, ``Wired's Newest iPad Issue Boasts Its Best Feature
Yet: Free'' AllThingsDigital, April 15, 2011. http://bit.ly/dImJ8w
8 ABC Publisher's Statement, Dec. 31, 2010.
9 Nat Ives, ``Conde Nast Taps Brakes on Churning Out iPad Editions for All Its Magazines,''
AdAge, April 22, 2011, http://bit.ly/i6KHv7.
10 Existing customers who subscribe to an app from a publisher's website and use the publisher's
billing services aren't subject to the fee.
11 MG Siegler, ``Apple Now Has 200 Million iTunes Accounts, Biggest Credit Card Hub On
Web'' TechCrunch, March 2, 2011. http://tcrn.ch/fRLtFW
12 Amir Efrati, Mary Lane and Russell Adams, ``Google Elbows Apple, Woos Publishers,''
Wall Street Journal, Feb. 17, 2011. http://on.wsj.com/iaYu7B
13 Aparajita Saha-Bubna, ``The Journal Adds 200,000 Mobile-Device Subscribers,''
Wall Street Journal, March 11, 2011. http://on.wsj.com/gzVUZZ
14 ABC Audit Statement, September 30, 2010
15 John Biggs, ``Time Inc. Releases Sports Illustrated Digital Subscriptions,'' TechCrunch,
Feb. 11, 2011. http://tcrn.ch/fG5mW4
16 In March 2011, Sports Illustrated subscriptions were priced in various ways. An ``All Access''
subscription was billed at \$4.99 monthly and included an Android app, print magazine and
web access. The same plan paid by the year cost \$48 and included a free windbreaker.
A digital-only subscription costs \$3.99 a month.
17 Staci Kramer, ``'Daily' Publisher Disputes Subscription Numbers; Says 5,000 Far Too Low,''
paidcontent.org, March 17, 2011. http://bit.ly/eeqQ2A
18 ``The comScore 2010 U.S. Digital Year in Review,'' comScore, Feb. 7, 2011, page 22.
http://bit.ly/eq7mjE
19 Alex Weprin, ``This Is Where CNN Makes Its Money,'' TVNewser, May 27, 2010.
http://bit.ly/edQDUd
20 Borrell Associates, ``Benchmarking Local Online Media: 2010 Revenue Survey,'' March 2011.
http://bit.ly/eSBkXU
21 News Release. ``LIN TV Corp. Announces Fourth Quarter and Full Year 2010 Results,''
March 16, 2011. http://bit.ly/gvSTwP
22 LIN Media website: http://bit.ly/hGC6Up
23 Based on internal count of unique users in 2010.
24 Mallary Jean Tenore, ``How The Miami Herald cultivates loyal audience for video, its second
biggest traffic driver,''


\section{Endnotes: Chapter 5}
%<h3 id="endnotes-chapter-5">Endnotes: Chapter 5</h3>
1 Stewart Brand, The Media Lab (Penguin, 1988), p. 202
2 John Lanchester, ``Let Us Pay,'' London Review of Books, Dec. 16, 2010. http://bit.ly/h8gIt8
3 Bill Grueskin, ``The Case for Charging to Read WSJ.com,'' guest post on Reflections of a
Newsosaur blog, March 22, 2009. http://bit.ly/f2UB3x
4 Joel Meares, ``Jim VandeHei talks Politico Pro,'' Columbia Journalism Review, Nov. 16, 2010,
http://bit.ly/fSmKZb; see also Jeremy W. Peters, ``Politico, Seeing a Market Need, Adds a Paid
News Service,'' New York Times, Oct. 25, 2010. http://nyti.ms/hg1lPA
5 ``Now Pay Up,'' The Economist, Aug. 27, 2009. http://econ.st/hJOUsM
6 David Smith, ``Papers in U.S. losing readers; Democrat-Gazette gains subscribers,''
Arkansas Democrat-Gazette, April 27, 2010. http://bit.ly/fcoFs7
7 Information provided by Miami Herald circulation department.
8 ``Old and New Media Go to Washington,'' from On the Media, NPR, May 8, 2009.
http://bit.ly/hwsTyu
9 Transcript of remarks at Carnegie Corporation, provided by James Moroney.
10 Justin Ellis, ``Dallas Morning News publisher on paywall plans: 'This is a big risk,' '' Nieman
Journalism Lab, Jan. 6, 2011. http://bit.ly/eg0dSH
11 ``Annual Report and Accounts 2009,'' Pearson PLC. http://bit.ly/eO7Yfj
12 FT's pricing plan details. http://on.ft.com/gY98fv. The FT's print circulation in 2011
was 381,658.
13 ``2010 Results Presentation,'' Pearson, Feb. 28, 2011. Slide 36, http://bit.ly/gXmf9p
The Story So Far: What We Know About the Business of Digital Journalism
82
14 Bill Grueskin, ``NYTimes.com Pay Scheme has a Great Big Hole,'' paidcontent.org,
March 18, 2011. http://bit.ly/i5hyaZ
15 The 85 percent figure comes from the Times' own story: Jeremy W. Peters, ``The Times
Announces Digital Subscription Plan,'' March 17, 2011. http://nyti.ms/hBJvt6
16 One estimate says the Times was spending \$40 million to \$50 million: Brett Pulley, ``New
York Times Fixes Paywall Flaws to Balance Free Versus Paid on the Web'' Bloomberg.com,
Jan. 28, 2011, http://bloom.bg/fyCwLV. Publisher Arthur Sulzberger has said that's not
accurate, and another journalist, Staci Kramer at paidcontent.org, put the price at \$25 million:
``New York Times Paywall Cost More Like \$25 Million,'' http://bit.ly/fE71o2. Meanwhile,
a former design director for the Times' site, Khoi Vinh, noted the opportunity cost for the
Times in his post, ``What the NYT Pay Wall Really Costs,'' subtraction.com, March 18, 2011,
http://bit.ly/eHb6F9
17 Staci D. Kramer, ``The NYT Pay Plan's Most Dangerous Foe: Perception,'' paidcontent.org,
March 27, 2011. http://bit.ly/dO7WdI
18 Michael DeGusta, ``Digital Subscription Prices Visualized (aka The New York Times is
Delusional),'' theunderstatement.com, March 21, 2011. http://bit.ly/hXON9f
19 Lauren Kirchner, ``Don't Call it a Paywall,'' CJR, April 6, 2011. http://bit.ly/f72MEa
20 Tiernan Ray, ``NY Times Sags; 100K Paid Digital Subs, and No Loss to 'Premium'
Advertising,'' Barrons, April 21,2011. http://bit.ly/fAHfjp
21 Jonathan Landman, the Times' former deputy managing editor for digital journalism, quoted in
Jeremy W. Peters, ``The Times' Online Pay Model Was Years in the Making,'' New York Times,
March 20, 2011. http://nyti.ms/ediOZO
22 Edward J. Delaney, ``Charging (a lot!) for news online: The Newport Daily News' new
experiment with paid content,'' Nieman Journalism Lab, June 8, 2009. http://bit.ly/gtV5Hn
23 ``The Consumer and Content: Benchmark Study,'' AOL, September 2010. Slide 47 et al.
http://bit.ly/fYAE1T
24 ``Economic Attitudes,'' State of the News Media 2010, Pew Research Center's Project for
Excellence in Journalism. http://


\section{Endnotes: Chapter 6}
%<h3 id="endnotes-chapter-6">Endnotes: Chapter 6</h3>
1 Video, ``Arianna and AOL CEO Tim Armstrong Teach Journalism Class At Brooklyn Middle
School,'' Huffington Post, March 17, 2011. http://huff.to/i4ozzo
2 Bill Keller, ``All the Aggregation That's Fit to Aggregate,'' New York Times, March 10, 2011.
http://nyti.ms/htb7Xk
3 The Lede, New York Times. http://nyti.ms/hNf3xZ
4 The Project for Excellence in Journalism and the Pew Internet & American Life Project,
``The State of the News Media: Nielsen Analysis.'' http://bit.ly/fflY0M ``In making these
categories PEJ looked at the front page of each site and counted the links on the site.
If two-thirds of the links on the site were original content, the site was labeled an originator.
If two-thirds of the links were to outside content, the site was categorized as an aggregator.
Commentary sites are those that do not have original content in terms of original reporting,
but have content that is mostly commenting or discussing reporting done by others.''
5 Ibid.
6 One such example: Arianna Huffington, ``Journalism 2009: Desperate Metaphors,
Desperate Revenue Models, And The Desperate Need For Better Journalism,'' Dec. 1, 2009.
http://huff.to/fEyZp1
7 Gabriel Sherman, ``Going Rogue on Ailes Could Leave Palin on Thin Ice,'' nymag.com,
March 13, 2011. http://bit.ly/hcqWRc
8 Jack Mirkinson, ``Roger Ailes Told Palin Not To Make 'Blood Libel' Video: NY Mag,''
Huffington Post, March 14, 2011. http://huff.to/ieYDh9
9 ``Tube Mogul Online Video Best Practices,'' December 2010. http://bit.ly/f426qV The average
video featured on the YouTube home page gets 86,000 views per day.
10 Kimberly Isbell, ``What's the law around aggregating news online?'' Nieman Journalism
Lab, Sept. 8, 2010. http://bit.ly/hIXmUc The definition and distinctions among kinds of
aggregation informed our discussion of these differences.
11 ``Top 20 Websites and Engines,'' Hitwise, April 16, 2011. http://bit.ly/frLcYt Hitwise is a
company that measures online audiences using data aggregated from Internet service providers.
12 ``About the updates to Google News,'' Google News site. http://bit.ly/eh1tZI
13 Erick Schonfeld, ``Exclusive: An Early Look At News.me, The New York Times' Answer To
The Daily,''TechCrunch, Feb. 1, 2011, http://tcrn.ch/fb2KRw; Russell Adams, ``Paper Starts
New Website; Washington Post's Trove to Allow Readers to Build Custom Views of Online
News,'' Wall Street Journal, Feb. 11, 2011. http://on.wsj.com/gbktyg
The Story So Far: What We Know About the Business of Digital Journalism
92
14 Video, ``The Huffington Post Posts About the''


\section{Endnotes: Chapter 7}
%<h3 id="endnotes-chapter-7">Endnotes: Chapter 7</h3>
1 U.S. Department of Commerce, ``The Emerging Digital Economy,'' April 1998.
http://bit.ly/fLtcKX
2 ``Inland's 'Rules of Thumb' 2008,'' Inland Press, Sept. 8, 2008. http://bit.ly/eHlfkX
3 John Paton, ``Presentation at INMA Transformation of News Summit,'' Digital First,
Dec. 2, 2010. http://bit.ly/dT0D3R
4 Richard Pérez-Peña, ``New York Times Plans to Cut 100 Newsroom Jobs,'' New York Times,
Feb. 14, 2008. http://nyti.ms/ezJtN9
5 Zachary M. Seward, ``An extremely expensive cover story—with a new way of footing the
bill,'' Nieman Journalism Lab, Aug. 28, 2009. http://bit.ly/fHZ8qU
6 John Paton, ``I Promised—You Delivered—The Checks Are Cut,'' Digital First,
March 14, 2011. http://bit.ly/fFmdHp
7 Heather Leonard, ``Google's Share Of The Total Online Ad Market To Increase Even More,''
Business Insider, March 3, 2011. http://read.bi/hzajdl
8 See The New York Times Company, ``2010 Annual Report,'' Feb. 22, 2011.
http://bit.ly/i4Zhxt and MarketWatch, ``Yahoo! Reports Fourth Quarter 2010 Results,''
Jan. 25, 2011. http://bit.ly/hVBW1B
9 Erik Schonfeld, ``Facebook Now Has Yahoo In Its Sites, Already Bigger In Pageviews,''
TechCrunch, Feb. 1, 2010. http://tcrn.ch/gKquzo
10 Henry Blodget, ``BUSINESS INSIDER SECRETS REVEALED: An Inside Look At
Our Readership And Financial Performance,'' Business Insider, March 7, 2011.
http://read.bi/gDGml9
11 Jeremy W. Peters, ``Web Focus Helps Revitalize The Atlantic,'' New York Times, Dec. 12, 2010.
http://nyti.ms/fSxPEJ and Matt Kinsman, ``The Atlantic Posts Profit for First Time In Years,''
Folio, Jan. 6, 2011. http://bit.ly/eCrPt2
12 Jeremy W. Peters, op.cit.
13 United States Conference of Mayors, ``U.S. Metro Economies: GMP—The Engines of
America's Growth,'' June 2008. http://bit.ly/gzYWRy
14 Bill Shea, ``Free Press, News struggle after revising their JOA,'' Crain's Detroit Business,
Aug. 1, 2010. http://bit.ly/i2v8GX
15 Bill Shea, ``Newspaper unions at Detroit dailies ratify two-year contract,'' Crain's Detroit
Business, Nov. 15, 2010. http://bit.ly/hBW9Yo
16 Figures provided to the authors by Detroit Media Partnership.

\section{Endnotes: Chapter 8}
%<h3 id="endnotes-chapter-8">Endnotes: Chapter 8</h3>
1 ``The Web Is Not a Selling Medium,'' Art Bin, August 1998. http://bit.ly/eQcBwa
2 Sweeney's bio is at http://bit.ly/i06QiK
3 Jeremy Peters, ``Newspaper Circulation Falls Broadly but at a Slower Pace,'' New York Times,
Oct. 25, 2010, http://nyti.ms/ey8hHC and AdAge circulation data, http://bit.ly/fwgh8k
4 ``ReachLocal Reports 44\% Annual Revenue Growth for 2010,'' ReachLocal website,
Feb. 15, 2011. http://bit.ly/fdkpSK
5 ``Local Media Reach: 2010,'' Internet Broadcasting, Jan. 26, 2011. http://bit.ly/eWo2SF
6 Sarah Jane Weaver, ``Deseret News set to lead, innovate,'' Deseret News, Sept. 1, 2010.
http://bit.ly/ewGLJy
7 Ibid.
8 Tom Harvey, ``Tribune to press ahead in face of News changes,'' Salt Lake Tribune,
Aug. 31, 2010. http://bit.ly/es3QnY
9 KSL.com firearm survey was active in March and April 2011 at http://bit.ly/hO809N
10 ``DMC unveils new digital media and broadcast operating divisions,'' KSL.com, Sept. 10, 2009.
http://bit.ly/fQkmjg
11 ``Newspaper Next: Blueprint for Transformation,'' American Press Institute, 2006.
http://bit.ly/iewNO8
12 Felicity Barringer, ``A General Whose Time Ran Out,'' New York Times, March 15, 2000.
http://nyti.ms/hRArcp
13 ``Disrupting Class: How Disruptive Innovation Will Change the Way the World Learns,''
Dec. 3, 2009. http://bit.ly/hPlWln
14 ``Clark Gilbert's Session at Borrell Associates' Local Mobile Advertising Conference
2010—Dallas, TX,'' Borrellassociates.com, http://bit.ly/fBlSFD; Gilbert speaks about KSL.
com's revenue around 14:30 into the video. The figure has since been updated by the company.
15 CareerBuilder: Profile—History. http://cb.com/fhfMTe
16 Press Release, ``McClatchy Reports Fourth Quarter 2010 Earnings,'' Feb. 8, 2011.
http://bit.ly/hsfERQ
Columbia Journalism School | Tow Center for Digital Journalism
New Users, New Revenue: Alternative Ways to Make Money 117
17 ``Facebook advertising metrics and benchmarks,'' go-Digital Blog. http://bit.ly/gMwxsw
18 Geoffrey A. Fowler and Emily Steel, ``Valuing Facebook's Ads,'' WSJ.com, Nov. 11, 2010,
http://on.wsj.com/faZueN Advertisers have posted click-through stats on a Facebook chat
page. The comments from 2010 and 2011 show varied results: One user said his ad drew 674
clicks out of 320,000 impressions, for a rate of about one click per 475 impressions. Another
got one click in 2,480 impressions. Another was one per 1,818. See ``Share your experience
with Facebook ads (CTR, CPM, etc),'' Facebook chatboard. http://on.fb.me/gfd0wI
19 Matthew Creamer, ``Think Different: Maybe the Web's Not a Place to Stick Your Ads,''
Advertising Age, March 17, 2008. http://bit.ly/fQrJwV
20 Sean Corcoran, ``Defining Earned, Owned And Paid Media,'' Forrester Blogs, Dec. 16, 2009.
http://bit.ly/dMR8sE
21 ``FTC Publishes Final Guides Governing Endorsements, Testimonials,'' Federal Trade
Commission website, Oct. 5, 2009. http://1.usa.gov/hh1meg
22 ``Twitter Users and Bloggers Open to More Than Earned Media,'' eMarketer, Sept. 22, 2010.
http://bit.ly/feLdMh. Original study was sent to authors by Izea.
23 Christopher Steiner, ``Meet The Fastest Growing Company Ever,'' Forbes, Aug. 30, 2010.
http://bit.ly/h0qQwf
24 Ryan Singel, ``Startup Hits Sweet Spot for Selling Local Services,'' Wired.com, Nov. 4, 2009.
http://bit.ly/egnGnt
25 ``Groupon Faces More Criticism After Flower Deal, Unhappy Restaurants,''
MyFoxChicago.com, Feb. 15, 2011. http://bit.ly/e62Kg7
26 Sarah Hartenbaum, ``Deals Galore, Competitors Abound: A Primer On Groupon-Like
Startups,'' TechCrunch.com, July 11, 2010. http://tcrn.ch/hfAC5V
27 David Kaplan, ``McClatchy Makes A Deal With Social Shopper Groupon,'' paidcontent.org,
July 1, 2010. http://bit.ly/fXe9Q2.
28 Kunur Patel, ``McClatchy, a Groupon Partner, Starts Selling Its Own Daily Deals, Too,'' AdAge,
March 23, 2011. http://bit.ly/fs4tgt
29 Evelyn M. Rusli and Andrew Ross Sorkin, ``Groupon Advances on I.P.O. That Could Value
It at \$15 Billion,'' NYTimes.com, Jan. 13, 2011. http://nyti.ms/gBo4xL
30 Photo gallery of 2010 Wired store: http://bit.ly/gshwh6


\section{Endnotes: Chapter 9}
%<h3 id="endnotes-chapter-9">Endnotes: Chapter 9</h3>
1 ``January 2011: Top U.S. Web Brands and News Sites'' Nielsen, Feb. 11, 2011.
http://bit.ly/evjGr6 and Compete.com, HuffingtonPost site analytics, April 22, 2011.
2 Technorati.com, Huffington Post is the most influential blog of the 1.2 million tracked.
April 22, 2011.
3 Speech, Harvard Business School Club of New York Media Guru breakfast, April 5, 2011.
The Story So Far: What We Know About the Business of Digital Journalism
126
4 Lewis DVorkin, ``9 big steps in 9 short months, now Forbes is building The New Newsroom,''
Forbes, March 1, 2011. http://bit.ly/fp1lwk
5 An example of this treatment is an article written by Eric Lai of SAP, ``Video Killed the
Radio Star, But Smartphones Did NOT Kill the Flip Cam,'' Forbes.com, April 14, 2011.
http://bit.ly/glYrnr
6 ``ASME Guidelines for Editors and Publishers,'' Magazine.org, Jan. 25, 2011.
http://bit.ly/e3UZkS
7 Speech, Harvard Business School Club of New York Media Guru breakfast, April 5, 2011
8 Jessica Vascellaro, ``Remaking AOL in Huffington's Image,'' Wall Street Journal, April 7, 2011.
http://on.wsj.com/heuDE7
9 Nicholas Carlson, ``Leaked: AOL's Master Plan,'' Business Insider, Feb. 1, 2011.
http://read.bi/eTBJBJ
10 The AOL Way, January, 2011, page 28.
11 The AOL Way, January, 2011, page 17.
12 Ibid.
13 The AOL Way, January, 2011, slide 18.
14 Jennifer Barton, `` Lady Gaga Goes Pantless in Paris,'' Stylelist, Dec. 10, 2010.
http://aol.it/eTvVbJ
15 The AOL Way, January, 2011, slide 33
16 About.com guides get paid by the page view but do have a minimum guarantee. Dan Frommer,
``About.com Cutting 10\% Of Staff, Pay Cuts For Guides,'' Business Insider, Feb. 5, 2009.
http://read.bi/i4NNV0
17 Jim Romenesko, ``Report: Gannett to give page view bonuses to writers,'' Poynter,
April 7, 2011. http://bit.ly/dE9p6f
18 Walter Winchell famously left the Evening Graphic and later joined the Daily Mirror in 1920s
New York.
19 Details of the site's financial information:
Details of family-oriented magazine's website financials
20 When LIN reports its ``digital revenue,'' it combines new-media sales with retransmission fees
from cable companies in the sum. Based on discussions with Barry Lucas, a senior vice president
at investment firm Gabelli & Co., who covers LIN, the authors estimate that half of the
``digital revenue'' represent new-media sales and half is retransmission fees.
